\documentclass{article}

\usepackage[italian]{babel}
\usepackage{listings}
\usepackage{hyperref}

\title{Introduzione LFS}
\author{Davide Luci}
\date{20/12/2024}

\begin{document}
\section{Introduzione}
    \subsection{Partizionamento}
        \setlength{\parindent}{0cm}
        Una partizione da 10GB basta per contenere i file sorgenti e il compilato.
        Comunque se si vogliono aggiungere funzionalita e' consigliabile avere una partizione da 30GB.
        E' utile anche attivare una partizione di \texttt{swap}, e con
        \texttt{free} e' possibile vedere quanta swap e' usata.

        I tool consigliati per partizionare sono: \texttt{cfdisk} e \texttt{fdisk}.
        La partizione di root dovrebbe essere di circa 20GB. Per i dischi partizionati con GPT
        il bootloader ha bisogno di una partizione di 1MB, questa su \texttt{fdisk} si chiama \texttt{BIOS Boot}.
        E' raccomandata l'instsallazione di una partizione dedicata a \texttt{/boot}, dalla grandezza di
        circa 200MB. 

        Si potrebbe considerare in oltre di creare partizioni per:
        \begin{itemize}
            \item \texttt{/home}: Per condividere le configurazioni degli utenti tra piu sistemi.
            \item \texttt{/usr}. Nella cartella radice, riesederanno \texttt{/bin}, \texttt{/lib}, \texttt{/sbin}, questi sono dei symlink per
            le omonime cartelle in \texttt{/usr}.
        \end{itemize}

        Formatteremo le partizioni usando ext4 in questo modo:
        \noindent
        \begin{lstlisting}[language=bash]
            $ mkfs -v -t ext4 /dev/<xxx>
        \end{lstlisting}

        E' utile impostare la variabile d'ambiente  LFS. Questa contiene il path che porta all punto di mount della radice 
        del sistema LFS.
        \begin{lstlisting}[language=bash]
            $ export LFS=/mnt/lfs
        \end{lstlisting}
    \subsection{Montare la partizione}
        \begin{lstlisting}[language=bash]
            $ mkdir -pv $LFS
            $ mount -v -t ext4 /dev/<xxx> $LFS
        \end{lstlisting}

        Per montare eventualmente altre partizioni:
        \begin{lstlisting}[language=bash]
            $ mkdir -v $LFS/home
            $ mount -v -t ext4 /dev/<yyy> $LFS/home
        \end{lstlisting}
    
\section{Pacchetti e Patch}
    Bisogna creare la cartella dove risiederanno i pacchetti compressi in tarball, e bisogna eseguire 
    questi comandi come \texttt{root}.
    \begin{lstlisting}[language=bash]
        $ mkdir -v $LFS/sources
        $ chmod -v a+wt $LFS/sources
    \end{lstlisting}.

    Per ottenere tutti i pacchetti e le patch useremo \href{https://www.linuxfromscratch.org/lfs/downloads/wget-list-sysv}{wget-list-sysv}
    come input per il comando:
    \begin{lstlisting}[language=bash]
        $ wget --input-file=wget-list-sysv --continue --directory-prefix=$LFS/sources
    \end{lstlisting}.

    Sarebbe anche una buona cosa verificare le chiavi md5 (sti cazzi).


\section{Ultimi passagi per la preparazione.}
    \subsection{Cartelle!}
        Come utente root:
        \begin{lstlisting}[language=bash]
            # crea le cartelle
            mkdir -pv $LFS/{etc,var} $LFS/usr/{bin,lib,sbin}

            # fai i link simbolici per bin, lib e sbin
            for i in bin lib sbin; do
                ln -sv usr/$i $LFS/$i
            
                case $(uname -m) in
                    x86_64) mkdir -pv $LFS/lib64 ;;
                esac
        \end{lstlisting}.
    \subsection{Utente LFS}
        Dobbiamo creare il gruppo lfs e il suo utente lfs.
        \begin{lstlisting}[language=bash]
            $ groupadd lfs
            $ useradd -s /bin/bash -g lfs -m -k /dev/null lfs
        \end{lstlisting}

        Sarebbe anche un'ottima idea impostare una password:
        \begin{lstlisting}[language=bash]
            $ passwd lfs
        \end{lstlisting}
    \subsection{Set up dell'Ambiente}
        Dopo aver effettuato l'accesso come \texttt{lfs} (usando il comando \texttt{su})
        \begin{lstlisting}[language=bash]
            cat > ~/.bash_profile << "EOF"
            exec env -i HOME=$HOME TERM=$TERM PS1='\u:\w\$ ' /bin/bash
            EOF
        \end{lstlisting}

        La shell bash dopo aver effettuato il login, legge il file \texttt{ .bash_profile} che dovrebbe contenere
        per lo piu le configurazioni delle variabili d'ambiente.

        Il codice inserito crea una nuova bash dove sono state resettate le principali variabili d'ambiente. 
        Questa in automatica andra' ad eseguire il file \texttt{.bashrc} che andremo a popolare con:

        \begin{lstlisting}[language=bash]
            cat > ~/.bashrc << "EOF"
            set +h
            umask 022
            LFS=/mnt/lfs
            LC_ALL=POSIX
            LFS_TGT=$(uname -m)-lfs-linux-gnu
            PATH=/usr/bin
            if [ ! -L /bin ]; then PATH=/bin:$PATH; fi
            PATH=$LFS/tools/bin:$PATH
            CONFIG_SITE=$LFS/usr/share/config.site
            export LFS LC_ALL LFS_TGT PATH CONFIG_SITE
            EOF
        \end{lstlisting}

\end{document}