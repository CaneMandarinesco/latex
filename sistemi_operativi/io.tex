Ci sono due prospettive quando si parla di IO:
\begin{itemize}
    \item \textbf{Ingegneri Elettronici}:
    \item  \textbf{Programmatori}
\end{itemize}

Ci sono due categorie di dispositivi IO:
\begin{itemize}
    \item \textbf{Dispositivi a Blocchi}: riguardano il trasferimento di dati in sequenze di lughezza fissa.
    \item \textbf{Dispositivi a Caratteri}: riguardano il trasferimento su una stream di dati, i dati non arrivano in modo strutturato.
    \item \textbf{Ibridi}
\end{itemize}

I dispositivi a blocchi sono gestiti dal file system.  \\
Bisogna tenere presente che il sistema operativo deve essere in grado di gestire
contemporaneamente dispositivi che hanno velocita di lettura/scrittura differenti tra loro:
una tastiera scrive di solito $10byte/s$, un mouse $100byte/s$, USB 3.0 arriva a $650MB/s$ mentre 
PICIe 6.0 a $128GB/s$.

C'e' una differenza importante tra una tastiera e una perificarica PCIe, infatti sulla scheda madre
queste periferiche sono gestite da due chip differenti:
\begin{itemize}
    \item \textbf{Northbridge}: collega la CPU ai bus piu' veloci, come DDR per la RAM e PCIe.
    \item \textbf{Southbridge}: collega la CPU ai bus piu' lenti, come USB, Bluetooth, Ethernet.
\end{itemize}

Il \textbf {Controller del dispositivo} si puo' trovare sulla scheda madre o sul dispositivo stesso.
A seconda del dispositivo, questo utilizza una standard che puo' essere: ANSI, IEEE, ISO oppure 
uno \textbf{standard de facto} come SATA, SCSI, USB e ThunderBolt. Quelle de facto sono interfacce create
da aziende e che avendo avuto successo, sono richieste sul mercato. \\

Nei dispositivi a blocchi, il controller non fa altro che leggere uno stream di dati dal dispositivo 
per poi strutturarlo in un blocco. Il buffer dei dati letti dal dispositivo viene 
messo in un \textbf{buffer} del controller che tipicamente e' una vera e propria RAM.

\paragraph{Hard Disk}
E' composto da:
\begin{itemize}
    \item MCU.
    \item Memoria Cache di tipo DDR SDRAM, da usare come buffer.
    \item VCM: gestisce la rotazione del disco e consuma corrente.
    \item Chip Flash: il software che gestisce l'hard disk.
    \item Accelerometro: in caso di urti la testina si blocca. 
\end{itemize}

\subsection{Port-Mapped IO}
Lo spazio degli indirizzi del controller e' completamente estraneo allo spazio virtuale
del computer. Si usano istruzioni \texttt{IN} e \texttt{OUT} per indirizzare i registri
del controller e impartire quindi dei comandi.

\subsection{Memory-Mapped IO}
Gli indirizzi di memoria dei registri sono mappati alla memoria virtuale.

Questo metodo e' efficiente per i compilatori:
il Port-Mapped IO prevede che il programmatore specifichi che l'indirizzo fornito sia esclusivamente per IO, 
mentre con Memory-Mapped cio' non succede, l'indirizzo fornito non puo' appartentere ad altri spazi virtuali. \\ 

Se si usa il \textbf{paging} per mappare l'IO in memoria ci sono altri vantaggi: 
con la tabella delle pagine posso regolare l'accesso al dispositivo, e posso facilmente
effettuare un link dei driver necessari per utilizzare il dispositivo.

Il problema principale di questo approccio e' il caching e il direttamente delle 
informazioni: la cache deve essere disabilitata per gli indirizzi IO, e in oltre 
bisogna capire quando un'indirizzo e' IO o non, ho due possibilita:
\begin{itemize}
    \item scrivo prima in memoria, se fallisco provo l'IO. Meh
    \item un dispositivo (\textbf{snooper}), intercetta l'indirizzo
\end{itemize}

Ci sarebbe anche un'ibrido tra i due: PMIO viene usato per configurare il dispositivo, e il MMIO viene usato per leggere e scrivere.

\subsection{DMA}
La cpu potrebbe richiedere i dati dal controller byte per byte, ma cosi spreca troppe istruzioni.
Il controller DMA ci aiuta: ha accesso al bus di sistema indipendentemente dalla CPU. Il controller DMA deve essere impostato dalla CPU, cosi
ha i parametri inizializzati per capire dove scrivere, poi quando viene richiesta una lettura, il DMA preleva le parole dal
controller e le mette in RAM. A fine lettura, il DMA manda un interrupt alla CPU.

Il DMA ha due modi per scrivere in RAM:
\begin{itemize}
    \item Cycle Stealing: quando il controller DMA preleva una parola, ottiene priorita sul bus e sfancula la CPU.
    \item Burst: quando il controller DMA deve scrivere dei blocchi, richiede il bus, fa tutte le scritture, e lo rilascia.
    \item Fly-by mode: il DMA dice al controller di scrivere direttamente in memoria, senza intermediari.
\end{itemize}

Con l'introduzione del DMA e' ancora necessario il buffer interno del controller? Si, perche'
prima di trasferire i dati verso il DMA, devo assicurarmi che la velocita' di trasferimento sia costante.
Nel caso di un hard disk, dove la testina ha bisogno di muoversi, la velocita di lettura non e' detto che sia costante.

\subsection{Interrupt}
Ci sono 3 tipi di interrupt in un computer:
\begin{itemize}
    \item \textbf{trap}: che corrispondono alle chiamate di sistema
    \item \textbf{fault e eccezioni}: corrispondono alla gestione di errori
    \item \textbf{interrupt hardware}: sono quelli che vengono utilizzati principalmente dai dispositivi di IO.
\end{itemize}

Il \textbf{Vettore degli interrupt} tiene in memoria gli indirizzi delle procedure da eseguire per ogni 
interrupt.
Ma dove metto le informazioni relative all'esecuzione del programma interrotto? Posso usare
i registri hardware oppure lo \textbf{stack}, che puo' essere \textbf{utente} o del \textbf{kernel}.
Il stack kernel e' quello piu' sicuro da usare, probabilmente il puntatore allo stack del kernel 
non e' errato e c'e' spazio per inserire i dati, ma bisogna generare una trap per cambiare contesto. 
In oltre cambiare contesto implica che la TLB sia invalidata. // 
Altrimenti nello spazio utente, ma ci sono piu' probabilita' che il puntatore allo stack sia errato, o che 
la pagina sia quasi completa. 

Bisogna anche tenere conto di come gestire gli interrupt: \textbf{precisi} o \textbf{imprecisi}?
Gli interrupt \textbf{precisi} garantiscono che le istruzioni vengano completate prima di gestire l'interrupt.
Gli interrupt \textbf{imprecisi} invece bloccano immediatamente l'esecuzione dell'istruzione nella pipeline, e ne
salvano lo stato per poi riprendere l'esecuzione dell'istruzione da dove era stata interrotta. 

\section {Principi del software di io}

Il software per l'io opera secondo i seguenti principi:
\begin{itemize}
    \item Il software dovrebbe essere indipendente dal dispositivo.
    \item \texttt{everything is a file}: con la chiamata \texttt{write} posso leggere
    i dispositivi io.
    \item gestione degli errori fisici di trasmissione
    \item trasferimenti sincroni e asincroni.
    \item \texttt{buffer}: e' fondamentale, i dati possono non arrivare in ordine e bisogna ricostruire la sequenza originale.
    Aiuta a prevenire la latenza in oltre.
    \item I dispositivi sono condivisi tra piu' utenti.
\end{itemize}

\subsection{Io Programmato}
L'io programmato prevede che l'utente comandi i dispositivi, mentre il sistema 
operativo offre l'interfaccia.

Quando l'utente vuole stampare una stringa, questa si trova in un buffer nello spazio utente.
La chiamata di sistema per la stampa trasferisce la stringa nello spazio del kernel.
La cpu si trova nello stato di \textbf{busy waiting}, copia i dati carattere per carattere 
nella stampante e deve aspettare ad ogni iterazione che la stampante sia pronta per accettare un carattere.

\subsection{Io guidato da interrupt}
Quando la stampante ha finito di stampare il carattere inviato, viene generato un interrupt.
Il sistema operativo cattura l'interrupt e copia il carattere successivo nella stampante, e chiama lo scheduler.
Quando ho finito di copiare i caratteri, il gestore degli interrupt viene disabilitato, e il controllo viene passato 
al processo utente.

\subsection{Io con DMA}
Il problema del metodo guidato da interrupt e' che oltre a cambiare utente, bisogna passare allo spazio
del kernel per ogni singolo carattere.
Piuttosto conviene sfruttare il controller DMA, copio i dati al suo interno e una volta che 
il trasferimento e' completato la stampante invia un interrupt.

I controller DMA possono essere complessi da gestire, spesso hanno piu' canali per gestire piu' 
scritture insieme.
