Bisogna andare in contro ai problemi della memorizzazione:
\begin{itemize}
    \item La RAM ha uno \textbf{spazio limitato}, e le sue informazioni si perdono.
    \item Devo garantire l'\textbf{accesso concorrente} alle informazioni.
\end{itemize}

La memorizzazione a lungo termine deve:
\begin{itemize}
    \item Gestire tante informazioni.
    \item Devono persistere sul disco.
    \item Deve essere condivisa tra processi.
\end{itemize}

\subsection{File System}
Un disco e' una sequenza lineare di \textbf{blocchi}: posso scrivere o leggere su \texttt{k}.
L'astrazione del file risove tutti i problemi elencati sopra, il sistema operativo quando gestisce un file
si occupa della: \textbf{struttura, denominazione, accesso, protezione e implementazione.}

Un file system ha informazioni visibili all'utente e altre nascoste (come la gestione della memoria e la sua struttura).

\begin{definition} File System \\
    Un modo per \textbf{organizzare} e memorizzare le informazioni. \\
    Sono un'astrazione per i dispositivi: Dischi Rigidi, SSD, ecc...
\end{definition}

\subsection{Nomi dei file}
I file sono l'astrazione che permettono di salvare e leggere su disco, nascondendo i dettagli tecnici all'utente.
\begin{itemize}
    \item \textbf{Nomenclatura dei file}: I file sono identificati tramite nomi (le possibilita dipendono dal file system)
    \item \textbf{Lunghezza}: a seconda del sistema operativo il nome di un file potrebbe avere una lunghezza massima.
    \item \texttt{ext4}: non posso usare certi nomi speciali. \texttt{FAT12} non posso usare alcuni caratteri.
    \item \textbf{Case Sensitive?}
\end{itemize}

L'estensione di un file puo' essere convenzionale, come per esempio in UNIX. In Windows invece 
l'estensione specifica quale programma puo' aprire il file.

\subsection{Struttura Di Un File}
\paragraph{Sequenza Non Strutturata Di Byte}
In questo caso i file sono una serie di byte senza significato, i programmi utente li devono interpretare: ricade nel caso di 
\textbf{UNIX, Linux, MacOS e Windows}.

\paragraph{Sequenza di Record di Lunghezza Fissa}
Il modello storico basato sulle vecchie schede perforate

\paragraph{Fle come Albero di Record}
Il file e' organizzato come un'albero

\subsection{Tipi di File}
In Unix e Windows esistono \textbf{File Normali} e \textbf{Directory}, le directory sono file che mantentono
una gerarchia dei file. \\
Ci sono due tipi di file \textbf{speciali}:
\begin{itemize}
    \item \textbf{file a caratteri}: per gestire dispositivi \textit{I/O}.
    \item \textbf{file speciali a blocchi}: per gestire i \textit{dischi}.
\end{itemize}

I File Normali possono essere interpretati come:
\begin{itemize}
    \item \textbf{ASCII}: file di testo.
    \item \textbf{Binari}: non leggibili come testo, vanno interpretati dai programmi.
\end{itemize}

\paragraph{File Eseguibilie}
Un file eseguibile e' una sequenza di byte ben strutturata, le sue \textbf{componenti} sono:
\begin{itemize}
    \item \textbf{Numero Magico}: il file e' eseguibile?
    \item \textbf{Intestazione/Header}: Punto di ingresso, dimensioni di ogni componente del file.
    \item \textbf{Testo e Dati}: Le parti del programma da caricare in memoria
    \item \textbf{Tabella dei simboli}: usata per il debug.
\end{itemize}

\paragraph{Archivio}
Sono file binari (interpretati come testo non avrebbero senso), che raccolgono procedure di libreria compilate
che devono essere collegate.

\paragraph{Accesso ai file}
L'accesso per via dell'hardware era \textbf{sequenziale} (si leggevano dei nastri magnetici).
Con l'avvento dei dischi si poteva leggere con \textbf{accesso casuale}.

Per leggere posso indicare una posizione specifica, dove inzia la lettura oppure con \texttt{seek}
posso impostare la posizione nel file corrente che sto leggendo: implementato sia un Unix che Windows.

\paragraph{Attributi comuni dei file.}
\begin{itemize}
    \item \textbf{Protezione e Accesso}: chi puo accedere al file? Posso leggere, scrivere e eseguire?
    \item Flag di controllo: sola lettura, nascosto, file di backup.
    \item Binario o ASCII?
    \item Attributi Temporali: Data e Ora, Ultimo accesso, Ultima modifica
    \item Dimensione.
\end{itemize}

\subsection{Programmazione!}
\subsubsection{Leggere un file}
\begin{lstlisting}[language=c]
    int fd = open("foo.txt", O_RDONLY);
    char buf[512];
    ssize_t bytes_read = read(fd, buf, 512);
    close(fd);
    printf("read %zd: %s\n", bytes_read, buf);
\end{lstlisting}

\subsubsection{\texttt{lseek}}
\begin{lstlisting}[language=c]
    int fd = open("foo.txt", O_RDONLY);
    lseek(fd, 128, SEEK_CUR);
    char buf[8];
    read(fd, buf, 8);
    close(fd);
\end{lstlisting}

Sposta di 128 byte in avanti la posizione nel file, a partire dalla posizione corrente.

\subsubsection{Scrivere}

\begin{lstlisting}[language=c]
    int fd = open("foot.txt", O_WRONLY |
         O_CREAT | O_TRUNC);
    char buf[] = "Dio Porco!";
    write(fd, buf, strlen(buf));
    close(fd);
\end{lstlisting}

La flag \lstinline{O_TRUNC} indica che se il file esiste lo ridimensiono a 0. 

\subsection{Directory}
\begin{definition} Directory \\
Sono file che tengono traccia degli altri file nel file system. 
\end{definition}

In generale una singola directory, denominata \texttt{root}, contiene tutti gli altri file: organizzazione a \textbf{Singolo Livello}.
L'organizzazione in directory offre semplicita e rapidita nella locazione di un file. 

E' da notare che l'uso dei file come astrazione e' ampiamente utilizzata: dispostivi embedded come fotocamere e lettori mp3, e tecnologie \textbf{RFID}.
L'idea dei file e' antica, ma si rinnova ad ogni generazione.

L'organizzazione a singolo livello non e' pratica per noi umani che teniamo vogliamo tenere in memoria
migliaia di file: si introduce una struttura ad albero per organizzare logicamente i file.

Di solito ogni utente ha una sua cartella privata per i suoi dati.

I percorsi che portano ad un file possono essere:
\begin{itemize}
    \item \textbf{Assoluti}: utile per caricare le procedure di libreria
    \item \textbf{Relativi}: e' legata alla \textbf{directory di lavoro}
\end{itemize}

Le operazioni possibili su una directory sono:
\begin{itemize}
    \item \texttt{create}: crea una directory vuota.
    \item \texttt{delete}: elimina la directory solo se e' vuota.
    \item \texttt{opendir}: apri e leggi il contenuto.
    \item \texttt{closedir}: chiudila e libera le risorse.
\end{itemize}

L'istruzione \texttt{readdir}, restituisce le prossima voce in una directory aperta.

\subsection{Archiviazione}
I file TAR (Tape Archive) servono per raccogliore piu file in un unico archivio, mantenendone la struttura
e i permessi originali. 

Di solito, dopo aver archiviato dei file, si comprimono con \texttt{gz}: il suo spazio viene ridotto.

\begin{lstlisting}[language=bash]
    tar -czvf nome-archivio.tar.gz /percorso/della/cartella
\end{lstlisting}
Dove:
\begin{itemize}
    \item \texttt{c}: crea un nuovo archivio
    \item \texttt{z}: comprimi usando l'algoritmo gzip
    \item \texttt{v}: verbose
    \item \texttt{f}: specifica il nome del file di archivio
\end{itemize}

Per "aprire" un archivio tar compresso con gzip, invece:
\begin{lstlisting}[language=bash]
    tar -xzvf nome-archivio.tar.gz
\end{lstlisting}
Simile a prima, l'opzione \texttt{x} indica che voglio estrarre. 

Altri comandi e formati per comprimere e creare archivi:
\begin{itemize}
    \item \texttt{zip}: riduce la dimensione dei file singolarmente e poi li archvia, e' ampiamente supportato.
    \item \texttt{unzip}: decomprime e estrare gli archivi \texttt{zip}
    \item \texttt{tar.gz}: raccoglie i file in un unico archivio e lo comprime. La compressione e' elevata, ed e' in grado di mantenere la sua struttura e i permessi
\end{itemize}

In conlusione \texttt{tar.gz} ha un tasso di compressione piu elevato rispetto a \texttt{zip}, ma quest'ultimo e' piu veloce nel crearli.
Tutta via \texttt{tar.gz} mantiene i meta dati dei file, mentre \texttt{zip} e' supportato su diverse piattaforme.