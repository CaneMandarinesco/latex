\documentclass{article}

\usepackage[italian]{babel}
\usepackage{listings}
\usepackage{amsmath}
\usepackage{subfiles}

\title{Riassuntazzo Sistemi Operativi}
\author{Davide Luci}
\date{2024}

\newtheorem{definition}{Definition}[section]

\begin{document}

    \maketitle
    \pagenumbering{gobble}
    \tableofcontents
    \newpage

    \pagenumbering{arabic}

    \setlength{\parindent}{0pt}
    \section{ Prima di leggere...}
    Questo documento non e' esaustivo al fine di comprendere al pieno la materia trattata.
    Le definizioni sono scarne, e spesso non esaustive, perche' ho preferito solo appuntare i concetti chiavi
    da ripassare in modo da trovare una giusta via di mezzo tra studiare e riassumere. Si puo' dire che sia compito del 
    lettore, quello di arricchire il contenuto di questo documento.

    \newpage

    \section*{ Introduzione}
    Diocane

    \section { Esercizio Countdown}
    \begin{lstlisting}[language=c]
        #include <stdio.h>
        #include <unistd.h>
        #include <stdlib.h>
        #include <time.h>

        int main() {
            // Inizializza il generatore di numeri casuali
            srand(time(NULL));

            for (int i = 20; i >= 1; i--) {
                printf("%d\n", i);
                fflush(stdout);  // Assicura che l'output venga visualizzato immediatamente

                // Dormi per un tempo casuale tra 0 e 2 secondi
                sleep(rand() % 3);
            }

            printf("\nBYE\n");
            fflush(stdout);

            return 0;
        }
    \end{lstlisting}
    
    \end{document}