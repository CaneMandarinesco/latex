\subsection{Come memorizzare i file?}
\begin{definition} File System \\
    Il file system e' il metodo utilizzato per organizzare e memorizzare dati sui dispositivi di memoria.
\end{definition}

\begin{definition} Partizioni \\
    Un disco puo' essere suddiviso in piu' partizioni, ciascuna con un proprio file system, \textbf{indipendente}.    
\end{definition}

\subsubsection{MBR - Master Boot Record}
L'MBR si trova nel settore 0 del disco ed e' essenziale per avviare il computer, contiene
la tabella delle partizioni, dove ogni \texttt{entry} ha informazioni su inizio e fine della partizione.

All'avvio del sistema il \textbf{BIOS} \textit{legge ed esegue l'MBR} che trova la partizione \textbf{attiva}, quella da usare per caricare il \textbf{boot block}
per avviare il sistema operativo.

\begin{quote}
    Ogni partizione inizia con un \textbf{boot block} anche se non ce l'ha!
\end{quote}

\subsubsection{Nuova scuola: UEFI (Unified Extensible Firmware Interface)}
E' un sistema che sostituisce il vecchio BIOS. Permette:
\begin{itemize}
    \item Avvio piu veloce: inizializzazione dell'hardware ottimizato.
    \item Comptabilita architetture 32 e 64 bit.
    \item Supporto di Interfaccia utente avanazata e mouse.
    \item Secure Boot
\end{itemize}

UEFI funziona con GPT: posso avere dischi piu grandi di \texttt{2,2 TB}.
\paragraph{GUID Partition Table (GPT)}
E' un sistema avanza di gestione delle partizione, supporta dischi enormi, 
puo' gestire un numero illimitato di partizioni (limitato dal sistema operativo che eseguiremo), include un sistema
di \textbf{backup della tabella delle partizioni} e fa anche il \textbf{CRC}. \\

Nei dischi GPT esiste la \textbf{EFI System Partition} (ESP), una partizione speciale che contiene i file di avvio: bootloader e driver, utilizza
il file system \textbf{FAT32} in modo da essere compatibile con UEFI. 

\paragraph{Secure Boot}
Impedisce l'avvio di software non autirizzato: questo software controllare le firme digitali
del \textit{boot loader, del sistema operativo e dei driver}, in modo da avviare solo software con firme
autentiche: \textit {previene l'esecuzione di malware e rootkit all'avvio}.

Lo svantaggio e' che alcuni sistemi operativi le cui firme non sono riconosciute potrebbero non avviarsi.

\subsection{Implementazione dei file}
Bisognia gestire l'associazione tra \textit{file e blocchi sul disco}. Ci sono vari tipi di allocazione 
per i file.

\subsubsection{Allocazione contigua}
I file sono memorizzati come sequenze contigue sul disco: \textit{un file da 50 blocchi, occupera 50 blocchi
consecutivi sul disco}.
Questo metodo e' semplice da implementare ed ha un'alta efficienza di lettura, con una sola lettura posso leggere il file
senza dover cambiare posizione durante la lettura (meglio per i dischi meccanici).  \\

Si ha pero' un problema analogo a quello della RAM: come gestisco la frammentazione? Potrebbe essere 
difficile trovare spazi per infilare i miei file, oppure anche se c'e' abbastanza spazio e' talmente frammentato
da non avere blocchi continui a disposizione.

\subsection{Allocazione a Liste Concatentate}
i file sono \textit{liste concatenate di blocchi sul disco}, ogni blocco ha un \textbf{puntatore} al blocco successivo. \\

Permette di usare meglio lo spazio sul disco, posso facilmente implementare la struttura a directory ma:
\begin{itemize}
    \item richiede di fare molti accessi casuali (devo chiamare \texttt{seek}).
    \item parte dei bit nel blocco sono usati per gestire i puntatori, quindi un blocco potrebbe non contenere dati che occupano una \textbf{dimensione standard}, ossia una potenza di 2. 
\end{itemize} 

\subsection{FAT}
Uso una File allocation table, caricata in RAM! Questa contiene le sequenze di blocchi che compongono un file.
E' efficiente ma per dischi troppo grandi si usa troppa RAM, la tabella e' allocata interamente per tutti i blocchi, anche
quelli inutilizzati, si usa piu che altro per chiavette usb e schede sd di piccole dimensioni.

\subsection{I-NODE}
\begin{definition} I-NODE
    Un Index-Node e' una struttura dati fondamentale in \texttt{ext2/ext3/ext4} (usato in Linux). 
    Contiene le informazioni su un file, \textit{tranne il nome e il contenuto}, ma ne comprende i metadati.
\end{definition}

Ogni file e directory e' rappresentato da un i-node, questi sono organizzati in una tabella. 
La tabella FAT rispetto a quella degli i-node non permette di gestire in modo ottimale i meta dati (altrimenti la grandezza della tabella in RAM crescerebbe).
\textit{I sistemi i-node sono efficienti su dischi grandi}. \\

Un'i-node e' una struttura, di \textbf{dimensione fissa}, che contiene attributi e \textit{gli indirizzi a blocchi} di un file, ad ogni i-node corrisponde un file.
\textit{Rispetto alla FAT, solo gli i-node dei file aperti si trovano in RAM.} \\

Dato che gli i-node hanno dimensione fissa, per i file che hanno tanti blocchi, l'ultimo indirizzo di un i-node potrebbe
puntare per esempio ad un \textit{blocco contentente altri puntatori}, posso gestire efficaciemente file grandi. \\

\subsection{Implementazione Delle Directory}
Le directory mappano i nomi ASCII dei file sulle informazioni per trovare i dati sul disco (primo blocco da leggere o i-node corrispondente).

Come alloco i dati sulla directory? Posso avere un tipo di associazione: \texttt{nome file|attributi} oppure \texttt{nome file|i-node}.
Ma i nomi dei file hanno lunghezze variaibile, di solito si imposta un limite massimo di caratteri: 255. Ma pochi file usano tutti e 255 
i caratteri quindi e' saggio usare delle voci con \textbf{lunghezza variabile}:
\begin{itemize}
    \item Ogni voce ha un header che dice quanto e' lungo il nome del file. \textit{Ma quando elimino file e cartelle si creano buchi di lunghezza variabile!}
    \item Il nome dei file e' contenuto in un \textbf{heap}, subito dopo la fine della directory. Cosi ogni entry ha lunghezza fissa.
\end{itemize}

Ora sorge un'altro problema: la ricerca del file in una directory e' lineare. Usiamo quindi una
tabella di \texttt{hash}, quindi se ho $n$ file, per ognuno di questi grazie ad una funzione di hash
sono in grado di accedere in tempo costante alla lista.

Se a piu file corrisponde lo stesso indice dopo che ho applicato l'hasing, uso delle liste concatenate.
Per velocizzare le cose posso usare una cache che e' efficiente se richiedo un piccolo insieme di file, bisogna
notare che stiamo \textit{aggiungendo complessita al nostro problema}, magari per directory piccole non serve
fare tutto sto casino.

\subsection{Link}
I file possono essere condivisi tra piu utenti tramite:
\begin{itemize}
    \item Hard Link: file che puntano \textbf{all'i-node} condiviso.
    \item Soft Link: file che puntano \textbf{al nome} di un file condiviso.
\end{itemize}

L'\textbf{Hard Link} e' ideale quando ho molti proprietari, l'i-node viene eliminato quando non ci sono piu'
Hard Link a questo, indipendentemente dal numero di Hard Link ho comunque un solo i-node.
L'eliminazione di un i-node si puo' effettuare quando per esempio il contatore di hard link a questo va a 0. \\

Il \textbf{Soft Link} e' piu' rognoso. Per ogni soft link ho un i-node che punta al percorso del file condiviso.

I link introducono un problema: se faccio il backup del disco e trovo i link che devo fa?

\subsection{Spazio del Disco}
Come tengo traccia dello spazio libero?
\paragraph{Lista Concatenata}
Uso una Lista Concatenata di \textbf{blocchi liberi}, le liste si trovano proprio in questi blocchi liberi.
Questo metodo richiede meno spazio se il disco e' quasi pieno.

\paragraph{Bitmap}
Uso una bitmap per tracciare i blocchi liberi, quindi un bit per ogni blocco del disco, generalmente
richiede meno spazio rispetto alla lista concatenata.

\paragraph{Ottimizzazioni sulla lista concatenata}
Conviene che ciascun blocco ha un \textbf{contatore} che segna i bit liberi, e un numero che rappresenta
il \textit{blocco a partire dal quale sono liberi}

\paragraph{Free List}
Uso una lista concatenata di puntatori: "free list". \textit{Mantengo solo un blocco di puntatori in memoria} e uso questo per
prendere i blocchi da usare.

Per migliorare le performance posso dividere un blocco di puntatori in due e metterlo tra ram e rom. (????)

\subsection{Meccanismo di Quote}
L'amministratore di sistema assegna a ogni utente un \textbf{massimo} \textit{di file e blocchi da usare}.
Il sistema operativo deve controllare che le quote siano rispettate:
\begin{itemize}
    \item Ho una tabella dei file aperti, ognuno ha un proprietario.
    \item Una tabella delle quote tiene conto delle statistiche di ogni utente.
    \item Nella tabella dei file, ad ogni entry corrisponde un puntatore ad una entry nella tabella delle quote.
\end{itemize}
Per ogni utente ci sono dei limiti  \textbf{soft} e \textbf{hard} da rispettare.

\subsection{\texttt{ext2}}
Le componenti  chiave sono:
\begin{itemize}
    \item \textbf{Superblocco}: layout della partizione, numero di I-node, e numero di blocchi.
    \item \textbf{Descrittore del Gruppo}: dettagli su blocchi liberi, I-node
    \item \textbf{Bitmap}: traccia i  blocchi liberi e gli i-node liberi
\end{itemize}