\subsection{Come memorizzare i file?}
\begin{definition} File System \\
    Il file system e' il metodo utilizzato per organizzare e memorizzare dati sui dispositivi di memoria.
\end{definition}

\begin{definition} Partizioni \\
    Un disco puo' essere suddiviso in piu' partizioni, ciascuna con un proprio file system, \textbf{indipendente}.    
\end{definition}

\subsubsection{MBR - Master Boot Record}
L'MBR si trova nel settore 0 del disco ed e' essenziale per avviare il computer, contiene
la tabella delle partizioni, dove ogni \texttt{entry} ha informazioni su inizio e fine della partizione.

All'avvio del sistema il \textbf{BIOS} \textit{legge ed esegue l'MBR} che trova la partizione \textbf{attiva}, quella da usare per caricare il \textbf{boot block}
per avviare il sistema operativo.

\begin{quote}
    Ogni partizione inizia con un \textbf{boot block} anche se non ce l'ha!
\end{quote}

\subsubsection{Nuova scuola: UEFI (Unified Extensible Firmware Interface)}
E' un sistema che sostituisce il vecchio BIOS. Permette:
\begin{itemize}
    \item Avvio piu veloce: inizializzazione dell'hardware ottimizato.
    \item Comptabilita architetture 32 e 64 bit.
    \item Supporto di Interfaccia utente avanazata e mouse.
    \item Secure Boot
\end{itemize}

UEFI funziona con GPT: posso avere dischi piu grandi di \texttt{2,2 TB}.
\paragraph{GUID Partition Table (GPT)}
E' un sistema avanza di gestione delle partizione, supporta dischi enormi, 
puo' gestire un numero illimitato di partizioni (limitato dal sistema operativo che eseguiremo), include un sistema
di \textbf{backup della tabella delle partizioni} e fa anche il \textbf{CRC}. \\

Nei dischi GPT esiste la \textbf{EFI System Partition} (ESP), una partizione speciale che contiene i file di avvio: bootloader e driver, utilizza
il file system \textbf{FAT32} in modo da essere compatibile con UEFI. 

\paragraph{Secure Boot}
Impedisce l'avvio di software non autirizzato: questo software controllare le firme digitali
del \textit{boot loader, del sistema operativo e dei driver}, in modo da avviare solo software con firme
autentiche: \textit {previene l'esecuzione di malware e rootkit all'avvio}.

Lo svantaggio e' che alcuni sistemi operativi le cui firme non sono riconosciute potrebbero non avviarsi.

\subsection{Implementazione dei file}
Bisognia gestire l'associazione tra \textit{file e blocchi sul disco}. Ci sono vari tipi di allocazione 
per i file.

\subsubsection{Allocazione contigua}
I file sono memorizzati come sequenze contigue sul disco: \textit{un file da 50 blocchi, occupera 50 blocchi
consecutivi sul disco}.
Questo metodo e' semplice da implementare ed ha un'alta efficienza di lettura, con una sola lettura posso leggere il file
senza dover cambiare posizione durante la lettura (meglio per i dischi meccanici).  \\

Si ha pero' un problema analogo a quello della RAM: come gestisco la frammentazione? Potrebbe essere 
difficile trovare spazi per infilare i miei file, oppure anche se c'e' abbastanza spazio e' talmente frammentato
da non avere blocchi continui a disposizione.

\subsection{Allocazione a Liste Concatentate}
i file sono \textit{liste concatenate di blocchi sul disco}, ogni blocco ha un \textbf{puntatore} al blocco successivo. \\

Permette di usare meglio lo spazio sul disco, posso facilmente implementare la struttura a directory ma:
\begin{itemize}
    \item richiede di fare molti accessi casuali (devo chiamare \texttt{seek}).
    \item parte dei bit nel blocco sono usati per gestire i puntatori, quindi un blocco potrebbe non contenere dati che occupano una \textbf{dimensione standard}, ossia una potenza di 2. 
\end{itemize} 

\subsection{FAT}
Uso una File allocation table, caricata in RAM! Questa contiene le sequenze di blocchi che compongono un file.
E' efficiente ma per dischi troppo grandi si usa troppa RAM, la tabella e' allocata interamente per tutti i blocchi, anche
quelli inutilizzati, si usa piu che altro per chiavette usb e schede sd di piccole dimensioni.

\subsection{I-NODE}
\begin{definition} I-NODE
    Un Index-Node e' una struttura dati fondamentale in \texttt{ext2/ext3/ext4} (usato in Linux). 
    Contiene le informazioni su un file, \textit{tranne il nome e il contenuto}, ma ne comprende i metadati.
\end{definition}

Ogni file e directory e' rappresentato da un i-node, questi sono organizzati in una tabella. 
La tabella FAT rispetto a quella degli i-node non permette di gestire in modo ottimale i meta dati (altrimenti la grandezza della tabella in RAM crescerebbe).
\textit{I sistemi i-node sono efficienti su dischi grandi}. \\

Un'i-node e' una struttura, di \textbf{dimensione fissa}, che contiene attributi e \textit{gli indirizzi a blocchi} di un file, ad ogni i-node corrisponde un file.
\textit{Rispetto alla FAT, solo gli i-node dei file aperti si trovano in RAM.} \\

Dato che gli i-node hanno dimensione fissa, per i file che hanno tanti blocchi, l'ultimo indirizzo di un i-node potrebbe
puntare per esempio ad un \textit{blocco contentente altri puntatori}, posso gestire efficaciemente file grandi. \\

\subsection{Implementazione Delle Directory}
Le directory mappano i nomi ASCII dei file sulle informazioni per trovare i dati sul disco (primo blocco da leggere o i-node corrispondente).

Come alloco i dati sulla directory? Posso avere un tipo di associazione: \texttt{nome file|attributi} oppure \texttt{nome file|i-node}.
Ma i nomi dei file hanno lunghezze variaibile, di solito si imposta un limite massimo di caratteri: 255. Ma pochi file usano tutti e 255 
i caratteri quindi e' saggio usare delle voci con \textbf{lunghezza variabile}:
\begin{itemize}
    \item Ogni voce ha un header che dice quanto e' lungo il nome del file. \textit{Ma quando elimino file e cartelle si creano buchi di lunghezza variabile!}
    \item Il nome dei file e' contenuto in un \textbf{heap}, subito dopo la fine della directory. Cosi ogni entry ha lunghezza fissa.
\end{itemize}

Ora sorge un'altro problema: la ricerca del file in una directory e' lineare. Usiamo quindi una
tabella di \texttt{hash}, quindi se ho $n$ file, per ognuno di questi grazie ad una funzione di hash
sono in grado di accedere in tempo costante alla lista.

Se a piu file corrisponde lo stesso indice dopo che ho applicato l'hasing, uso delle liste concatenate.
Per velocizzare le cose posso usare una cache che e' efficiente se richiedo un piccolo insieme di file, bisogna
notare che stiamo \textit{aggiungendo complessita al nostro problema}, magari per directory piccole non serve
fare tutto sto casino.

\subsection{Link}
I file possono essere condivisi tra piu utenti tramite:
\begin{itemize}
    \item Hard Link: file che puntano \textbf{all'i-node} condiviso.
    \item Soft Link: file che puntano \textbf{al nome} di un file condiviso.
\end{itemize}

L'\textbf{Hard Link} e' ideale quando ho molti proprietari, l'i-node viene eliminato quando non ci sono piu'
Hard Link a questo, indipendentemente dal numero di Hard Link ho comunque un solo i-node.
L'eliminazione di un i-node si puo' effettuare quando per esempio il contatore di hard link a questo va a 0. \\

Il \textbf{Soft Link} e' piu' rognoso. Per ogni soft link ho un i-node che punta al percorso del file condiviso.

I link introducono un problema: se faccio il backup del disco e trovo i link che devo fa?

\subsection{Spazio del Disco}
Come tengo traccia dello spazio libero?
\paragraph{Lista Concatenata}
Uso una Lista Concatenata di blocchi, questi contengono gli indirizzi dei blocchi liberi.
In caso un blocco non basti a referenziare tutti i blocchi liberi, questo punta ad un altro blocco contenente
altri blocchi liberi.

\paragraph{Bitmap}
Uso una bitmap per tracciare i blocchi liberi, quindi un bit per ogni blocco del disco, generalmente
richiede meno spazio rispetto alla lista concatenata.

\paragraph{Ottimizzazioni sulla lista concatenata}
Conviene che ciascun blocco ha un \textbf{contatore} che segna i bit liberi, e un numero che rappresenta
il \textit{blocco a partire dal quale sono liberi}

\paragraph{Free List}
Uso una lista concatenata di puntatori: "free list". \textit{Mantengo solo un blocco di puntatori in memoria} e uso questo per
prendere i blocchi da usare.

Per migliorare le performance posso dividere un blocco di puntatori in due e metterlo tra ram e rom. (????)

\subsection{Meccanismo di Quote}
L'amministratore di sistema assegna a ogni utente un \textbf{massimo} \textit{di file e blocchi da usare}.
Il sistema operativo deve controllare che le quote siano rispettate:
\begin{itemize}
    \item Ho una tabella dei file aperti, ognuno ha un proprietario.
    \item Una tabella delle quote tiene conto delle statistiche di ogni utente.
    \item Nella tabella dei file, ad ogni entry corrisponde un puntatore ad una entry nella tabella delle quote.
\end{itemize}
Per ogni utente ci sono dei limiti  \textbf{soft} e \textbf{hard} da rispettare.

\subsection{\texttt{ext2}}
Le componenti  chiave sono di un gruppo di blocchi sono:
\begin{itemize}
    \item \textbf{Superblocco}: informazioni sul layout, numero di I-node, e numero di blocchi. Contiene gli \textbf{indirizzi delle bit map} per lo spazio libero: per gli inode e per i blocchi.
    \item \textbf{Descrittore del Gruppo}: dettagli su blocchi liberi, I-node, quante directory ci sono nel gruppo.
    \item \textbf{Bitmap}: traccia i  blocchi liberi e gli i-node liberi
\end{itemize}

Ogni volta che viene creato un nuovo file, ext2 riserva per questo 8 blocchi aggiuntivi inmette a disposizione caso
di scritture future, per cercare di limitare la frammentazione.

Per cercare aree libere in modo efficiente viene utilzzata una bitmap, inoltre il ext2 cerca di posizionare i file
nella stessa cartella nello stesso blocco a cui appartiene la directory genitore.

L'\textbf{accesso} ai file tramite percorso e' ottimizzato grazie ad una \textbf{cache} che mantiene le 
directory a cui ho fatto accesso di recente. Quando accedo devo accedere ad un file devo navigare tutti gli 
inode delle directory che compongono il percorso, fino a che non trovo l'inode del file.

\subsection{Migliorare le performance}
Posso usare una \textbf{cache}, per tenere i blocchi piu' usati. Semplicemente, quando devo leggere un blocco
lo porto in RAM, in modo da poterlo leggere da li per delle letture future.
Posso applicare l'algoritmo LRU in modo efficiente quando lavoro con i blocchi, ma non e' ottimale. \textit{In caso di crash improvviso
i dati modificati in RAM non sono stati scritti sul disco}.
Infatti ci sono tipi di file che sono considerati critici e che il sistema operativo scrive spesso sul disco, alternativamente un programma
utente puo' \textbf{forzare} la scrittura di un file sul disco con la chiamata \texttt{sync}. Grazie all'\textbf{hasing} posso migliorare il tempo di
ricerca nella cache. \\

Il comando \texttt{free} infatti ha una colonna \texttt{cache}, questa mostra quanta RAM e' riservata per i blocchi
mantenuti in memoria.

Esiste in'oltre la \textbf{page cache}, che memorizza le pagine del filesystem virtuale in RAM, prima di 
passare al driver del dispositivo.

\subsection{Deframmentazione}
Col tempo i dischi tendono a frammentarsi: i blocchi di uno stesso file si sparpagliano su tutto il disco.
L'operzione di deframmentazione, cerca di compattare i file, in modo da migliorare le performance su dischi meccanici.

L'utility \texttt{defrag} di windows serve proprio a questo, ma e' sconsigliato su dischi SSD in quanto eseguo operazioni
di scrittura inutili che non fanno altro che diminuire il tempo di vita del disco.

\subsection{Ottimizzazione dello spazio}
Attraverso l'uso di vari algoritmi si possono ridurre sequenze di dati ripetuti: file system come NTFS (Windows), Btrfs e ZFS 
comprimono automaticamente i dati.

E' possibile anche osservare quali sono i file duplicati, in modo da mantenerne solo una copia. Posso mantenere 
le copie a blocchi o per porzioni di file.

\subsection{Affidabilita: Backup}
Ci sono vari eventi che minacciano l'affidabilita del disco:
\begin{itemize}
    \item Dei \textbf{Blocchi Danneggiati}: settori che sono illeggibili.
    \item \textbf{Errori sull'intero disco}: dovuti a errori hardware.
    \item \textbf{Scritture inconsistenti} dovute a tensioni irregolari o interruzioni di energia.
    \item \textbf{Errori Umani}: cancello per sbaglio qualcosa 
    \item \textbf{Smarrimento} del computer, me lo rubano.
    \item \textbf{Malware e Ransomware}: virus e software dannosi che criptano o danneggiano i file.
\end{itemize}

Conviene quindi fare backup del disco nel caso qualcuno sia coglione o che succeda un disastro fisico (terremoti, incendi, blackout).
Ci sono due modi per fare backup:
\begin{itemize}
    \item \textbf{Completo}: faccio una copia totale di tutti i dati, su base settimanale o mensile di solito.
    \item \textbf{Incrementale}: copio solo le porzioni di file modificati, o file interi aggiunti rispetto all'ultimo backup.
\end{itemize}

E ci sono due tipi di backup:
\begin{itemize}
    \item \textbf{Backup Fisico}: Copio tutti i blocchi, da 0 fino all'ultimo.
    \item \textbf{Backup Logico}: Copio file e directory specifiche, ignoro file di sistema e blocchi danneggiati.
\end{itemize}

Su un backup posso considerare di farne la \textbf{compressione} (sconsigliato perche' puo' essere soggetta ad errori). \\
Una tecnica prevede di fare lo \textbf{snapshot} del sistema per garantirne la coerenza del backup in caso di un sistema in uso. \\

\subsubsection{Backup Fisico}
Quando viene fatto un backup fisico e ne copio i dati devo evitare di leggere i blocchi danneggiati: altrimenti ho degli errori in lettura.
Inoltre dovrei evitare di copiare i file temporanei: come i file di paginazione e ibernazione.
Comunque il Backup Fisico e': \textit{facile da implementare e ha poche probabilita di fallire}.

La limitazione piu' grande del backup fisico riguarda la \textit{difficolta nel ripristinare file individuali a partire da questo.}

\subsubsection{Backup Logico}
Parto da specifiche directory e faccio il backup dei file modificati dopo una data specifica,
che e' l'ideale per backup incrementali o completi. E' facile anche fare il recupero di specifici file o directory.

\subsubsection{\texttt{rsync}}
\texttt{rsync} e' usato in unix per \textit{sincronizzare file e cartelle tra due posizioni diverse}.
Durante il trasferimento dei file \textit{solo le parti di file modificate vengono trasferite}, ed e' ideale per backup, ripristino e sincronizzazione
in ambito di rete.

E' in grado di effettuare trasferimenti usano la connessione \textbf{SSH}, in modo da essere criptato.

\subsection{Coerenza del file system: Journaling}
Unix usa \texttt{fsck}, mentre Windows usa \texttt{sfc}, ovvero utility per controllare la coerenza.
Questo processo e' cruciale per mantenere l'\textbf{integrita' dei dati}. Queste utility vengono eseguite spesso
dopo un crash, tutta via i file system con \textbf{journaling} gestiscono autonomamente queste incoerenze.

In un file system con journaling, viene mantenuto un \textbf{log} con le operazioni da eseguire. \textit{Tiene traccia
delle modifiche che verranno apportate al file system}, prima che avvengano effettivamente.
Le operazioni vengono svolte nel seguente ordine:
\begin{itemize}
    \item Il software vuole modificare il file system.
    \item L'operazione da eseguire viene registrata nel file di log.
    \item Dopo che e' stata registrata, viene eseguita l'operazione.
    \item Alla fine l'operazione viene marcata come completata nel file di log.
\end{itemize}

Il sistema in caso di crash, non deve far altro che recuperare le operazioni da completare dal file di log,
il sistema sa esattamente cosa deve essere fatto!

\subsection{Sicurezza dei dati: eliminazione sicura}
Una cancellazione standard non e' abbastanza per rimuovere i file \textbf{fisicamente} dal disco, bisogna
sovrascrivere in maniera approfondita i dati.

Nei dischi magnetici ci potrebbero essere dei \textbf{residui} che permettono di recuperare i dati con tecniche avanzate.

L'approccio piu' sicuro e' usare \textbf{AES} per cifrare il disco: Windows lo fa in background. La chiave
master del volume viene cifrata tramite password utente, chiave di ripristino o TPM.
I dispositivi \textbf{sed} hanno la cifratura integrata ma e' soggetta a vulnerabilita' di sicurezza.

\subsection{VFS: File System Virtuali}
I sistemi operativi spesso gestiscono piu' file sistem contemporaneamente. Windows usa \texttt{C:}, \texttt{D:}, ecc...
mentre i sistemi UNIX integrano piu' file sistem dentro una singola struttura gerarchica.

Il \textit{VFS permette di integrare vari file system in una struttura unificata}. Questo ha un livello comune, che interagisce con i 
file system sottostanti.

L'interfaccia \textbf{inferiore} ha delle funzioni per gestire i file system sottostanti, mentre l'interfaccia \textbf{superiore}
interagisce con le chiamate di sistema. Queste chiamate sono studiate in modo da funzionare indipendentemente da dove 
si trova fisicamente il file system con cui mi interfaccio, funziona anche via internet. \\

Ogni volta che un disco deve essere montato, questo si deve \textbf{registrare al VFS}, e deve fornirgli gli indirizzi
delle procedure che permettono di eseguire le operazioni di lettura e scrittura per quel determinato file system.

IL VFS e' composto da varie strutture dati:
\begin{itemize}
    \item \textbf{Superblocco}: Rapprsenta il descrittore di alto livello di un file sistem specifico registrato.
    Contiene informazioni come la sua dimensione e il tipo.
    \item \textbf{V-node}: E' l'astrazione di un file interno al VFS. Contiene i metadati di un file e non dipende dal file sistem, permette 
    di accedere al file e i suoi dati indipendentemente dal file system astratto.
    \item \textbf{Directory}: Contiene le associazioni tra nomi dei file e v-node.
\end{itemize}

Queste informazioni vengono caricate in RAM all'occorrenza.

\subsection{RAID: redoundant array of inxpensive disks}
\begin{itemize}
    \item RAID 0: distribuisco i dati in \textbf{strisce} su tutti i dischi disponibili. Migliori performance
    quando leggo molti dati.
    \item RAID 1: rispetto a RAID 0 duplico il numero di dischi e li uso come backup. Migliora le performance in lettura.
    \item RAID 2: i bit di una strip sono distribuiti su altrettanti dischi
    \item RAID 3: ho 4 dischi per scrivere parole di 4 bit e ne ho uno di parita. Cosi posso
    ricostruire un disco se si rompe.
    \item RAID 4: come RAID 3 ma lavora usando strip di dati. 
    \item RAID 5: come RAID 4, ma le strip di parita sono distribute su tutti i dischi.
    \item RAID 6: come RAID 5, ma invece di 1 strip di parita ne ho 2 per gli stessi dati.
\end{itemize}

\subsection{Considerazioni sui file system}
\textbf{ext2} viene introdotto nel 1993, per risolvere i problemi di ext.
Con \textbf{ext3} viene introdotto il journaling, e il progetto viene costantemente modificato
ma ad un certo punto si decide di passare a \textbf{ext4}.

\subsubsection{ext4}
Le caratteristiche di \textbf{ext4} sono:
\begin{itemize}
    \item uso degli \texttt{extent} per gestire prozioni continue di blocchi.
    \item compatibile con \texttt{ext3} e \texttt{ext2}
    \item file grandi fino a $16TB$ e file system grandi fino a $1EB.$
    \item uso di un Journaling Block Device per le operazioni di log.
\end{itemize}

\subsubsection{btrfs}
e' figo perche':
\begin{itemize}
    \item Copy-on-write: quando duplico un file ne viene creata una copia effettiva solo quando lo modifico.
    \item Quando modifico i dati, queste si trovano in una porzione differente del disco.
    \item File di grandezza fino a $16exbibyte$
    \item supporto a RAID
    \item deframmentazione e ridimensionamento mentre il sistema e' acceso.
    \item allocazione dinamica degli inode.
    \item supporto per snapshot
    \item supporto per checksum
    \item migliori prestazioni su ssd, "ne allunga la vita".
\end{itemize}

anche se btrfs ha molte funzionalita avanzate e' giovane rispetto a ext4, quindi meno affidabile.

\subsection{Struttura delle cartelle!}
\begin{itemize}
    \item \texttt{/bin}: binari dei principali comandi di sistema, come \texttt{cat,ls,pwd}
    \item \texttt{/etc}: file di configurazione
    \item \texttt{/usr}: file di sistema, e pacchetti utente
    \item \texttt{/sbin}: binari di sistema, comandi come \texttt{ifconfig,mkfs,fdisk}
    \item \texttt{/lib}: librerie condivise.
\end{itemize}