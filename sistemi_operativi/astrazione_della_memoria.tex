Quando eseguiamo piu programmi contemporaneamente ci sono 2 problemi principali:
\begin{itemize}
    \item Gestire il \textit{sovraccarico di memoria}. Cosa succede quando la occupo tutta?
    \item Tenere il sistema \textit{stabile e sicuro}. Per sicuro si intende il fatto che un processo non
    dovrebbe poter accedere ai dati di un'altro programma.
\end{itemize}

\subsection{Modello Monoprogrammazione}
Questo metodo puo' essere applicato in 3 modi:
\begin{itemize}
    \item SO caricato in RAM, poco usato.
    \item SO caricato in ROM, usato in sistemi embedded.
    \item Misto: Sitema Operativo in RAM e i driver in ROM.
\end{itemize}

Non c'e' nessun tipo di astrazione: prendo il programma, lo butto in memoria e basta.
Con questo modello nascono dei problemi dal momento in cui gli 
\textbf{indirizzi assoluti} potrebbero essere interpretati in maniera errata.


\subsection{Implementazione con registri di base}
Questa tecnica sfrutta due registri, che si trovano fisicamente nella cpu:
\begin{itemize}
    \item Registro base: contiene l'indirizzo in cui inizia l'area dedicata al programma
    \item Registro limite: quanta memoria ho allocato?
\end{itemize}

Questi 2 registri permettono di controllare quando un indirizzo assoluto fa 
un riferimento oltre la memoria del programma, anche se cio' richiede di effettuare dei
controlli ogni volta che si fa un riferimento.

\subsection{Sovraccarico Della Memoria}
Il sovraccarico della memoria si puo gestire con varie tecniche, una di queste e' lo
\textbf{swapping}: i processi in memoria vengono spostati sulla memoria non volatile, questa tecnica
richiede molte risorse: la CPU deve scrivere in memoria ROM.
Un'altra tecnica e' la \textbf{memoria virtuale}: uso il processo anche se parazialmente in RAM.
Una soluzione drastica e' quella di uccidere il processo.

\subsection{Gestione della Memoria Libera}
Ogni programma quando viene caricato in memoria e' costituito da:
\begin{itemize}
    \item Testo: il vero e proprio programma che deve essere letto ed eseguito
    \item Dati: porzione di memoria che \textit{cresce in altezza}
    \item Stack: porzione di memoria che \textit{cresce verso il basso}
\end{itemize} 

Bisogna dare la dimensione giusta al programma in modo che Stack e Dati possano crescere 
senza scavalcarsi, altrimenti dovrei riallocarlo di nuovo.

Per tenere traccia delle aree libere di memoria ho due tecniche:
\begin{itemize}
    \item \textbf{Bitmap}: un array di bit, ognuno di questi mi dice se una prozione di memoria e' occupata o no
    \item \textbf{Lista}: ho delle liste collegate, ogni entry mi dice quando una porzione di memoria inizia, quanto e lunga
    e se e' occupata
\end{itemize}

Bitmap e Lista sono rispettivamente: rappresentazione sparsa e densa.

\subsection{Algoritmi di allocazione della memoria}
\begin{itemize}
    \item First Fit: Trova il primo spazio disponibile
    \item Next Fit: Scegli il secondo. Prestazioni peggiori rispetto a First Fit
    \item Best Fit: Tende a frammentare la memoria
    \item Worst Fit: Prestazioni scadenti
    \item Quick Fit: Ho delle linked list divise per dimensione, scarsa performence nella coalescenza
    \item Buddy Allocation: migliora il Quick Fit
\end{itemize}

\subsubsection{Buddy Memory Allocation e Slab Allocator}
    La memoria e' un singolo pezzo contiguo. Ad ogni richiesta di allocazione la
    memoria e' divisa in potenze di 2.
    I blocchi di memoria contigui vengono uniti. \\

    Questo metodo puo' portare a frammentare inutilmente la memoria: se richiedo 65 pagine
    l'algoritmo va ad allocarne 128.  

    Per questo si usa anche lo \textbf{Slab Allocator}. Questo lavora sugli slab che possono essere:
    vuoti, parzialmente pieni o pieni. 
    Lo Slab Allocator viene usato per strutture di memoria piccole che vengono create e eliminate molto spesso.
    Una volta che viene allocato uno slab e infine liberato, questo rimane in cache in modo che possa essere riutilizzato
    per tenere un nuovo dato dello stesso tipo: evito l'overhead dell'inizializzazione. 