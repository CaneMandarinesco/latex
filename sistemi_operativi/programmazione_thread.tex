\subsection{\texttt{pthread\_create}, \texttt{pthread\_join}, \texttt{pthread\_exit}}

\texttt{pthread\_join} aspetta che il thread specificato termini, posso anche
fornire un puntatore alla variabile che conterra' il valore di ritorno. \\

\texttt{pthread\_create} crea il thread. Puo' richiedere un oggetto \texttt{pthread\_attr\_t}
con gli attributi di inizializzazione del thread. 

\begin{lstlisting}[language=C]
#define NUMBER_OF_THREADS 10

void *print_hello_world(void * tid){
    int r = rand()%5;
    sleep(r);
    printf("Thread: %d\n", tid);
    pthread_exit(NULL);
}

int main(int arg, char* argv[]) {
    pthread_t threads[NUMBER_OF_THREADS];
    int status, i;

    for(i=0; i < NUMBER_OF_THREADS; i++){
        status = pthread_create(&threads[i], NULL, 
            print_hello_world, (void *)i);

        if(status == 0)
            printf("thread %d created, status %d\n", 
                i, status);
        else {
            printf("error: thread %d\n", i);
            exit(-1);
        }
    }

    for (i=0; i < NUMBER_OF_THREADS; i++) {
        status = pthread_join(threads[i], NULL);
        printf("%d terminated, status %d", i, status);

        if(status != 0){
            printf("%d has a problem\n", i);
            exit(-1);
        }
    }
    return 0;
}
\end{lstlisting}

\subsection{Semafori}
\texttt{mutex, empty e full} sono i semafori che regolano l'accesso al buffer. 
Il produtture prima di inserire un oggetto nel buffer controlla il semaforo \texttt{empty} e \texttt{mutex}.
Il consumatore invece controlla il semafor \texttt{full} 

\begin{lstlisting}[language=c]
#include <pthread.h> 
#include <stdio.h> 
#include <stdlib.h>
#include <unistd.h>
#include <semaphore.h>

#define N 10
#define TRUE 1

sem_t mutex;
sem_t empty;
sem_t full;

int buffer[N];
int in = 0;

void down(sem_t *sem){
    sem_wait(sem);
}

void up(sem_t *sem){
    sem_post(sem);
}

void insert_item(){
    int item = rand() % 100;
    printf("%d, %d", item, n+1);
    buffer[in] = item;
    in++;
}

void remove_item() {
    printf("%i", in);
    int item = buffer[in - 1];
    printf("\nPrelevo %d da posizione %d\n", item, in);
    in--;
}

void *producer(void *args){
    while(TRUE) {
        down(&empty);
        down(&mutex);

        insert_item();
        print_buffer();

        up(&mutex);
        up(&full);
    }
}

void *consumer(void *args) {
    while (TRUE){
        down(&full);
        down(&mutex);

        remove_item();
        print_buffer();

        up(&mutex);
        up(&empty);
    }
}

int main() {
    pthread_t prod, cons;

    sem_init(&mutex, 0, 1);
    sem_init(&mutex, 0, N);
    sem_init(&full,  0, 0);

    pthread_create(&prod, NULL, producer, NULL);
    pthread_create(&prod, NULL, consumer, NULL);

    pthread_join(prod, NULL);
    pthread_join(cons, NULL);

    sem_destroy(&mutex);
    sem_destroy(&empty);
    sem_destroy(&full);

    return 0;
}
\end{lstlisting}

\subsection{Lettori e Scrittore}
Ipotizzo di avere tanti lettori e un solo scrittore. L'accesso ai lettori e' regolato nel seguente modo:
la variabile \texttt{rc} conta quanti lettori ci sono al corrente, una volta che un lettore blocca lo scrittore 
dall'accesso al buffer, non c'e' bisogno che gli altri lettori facciano lo stesso. Solo il primo blocca \texttt{db}
e l'ultimo lettore a legger sblocca \texttt{db}.

L'unica cosa che deve essere regolata tra tutti i lettori e' l'accesso al \texttt{rc} stesso.

\begin{lstlisting}[language=c]
void *reader(void *argc){
    wihle(TRUE){
        down(&mutex);
        rc++;
        if(rc==1) down(&db);
        up(&mutex);

        read_database();

        down(&mutex);
        rc--;
        if(rc==0) up(&db);
        up(&mutex);

        use_data_read();
    }
}

void *writer(void *arg){
    while(TRUE)[
        think_up_data();
        
        // L'accesso al database e' bloccato
        down(&db);
        write_database();
        up(&db);
    ]
}
\end{lstlisting}

\subsection{Mutex}
\begin{lstlisting}[language=c]
pthread_mutex_t the_mutex;
pthread_cond_t condc, condp;

int buffer = 0;

void *producer(void *ptr){
    int i;
    for (i = 1; i <= MAX; i++) {
        pthread_mutex_lock(&the_mutex);

        while (buffer != 0)
            pthread_cond_wait(&condp, &the_mutex);
        
        buffer = i;

        pthread_cond_signal(&condc);
        pthread_mutex_unlock(&the_mutex);
    }

    pthread_exit(0);
}

void *consumer(void *ptr) {
    int i;
    for(i=1; i<=MAX; i++){
        pthreaD_mutex_lock(&the_mutex);
        while(buffer == 0){
            pthrea_cond_wait(&condc, &the_mutex);
        }

        // usa il buffer

        pthread_cond_signal(&condp);
        pthread_mutex_unlock(&the_mutex);
    }
    pthread_exit(0);
}
\end{lstlisting}

\subsection{Esercizio}
Il programma chiesto deve creare altri due processi \texttt{p1} e \texttt{p2}.
Entrambi generano dei numeri da 0 a 100, ma, p1 invia al processo padre solo
i numeri dispari, mentre p2 invia al padre solo i numeri pari.
Una volta che il processo padre ha ricevuto una coppia di numeri pari e dispari, ne fa la 
somma, se e' maggiore di 190 l'esecuzione del programma termina.

Le chiamate di sisteme e procedure necessarie per creare il codice sono:
\begin{itemize}
    \item \texttt{pipe}, per creare le due pipe, una per processo figlio.
    \item \texttt{fork}, per creare i processi figli.
    \item \texttt{perror}, per comunicare gli errori delle chiamate di sistema.
    \item \texttt{read/write}, per leggere e scrivere sui file descriptor.
    \item \texttt{kill}, per mandare il segnale di terminazione ai processi figli.
    \item \texttt{waitpid}, per aspettare che i figli terminino l'esecuzione.
\end{itemize}

Per randomizzare al meglio la generazione dei numeri si aggiora il seed con: \texttt{srand(time(NULL) \^ getpid())}.
