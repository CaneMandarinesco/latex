Una maccina ha due prospettive:
\begin{itemize}
    \item Dimensione software: interfaccia untente, applicazioni e \textbf{sitema operativo}. 
    \item Dimensione hardware: al programmatore non interessa.
\end{itemize}

Il Sistema Operativo deve mettere a disposzione dei programmi le risorse hardware,
funge da arbitro per i processi e vengono eseguiti in modo \textbf{esclusivo}, infatti deve
poter gestire piu' utenti e condividere tra questi CPU e Memoria.

Il Sistema Operativo mette in sicurezza le risorse hardware, solo il codice del 
sistema operativo puo' accedere a certe aree di memoria e solo lui puo' eseguire alcune
operazioni. Questo comportamento e' determinato da alcuni registri nella CPU.

Il Sistema Operativo puo' essere inteso come un'estensione della macchina, un gestore delle risorse.

\subsection{Concetto di chiamata}
Quando chiamo \texttt{read} il sistema operativo mette in pausa il processo chiamante,
salva lo stato dei registri, e esegue la sua routine. Quando il sistema operativo ha finito
ripristina lo stato dell'esecuzione nel processore e tutto torna come prima.
Questo \textbf{cambio di contesto} richiede molte risorse.

Il codice infatti puo' essere eseguito in due modalita:
\begin{itemize}
    \item kernel mode: non ho nessun tipo di restrizione.
    \item user mode: non posso fare quello che voglio.
\end{itemize}

L'esecuzione di una chiamata di sistema e' composto da:
\begin{itemize}
    \item Programma utente chiama sys call
    \item Il controllo dell'esecuzione passa al sistema operativo
    \item salva lo stato del programma utente ed esegui la procedura
    \item ripristina lo stato del processore ed esegui il processo chiamante.
\end{itemize}

\subsection{Storia}
All'inizio vi erano i computer a \textbf{valvole termoioniche}. Non c'era un 
sistema operativo, la programmazione avveniva tramite linguaggio macchina, ed erano molto lenti. \\

Con i \textbf{transistor} nascono i primi sistemi che eseguono batch e job.
Un batch e' un programma che \emph{non richiede l'interazione con l'utente}, puo' essere
eseguito in \textbf{background}.
Nasce il concetto di parallelizzazione del carico di lavoro: una macchina legge (e scrive il codice su nastro magnetico) il programma,
una esegue i calcoli, e un'altra stampa (l'output su nastro magnetico).
Le prime macchine a transistor erano generalmente dei \textbf{mainframe}.
C'e' anche il concetto di divisione delle aree di memoria, un programma batch e' costituito da:
\begin{itemize}
    \item \texttt{JOB}: tempo massimo di esecuzione
    \item \texttt{FORTRAN}: compilatore
    \item il \texttt{Programma} vero e proprio
    \item \texttt{LOAD}
    \item \texttt{RUN}: regole di esecuzione, parametri
    \item \texttt{END}
\end{itemize}

Con i \textbf{circuiti integrati} nascono i primi sistemi operativi, con
l'obiettivo di rendere il codice portabile tra diversi sistemi.
Nasce anche il concetto di famiglia di computer.
I computer sono in grado di tenere in memoria piu' job, con \textbf{multics} 

Il primo sistema \textbf{time sharing} e' \texttt{multics}. Invece \texttt{unix} 
era una versione monoutente di \texttt{multics}. \texttt{minix} nasce come variante di
\texttt{multics}, e' il sistema modulare di Tanenbaum che nasce come progetto didattico.
Successivamente nasce \texttt{linux} come variante di \texttt{minix}.

Dato che molte persone hanno preso \texttt{unix} per modificarlo, nasce la necessita' 
di introdurre uno standard, ossia POSIX, per rendere il codice portabile tra piu sistemi operativi.

\subsection{Processore}
Generalmente il processore e' composto da i registri:
\begin{itemize}
    \item \texttt{PSW}: contiene informazioni sullo stato del processore.
    \item \texttt{SP}: puntatore allo Stack, area di memoria che solitamente contiene indirizzi delle procedure, 
    parametri delle procedure e variabili locali.
    \item \texttt{PC}: indirizzo dell'istruzione in esecuzione
\end{itemize}

La CPU e' molto complessa, contiene infatti: una pipeline con un buffer che raccoglie l'output delle operazioni,
spesso ci sono piu' core, e ci sono le cache L1, L2 e L3. 

I thread della CPU permettono di tenere salvato lo stato di esecuzioni di processo e 
di scambiare nell'ordine di 1 nanosecondo il programma in esecuzione sul core.

Il sistema operativo deve gestire queste componenti per permettere il \textbf{multiplexing} delle risorse.

\subsection{IO}
Un dispositivo di input output e' composto dal controller e dal dispositivo vero e proprio.
Il sistema operativo si interfaccia con il controller per governarlo, e viene fatto
attraverso le operazioni di tipo \texttt{in/out} (richiedono che il processo rimanga in attesa), o 
attraverso gli interrupt o piu banalmente attraverso lo scambio di messaggi, e in fine attrverso il DMA.

Tutte queste operazioni sono disponibili solo in kernel mode, avviene un cambio di contesto ogni volta 
che bisogna usare un dispositivo IO.

La CPU e' in grado di disabilitare gli interrupt: se sto gestendo un'interrupt non posso riceverne un'altro!

Il software che permette di manovrare il controller e' il \textbf{driver}, questo fornisce
le mappature e le procedure da usare per manovrarlo. Quando sono installate le procedure devono essere
linkate al kernel.

\subsection{BUS}
un computer e' composto da diversi tipi di bus, ognuno con velocita di trasferimento diverse:
DDR4 e DDR5 sono dei bus separati che collegano esclusivamente CPU e RAM, mentre PCIe e'
un bus che viene utilizzato per collegare diverse periferiche: schede video, schede di rete, controller 
di vario tipo, ed e' il sostituto del vecchio ISA.

Il bus USB nasce invece come metodo universale per collegare periferiche al volo al computer,
ed era fatto principalmente per dispositivi lenti, anche se oggi le perifieriche usb riescono 
a trasmettere 5GB/s.

\subsection{BOOT}
Una memoria sulla scheda madre contiene il \textbf{firmware}, ossia il 
\textbf{BIOS}, un programma che esegue delle routine di controllo all'avvio: controlla le periferiche
installate, le inizializza, imposta i servizi base per IO, e cerca nel CMOS il percorso impostato
dove si trova il boot loader. Quest'ultimo viene avviato e a sua volta avvia il sistema operativo.

\subsection{Diversi tipi di OS}
I \textbf{sistemi Mainframe} si occupano di gestire l'esecuzione di 
processi che non richiedono l'interazione con l'utente, si occupano di
elaborare grandi quantita di transazioni.

Spesso sono sistemi \textbf{time sharing}: piu' utenti possono connettersi e richiedere
l'esecuzione di un job. \\

I \textbf{server} sono ottimizzati per gestire richieste che viaggiano 
sul web. Sono installati su macchine simili a personal computer, o anche su dei mainframe. \\

i sistemi \textbf{Real Time} sono sistemi in cui e' importante rispettare 
le scadenze di esecuzione. Possono essere Soft o Hard.

In tutti i casi il sistema operativo protegge la macchina da se stessa: evita un'uso 
scorretto e dannoso, evita che un'utente invada lo spazio di un'altro utente.

\subsection{Processo}
Ogni processo ha:
\begin{itemize}
    \item spazio degli indirizzi
    \item risorse assegnate: file aperti, dispositivi in uso, allarmi
\end{itemize}

Lo spazio degli indirizzi e' costituito da: 
\textbf{stack} che contiene indirizzi di procedure e chiamate, \textbf{data} contiene le variabili, 
\textbf{text} contiene il testo del programma.

Un processo puo' creare altri processi, in unix avviene tramite la duplicazione,
e alcuni sistemi operativi mantengono una struttura ad albero che descrive la parentela dei processi.
La comunicazione tra piu' processi e' possibile attraverso le \textbf{pipe} e alla \textbf{Interprocess comunication}

I processi sono di proprieta' di utenti, che a sua volta appartengono a gruppi.

Le informazioni dei processi si trovano nella \textbf{tabella dei processi}.

\subsection{File}
Un file e' l'astrazione del disco fisico. Viene letto fornendo al driver 
i byte da leggere. 
In unix un file non e' necessariamente una sequenza di byte in un disco fisico,
ma puo' rappresentare dispositivi di i/o, o per esempio informazioni di vario genere
sul sistema operativo. \\

Il \textbf{file descriptor} e' un intero che identifica un file aperto, questo 
viene richiesto attraverso il nome del file e viene fornito dal sistema operativo se 
ho i diritti di accesso. \\

La \textbf{pipe} e' uno pseudo file che viene usato per collegare piu' processi
insieme, in modo da permettere lo scambio di dati.

\subsubsection{diritti di accesso}
ogni file e' protetto con 3 bit: \texttt{r}, \texttt{w},\texttt{x}.
Ci sono generalemte 3 triplette di questi bit: i primi 3 riguardano i permessi del proprietario,
i secondi 3 per quanto riguarda il gruppo di appartenza del proprietario e gli ultimi per gli altri utenti.

\subsection{Struttura di un sistema operativo}
In un computer il \textbf{programma principale} invoca le chiamate di sistema mentre
il \textbf{kernel}: di solito un blocco monolitico, le esegue.

Il contentuto del kernel monolitico si trova in un'unica cartella, la maggior parte dei sistemi
sono monolitici perche':
\begin{itemize}
    \item Ho un'unico programa strutturato invece di avere piu' servizi che dialogano tra loro.
    \item Implementare i servizi richiede complessita e peggiora l'efficienza
    \item E' piu' complesso compilarlo: le procedure dei moduli devono essere collegate.
    \item Un modulo potrebbe chiamare moduli che non dovrebbe chiamare, bisogna implementare questo aspetto della sicurezza.
    \item Cambiare processo vuol dire spesso cambiare modulo
\end{itemize}

Un kernel monolitoco spesso ha dei moduli al suo interno, ma spesso, uno dipende dall'altro
quindi non posso cambiare moduli al volo, spesso devo cambiare la tabella delle procedure disponibili
quando installo un driver.

In unix ci sono le librerie condivise: \texttt{shared libs}, una collezione di procedure visibili a tutti.

Un sistema monolitoco e' ben strutturato, infatti: il programma utente 
e' in grado di chiamare le librerie utente, le librerie utente attivano le chiamate di sistema, 
che a loro volta si interfacciano col kernel. \\

La struttura a \textbf{microkernel} prevede la suddivisione del sistema operativo 
in tanti kernel eseguiti su processi diversi. Il vantaggio di questa struttura e' che un'errore
su un microkernel non sfonda tutto il sistema, al limite basta riavviare un singolo processo. \\

Il modello \textbf{client/server } prevede uno o piu sistemi client che fanno richieste ad un server. \\

\subsection{Macchine Virtuali}
Le macchine virtuali possono essere eseguite solo su CPU virtualizzabili, ossia che 
riescano ad ingannare il software facendogli credere di star eseguendo in modalita privilegiata,
quando in verita' e' solo una macchina virtuale. \\

La virtualizzazione di \textbf{tipo 1} permette l'esecuzione del sistema operativo sull'hardware.
La virtualizzazione di \textbf{tipo 2} usa un \textbf{Machine Simulator} per far dialogare
kernel e il sistema operativo host. Tutte le risorse vengono richieste al sistema host, il che non e' efficiente,
infatti alcune istruzioni bypassano la struttura.

\subsection{Container}
Il container e' un tipo di virtualizzazione \textbf{leggera}, virtualizzo il sistema operativo usando il kernel 
dell'host. Quindi ci deve essere una compatibilita tra host e container.
Il Container e' logicamente separato dal sistema operativo, i programmi dentro al container non sanno di essere li dentro.

\subsection{Exokernel}
Divido a priori le risorse hardware tra le macchine virtuali da eseguire.

\subsection{Unikernel}
Sistemi operativi minimali, non sono general purpose ma 
mettono a disposizione poche librerie per eseguire poche applicazioni,
se tolgo funzionalita al sistema, ho meno vulnerabilita'.