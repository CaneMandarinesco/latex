Nota: \texttt{unistd.h} contiene tutte le API richieste dallo standard POSIX,
cioe' i wrapper per le chiamate di sistema richieste da POSIX.

\subsection{\texttt{write}}
\begin{lstlisting}[language=c]
    #include <unistd.h>

       ssize_t write(int fd, const void buf[.count], size_t count);
\end{lstlisting}

Scrive \texttt{count} byte presi dal buffer in input.

Esempio:
\begin{lstlisting}[language=c]
    #include <unistd.h>

    char msg[] = "Hello World!\n";
    write(STDOUT, msg, sizeof(msg));
\end{lstlisting}

Esempio deprecato (pericoloso), bypasso il la funzione wrapper:
\begin{lstlisting}[language=c]
    #include <unistd.h>
    #include <stdio.h>
    #include <sys/syscall.h>

    char msg[] = "Hello World!\n";
	int nr = SYS_write;
	syscall(nr, STDOUT, msg, sizeof(msg));
\end{lstlisting}

\subsection{\texttt{fork}}
\begin{lstlisting}[language=c]
    #include <stdio.h> 
    #include <stdlib.h>
    #include <unistd.h> 
    int main(){

        int pid, child_status;

        if ((pid = fork()) == 0) {
            // Codice del figlio
        } else { 
            // Codice del padre
            wait(&child_status);
        }
    }
\end{lstlisting}

\subsection{\texttt{fork}, \text{dup}, \text{execl}, \text{waitpid}}

\begin{lstlisting}[language=c]
    #include <unistd.h>

    int dup(int oldfd);
    int dup2(int oldfd, int newfd);
\end{lstlisting}

\texttt{dup} crea un nuovo file descriptor che punta allo 
stesso file del descrittore \texttt{oldfd}. Il risultato e' che
se prima chiudo un file descriptor e poi chiamo \texttt{dup}, 
usare il file descriptor chiuso equivale ad utilizzare il nuovo file descriptor.
Infatti, puntano entrambi allo stesso file.

Per meglio precisare: il vecchio file descriptor scelto da \text{dup} e' quello 
piu' basso che e' stato chiuso.

\begin{lstlisting}[language=c]
    #include <stdio.h> 
    #include <stdlib.h>
    #include <unistd.h> 
    int main(){

        pid_t cat_pid, sort_pid;
        int fd[2];

        pipe(fd);
        
        cat_pid = fork();
        if(cat_pid==0){
            // processo cat
            close(fd[PIPE_RD]);
            close(STDOUT);

            // ora l'output e' dirottato sulla pipe
            dup(fd[PIPE_WR]);
            execl("/bin/cat", "cat", "names.txt", NULL);
        }

        sort_pid = fork();

        if(sort_pid == 0){
            close(fd[PIPE_WR]);
            close(STDIN);

            dup(fd[PIPE_RD]);

            execl("/usr/bin/sort", "sort", NULL)
        }

        // chiude il file descriptor
        close(fd[PIPE_RD]);
        close(fd[PIPE_WR]);

        // aspetta che il processo termina
        waitpid(cat_pid, NULL, 0);
        waitpid(sort_pid, NULL, 0);
    }
\end{lstlisting}

\subsection{\texttt{signal}}
\begin{lstlisting}[language=c]
    #include <signal.h>
    #include <unistd.h>
    #include <string.h>

    void sigHandler(int signum){
        printf("Interrupt sig %d: %s", signum, 
            strsignal(signum));
        exit(signum);
    }

    int main(){
        signal(SIGINT, sigHandler); // CTRL+C

        while(1){
            printf("Sleep\n");
            sleep(1);
        }

        return 0;
    }
\end{lstlisting}

Si puo' fare la stessa cosa ma con \texttt{SIGALARM}.
Questo segnale viene scatenato dalla funzione \texttt{alarm(1)}.

\subsection{Shell}
\begin{lstlisting}[language=c]
#include <stdio.h>
#include <stdlib.h>
#include <string.h>
#include <sys/types.h>
#include <sys/wait.h>
#include <unistd.h>

int main(){
    while(1){
        char cmd[256], *args[256];
        int status;
        pid_t pid;

        read_command(cmd, args);

        if (strcmp(cmd,"exit") == 0) 
            exit(0);
        
        pid = fork();
        if(pid < 0){
            perror("fork failed");
            exit(0);
        }
        if (pid == 0){
            if(execvp(cmd, args) == -1){
                perror("...");
                exit(0);
            }
        } else {
            if (wait(&status) == -1){
                perror("...");
                exit(0);
            }
        }

        free_args(args);
    }
}
\end{lstlisting}

\texttt{perror} scrive un messaggio nello \texttt{stderr}. \\
\texttt{wait} aspetta che un figlio cambi stato.

\begin{lstlisting}[language=c]
void read_command(char *cmd, char **args){
    char* input = NULL;
    size_t len = 0;
    printf("myshell> ");
    ssize_t read_len = 
        getline(&input, &len, stdin);

    if(read_len == -1){
        // CTRL-D pressed.
        printf("\n");
        exit(0);
    }

    chat* token = strtok(input, " \n");

    int i = 0;
    while(token){
        args[i] = strdup(token);
        if(i == 0)
            strcpy(cmd, token);
        token = strtok(NULL, " \n");
        i++;
    }
    args[i] = NULL;
    free(input);
}

void free_args(char** args){
    int i = 0;
    while(args[i]){
        free(args[i])
        i++;
    }
}
\end{lstlisting}

\texttt{strtok} prende la sequenza di caratteri contenuta in uno 
dei token forniti. Con \lstinline{strtok(NULL, " \n")} ottengo la sequenza successiva.
