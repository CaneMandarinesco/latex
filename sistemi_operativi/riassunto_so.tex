\documentclass{article}

\usepackage[italian]{babel}
\usepackage{listings}
\usepackage{amsmath}

\title{Riassuntazzo Sistemi Operativi}
\author{Davide Luci}
\date{2024}

\newtheorem{definition}{Definition}[section]

\begin{document}

    \maketitle
    \pagenumbering{gobble}
    \tableofcontents
    \newpage

    \pagenumbering{arabic}

    \setlength{\parindent}{0pt}
    \section{ Prima di leggere...}
    Questo documento non e' esaustivo al fine di comprendere al pieno la materia trattata.
    Le definizioni sono scarne, e spesso non esaustive, perche' ho preferito solo appuntare i concetti chiavi
    da ripassare in modo da trovare una giusta via di mezzo tra studiare e riassumere. Si puo' dire che sia compito del 
    lettore, quello di arricchire il contenuto di questo documento.

    \section {Astrazione della memoria}
        Quando eseguiamo piu programmi contemporaneamente ci sono 2 problemi principali:
\begin{itemize}
    \item Gestire il \textit{sovraccarico di memoria}. Cosa succede quando la occupo tutta?
    \item Tenere il sistema \textit{stabile e sicuro}. Per sicuro si intende il fatto che un processo non
    dovrebbe poter accedere ai dati di un'altro programma.
\end{itemize}

\subsection{Modello Monoprogrammazione}
Questo metodo puo' essere applicato in 3 modi:
\begin{itemize}
    \item SO caricato in RAM, poco usato.
    \item SO caricato in ROM, usato in sistemi embedded.
    \item Misto: Sitema Operativo in RAM e i driver in ROM.
\end{itemize}

Non c'e' nessun tipo di astrazione: prendo il programma, lo butto in memoria e basta.
Con questo modello nascono dei problemi dal momento in cui gli 
\textbf{indirizzi assoluti} potrebbero essere interpretati in maniera errata.


\subsection{Implementazione con registri di base}
Questa tecnica sfrutta due registri, che si trovano fisicamente nella cpu:
\begin{itemize}
    \item Registro base: contiene l'indirizzo in cui inizia l'area dedicata al programma
    \item Registro limite: quanta memoria ho allocato?
\end{itemize}

Questi 2 registri permettono di controllare quando un indirizzo assoluto fa 
un riferimento oltre la memoria del programma, anche se cio' richiede di effettuare dei
controlli ogni volta che si fa un riferimento.

\subsection{Sovraccarico Della Memoria}
Il sovraccarico della memoria si puo gestire con varie tecniche, una di queste e' lo
\textbf{swapping}: i processi in memoria vengono spostati sulla memoria non volatile, questa tecnica
richiede molte risorse: la CPU deve scrivere in memoria ROM.
Un'altra tecnica e' la \textbf{memoria virtuale}: uso il processo anche se parazialmente in RAM.
Una soluzione drastica e' quella di uccidere il processo.

\subsection{Gestione della Memoria Libera}
Ogni programma quando viene caricato in memoria e' costituito da:
\begin{itemize}
    \item Testo: il vero e proprio programma che deve essere letto ed eseguito
    \item Dati: porzione di memoria che \textit{cresce in altezza}
    \item Stack: porzione di memoria che \textit{cresce verso il basso}
\end{itemize} 

Bisogna dare la dimensione giusta al programma in modo che Stack e Dati possano crescere 
senza scavalcarsi, altrimenti dovrei riallocarlo di nuovo.

Per tenere traccia delle aree libere di memoria ho due tecniche:
\begin{itemize}
    \item \textbf{Bitmap}: un array di bit, ognuno di questi mi dice se una prozione di memoria e' occupata o no
    \item \textbf{Lista}: ho delle liste collegate, ogni entry mi dice quando una porzione di memoria inizia, quanto e lunga
    e se e' occupata
\end{itemize}

Bitmap e Lista sono rispettivamente: rappresentazione sparsa e densa.

\subsection{Algoritmi di allocazione della memoria}
\begin{itemize}
    \item First Fit: Trova il primo spazio disponibile
    \item Next Fit: Scegli il secondo. Prestazioni peggiori rispetto a First Fit
    \item Best Fit: Tende a frammentare la memoria
    \item Worst Fit: Prestazioni scadenti
    \item Quick Fit: Ho delle linked list divise per dimensione, scarsa performence nella coalescenza
    \item Buddy Allocation: migliora il Quick Fit
\end{itemize}

\subsubsection{Buddy Memory Allocation e Slab Allocator}
    La memoria e' un singolo pezzo contiguo. Ad ogni richiesta di allocazione la
    memoria e' divisa in potenze di 2.
    I blocchi di memoria contigui vengono uniti. \\

    Questo metodo puo' portare a frammentare inutilmente la memoria: se richiedo 65 pagine
    l'algoritmo va ad allocarne 128.  

    Per questo si usa anche lo \textbf{Slab Allocator}. Questo lavora sugli slab che possono essere:
    vuoti, parzialmente pieni o pieni. 
    Lo Slab Allocator viene usato per strutture di memoria piccole che vengono create e eliminate molto spesso.
    Una volta che viene allocato uno slab e infine liberato, questo rimane in cache in modo che possa essere riutilizzato
    per tenere un nuovo dato dello stesso tipo: evito l'overhead dell'inizializzazione. 
    \section {Memoria Virtuale}
        Negli anni 60, prima di una vera e proprioa memoria virtuale c'era l'overlay,
la cui implemnetazione era a carico del programmatore. 

Il programma era diviso in \textbf{overlay} ognuno "indipendente" dall'altro: il 
sistema operativo poteva quindi evitare di caricare tutto il programma in memoria. \\

La memoria virtuale estende il concetto di registro base e limite: ogni programma ha uno spazio degli
indirizzi diviso in \textbf{pagine}: ossia intervalli continui di memoria, per esempio 4KB.

La memoria virtuale da al processo \textit{l'illusione di avere un spazio di indirizzi ampio}, per esempio a 48 bit
anche se la RAM non ha fisicamente spazio. La memoria virtuale e' organizzata quindi in pagine che devono essere mappate 
alle pagine della memoria fisica. \\

\subsection{Tabella delle pagine}
    Molte delle informazioni riguardanti la paginazione, come per esempio
    le \textit{associazioni tra pagina virtuale e fisica}, si trovano nella tabella delle pagine: di proprieta del processo (Quindi esiste una tabella per ogni processo).
    Gli indirizzi della memoria virtuale (usati dal processo) devono essere tradotti in indirizzi per la memoria fisica, questo
    lo fa la \textbf{MMU}, questa prevede una cache per ridurre i tempi di accesso alla tabella delle pagine (???).

    Se la pagina virtuale richiesta dal processo non e' associata ad una pagina fisica si crea un \textbf{Page Fault} (La RAM e' piena!)
    e dove capire qualche pagina spostare sul disco

    Un indirizzo virtuale puo' essere costituito per esempio da: 4 bit che indicano la pagina e 12 bit che indicano l'offset.
    Con questo layout posso gestire 16 pagine dove ognuna ha 4096 byte.

    Ogni voce della tablla ha:
    \begin{itemize}
        \item Bit Presente/Assente: la pagina virtuale e' in memoria?
        \item Bit Protezione: che diritti di accesso ho sulla pagina?
        \item Bit Supervisor: solo il sistema operativo puo' accedere alla pagina?
        \item Bit Modificato e Riferimento: la pagina e' stata modificata? e' stato effettuato 
        un accesso alla pagina?
    \end{itemize}

    Ora l'accesso alla tabella delle pagine non deve essere un collo di bottiglia, ci sono due tecniche:
    \begin{itemize}
        \item Dei registri mantengono i dati per ciascuna pagina virtuale. Ad ogni cambio di contesto devo ricaricare i registri dalla RAM.
        \item Un solo registro punta all'indirizzo di memoria che contiene tutte le pagine. Accedere ad ogni pagina richiede la lettura della RAM.
    \end{itemize}

    \subsubsection{TLB}
        Osservazione: i programmi tendono a fare riferimenti a un piccolo numero di pagine. Introduciamo quindi il \textbf{TLB},
        un aggieggio hardware che mappa gli indirizzi virtuali e fisici senza dover passare per la RAM, funziona
        come una vera e propria \textbf{cache}, infatti ha poche voci, dove ognuna contiene: \textit{pagina virtuale, page frame, e altri bit di controllo} come il bit \textbf{valid}
        L'introduzione di questa e' necessaria dato che senza per ogni richieste di accesso alla memoria, duplicherei le operazioni da fare.

        Dato che questa e' una cache, posso avere dei \textbf{TLB miss}, quindi il sistema operativo deve aggiornare
        il suo contenuto e rieseguire l'istruzione.

    \subsubsection{Organizzazione della page table}
        Bisogna trovare il giusto numero di pagine in cui suddividere la memoria, con pagine troppo piccole, per esempio
        in un'architettura a 32 bit, se ne uso 12 per l'indirizzo e 20 per selezionare la pagina, allora ho $2^20=1.048.576$ pagine e altrettante
        voci nella tabella delle pagine: \textit{uso almeno 1GB di RAM solo per la tabella.}

        Introduciamo le page table a due livelli. In questo caso le page table sono dei veri e propri cammini navigabili
        In una macchina a 32-bit potrei avere per esempio: 10 bit che selezionano il primo livello, 10 bit per il secondo, 12 bit per l'indirizzo.

        Cosi facendo non ho bisogno di allocare memoria per le page table non usate, l'uso di una page
        table monolivello  implica che il sistema debba allocare tutta la memoria richiesta per ogni entry della pagina.

    \subsubsection{TLB miss}
        I TLB miss sono molto comuni dato che posso tenere poche voci. Posso avere due tipi di miss:
        \begin{itemize}
            \item Soft: La pagina e' in memoria ma non nel TBL
            \item Hard: La pagina non e' in memoria e, quindi nemmeno nel TLB.
        \end{itemize}

        Se invece faccio un'accesso ad un indirizzo non valido ho un segmentation fault.
\subsection{Algoritmi di sostituzione delle Pagine}
    \subsubsection{Algoritmo Ottimale}
        Rimuovo la pagina che in futuro sara' usata meno volte. Questo algoritmo e' impossibile da implementare.
    \subsubsection{NRU}
        Questo algoritmo cerca le pagine che non sono state modificate e accedute di recentemente.
        Ad ogni ciclo di clock il bit R viene ripulito per identificare le pagine meno usate di recente.
        NRU non fa altro che elminare le pagine di classe piu bassa: con meno bit M e R accesi.
    \subsubsection{FIFO}
        La pagina piu vecchia (la prima ad essere inserita) e' anche la prima candidata ad essere buttata
        fuori  (la prima ad uscire). Il problema di questa implementazione e' che le pagine piu vecchie potrebbero essere
        ancora le piu utilizzate.
    \subsubsection{Second Chance - FIFO}
        come FIFO ma prima controllo i bit R, se e' uguale a 1 allora prendo la pagina e' la inserisco nella lista (in fondo), 
        altrimenti la rimuovo.
        Se tutte le pagine sono state referenziate allora alla fine diventa FIFO puro.
    \subsubsection{Algoritmo di Clock}
        Ho una lista a cerchio di pagine, ad ogni tick vado avanti di pagina e controllo il bit R come in FIFO.
        E' piu efficiente rispetto a FIFO e Second Chance dato che non devo fare operazioni di inserimento sulla lista.

    \subsubsection{LRU}
        Assumiamo che le pagine meno usate di recente sono le candidate ad essere eliminate.
        A livello di costo computazionale e' inefficiente: dovrei mantenere una lista di pagine ordinate per utilizzo,
        per essere efficiente richiede hardware aggiuntivo.
    
    \subsubsection{NFU}
        L'algoritmo NFU e' il migliore, perche' si realizza a livello software molto facilmente:
        ogni pagina ha un contatore, e ad ogni clock ognuno di questi viene incrementato usando il valore del bit R.
        Tutta via non e' ancora ottimale: le prime pagine usate dal sistema durante l'avvio avranno un contatore elevato e saranno tenuti
        quindi in memoria per tantissimo tempo anche se non verranno piu utilizzate

        \paragraph{Aging}
        Si introduce una tecnica di aging: fisso il numero di bit da usare per ogni contatore, e ad ogni ciclo faccio lo shift verso destra
        del contatore e poi aggiungo il bit R al lato sinistro.
        
    \subsubsection{Working Set}
        Un Working Set e' l'insieme delle pagine usato da un processo durante una fase dell'esecuzione.
        All'avvio di un programma si verificano molti page fault.
        Andiamo a definire iil working set: $w(k,t)$
        \begin{itemize}
            \item $w(k,t)$ e' l'insieme di pagine usate negli ultimi $k$ riferimenti all'istante $t$.
            \item $w(k,t)$ e' monotona non decrescente al crescere di $k$
            \item Il limite di $w(k,t)$ e' finito, e' correllato allo spazio degli indirizzi del programma.
        \end{itemize}

        esiste un'ampio valore $k$ per cui il working set non varia, possiamo fare una stima di quali pagine
        servano ad un processo quando viene riavviato.

        L'algoritmo basato su working set funziona cosi: piuttosto che ragionare per ultimi riferimenti, semprende
        in esame il tempo di esecuzione, quindi quante pagine sono state usate negli ultimi $\tau$ secondi.

        Un ciclo periodico resetta il bit R, e controllo le pagine: se R=1 vuol dire che la pagina e' nel working set e aggiorno il suo istante di utilizzo
        , se R=0 e $Eta>\tau$ allora la pagina non e' nel working set e posso rimuoverla altrimenti, la contrassegno come vecchia per possibile rimozione.
        Se nessuna paggina e' rimuovibile allora seleziono la piu vecchia con R=0. 

        Questo algoritmo puo' essere migliorato usando una struttura a cerchio: \textbf{WS Clock}, ad ogni tick controllo se la pagina 
        ha bit R=1 allora lo imposto a 0 e vado alla prossima, altrimenti se R=0 controllo l'eta e in caso la sfratto. Devo controllare anche se \textit{la pagina e' stata modificata}:
        se lo e' e ha bit R=0, allora devo schedulare la sua scrittura su disco, ma a lungo andare potrei aver schedulato moooolte scritture, quindi imposto un limite di scritture da schedulare.
        Continuo al ciclare sulla struttura finche' una delle pagine che ho schedulato non e' stata scritta sul disco.
    \section {Progettazione della Paginazione}
        
\subsection{Allocazione globale e locale}
Quando si verifica un page fault ho due scelte:
\begin{itemize}
    \item \textbf{Allocazione Locale}: elimino una pagina del processo che ha generato il page fault
    \item \textbf{Allocazione Globale}: elimino una pagina presa tra tutti i processi in esecuzione
\end{itemize}

L'allocazione locale e' semplice da implementare ma porta al piu' di frequente al trashing nel caso in 
cui il working set di un processo sia variabile e cresce piu della memoria allocata.

L'allocazione globale e' piu efficiente perche' si adatta meglio nel caso in cui il working set varia nel tempo ed inoltre:
\begin{itemize}
    \item Il sistema operativo puo assegnare dinamicamente i frame ai processi
    \item I bit di aging non bastano per fare una stima dei cambiamenti sul working set.
\end{itemize}

\subsection{Allocazione equa e proporzionale}
Se uso un tipo di \textbf{allocazione equa}, quindi uniforme tra tutti i processi, non posso tenere conto
delle diverse esigenze dei processi, invece \textbf{l'allocazione proporzionale} rispecchia meglio 
le necessita di ogni processo. Nota che l'allocazione proprozionale e' possibile solo con l'allocazione globale.

Bisogna assicurarsi che ogni processo abbia \textit{abbastanza pagine per eseguire le \textbf{istruzioni fondamentali}},
e bisogna prevenire invece le istruzioni che richiedono piu pagine.

\subsection{Allocazione dinamica e PFF}
Inizialmente ogni processo ha tante pagine in base alla sua grandezza, e aggiorno dinamicamente 
l'allocazione in base all'evoluzione dell'esecuzione.
Con \textbf{PFF} monitoro la frequenza dei page fault per ogni processo, e di conseguenza aumento le pagine 
assegnate ad ogni processo.

osservazione: piu frame ha un processo, meno page fault si generano.

Tuttavia anche implementando il migliore degli algoritmi possible, inevitavilmente se il carico
e' troppo elevato per la macchina, questa andra in trashing. Che faccio?
\begin{itemize}
    \item \textbf{Out Of Memory Killer}: ammazzo i processi piu cattivi (mantengo una specie di punteggio)
    \item \textbf{Swapping}: sposto i processi in memoria non volatile, evito di uccidere i processi.
\end{itemize}

\subsection{Scheduling a Due Livelli}
Nei sistemi interattivi molti processi sono in background e per la maggiorparte del tempo non fanno niente finche
l'utente non fa qualcosa o si attivano ogni tot secondi: li butto in memoria non volatile.
Per fare questo devo considerare se: sto considerando un processo CPU bound o IO bound, e qual'e la dimensione e freqiemza dei processi.

Oltre ad applicare lo scheduling a due livelli, posso considerare di compattare o comprimere le pagine da mettere in memoria non volatile. 

\subsection{Policy di Pulizia e Paging Daemon}
Il \textbf{pagin daemon} e' un processo in background che ogni tanto si sveglia e controlla
lo stato della memoria. Il paging daemon migliora le performance dell'algoritmo di aging, che performa 
meglio quando ci sono pagine libere in memoria (\textbf{policy di pulizia}) (altrimenti dovrei sfrattarne una, se questa e' stata modificata dovrei scriverla in memoria,
aggiornandone i dati, se e' gia presente ed e' pulita allora piscio). 

\subsection{Dimensione delle pagine}
Devo scegliere la giusta dimensione porcozzio. Pagine piu piccole garantiscono minore frammentazione ma tabelle
delle pagine piu grandi. La dimensione ottimale si trova bilanciando \textit{frammentazione interna e overhead della tabella
delle pagine}.

Posso scegliere di sfruttare pagine di grandezze diverse: ad esempio una grandezza maggiore per il kernel.

\subsubsection{Calcolo della dimensione ottimale}
I dati del problema sono:
\begin{itemize}
    \item Dimensione media del processo: $s$ byte.
    \item Dimensione della pagina: $p$ byte, \textbf{da calcolare!}
    \item Dimensione di ogni voce nella tabella delle pagine: $e$ byte.
\end{itemize}

Devo calcolare il mio overhead nella tabella delle pagine:
\begin{itemize}
    \item ogni processo ha: $\sim s/p$ pagine.
    \item la tabella delle pagine occupa quindi: $s \cdot e /p$ byte.
    \item La memoria sprecata nell'ultima pagina dovuta alla frammentazione e': $p/2$
\end{itemize}

L'overhead totale e' $s \cdot e /p + p/2$. Derivando rispetto a $p$ e impostando l'uguaglianza a zero
ottengo che: $p=\sqrt{2se}$ .

\subsection{Spazio indirizzi separato}
Si puo' scegliere di separare lo spazio degli indirizzi tra il testo del programma e i suoi dati, rispettivamente:
\textbf{I-space} e \textbf{D-space}

\subsection{Condivisione delle pagine}
In contesti di multiprogrammazione piu processi vorebbero condividere la stessa pagina, capita spesso che 
due processi magari leggano lo stesso testo del programma: per facilitare cio' si usa lo spazio degli
indirizzi separato, cosi possono avere stesso testo ma dati diversi, oppure dati uguali.

Quindi due processi potrebbero condividere lo stesso spazio degli indirizzi per il testo. Ma nel caso di una
\texttt{fork} sono condivisi anche i dati. Se $A$ modifica una pagina condivisa con $B$, allora 
il sistema operativo genera un trap, crea una copia della pagina modificata che solo $A$ puo' vedere.

Bisogna fare attenzione a non rimuovere le pagine condivise tra piu processi quando avviene un page fault, senno vaffanculo!

\subsubsection{Librerie Condivise}
Il sistema operativo condivide automaticamente le pagine di testo usate piu comunemente: ossia le DDL, o shared libs.
Le librerie per poter funzionare sono compilate usando indirizzi relativi, una volta caricata la libreria in memoria
aggiungo l'indirizzo base per accedere ad una determinata procedura.

\subsubsection{File Mappati in Memoria}
Posso mappare un file ad una pagina, o meglio, allo spazio degli indirizzi. Una volta che ho finito di
leggere e scrivere sul file mappato, posso applicare le modifiche sul file.

\subsection{Problemi di implementazione}
Il sistema operativo dovrebbe:
\begin{itemize}
    \item Determinare le dimennsioni iniziali del programma e dei dati
    \item Creare la tabella delle pagine
    \item Allocare spazio per lo swap
\end{itemize}

ed ogni volta che eseguo un processo devo:
\begin{itemize}
    \item Azzerare la MMU e svuotare il TLB.
    \item Caricare la tabella delle pagine del processo
    \item Precaricare alcune pagine per evitare page fault.
\end{itemize}

ed ogni volta che ho un page fault devo:
\begin{itemize}
    \item Determinare l'indirizzo che ha causato il fault
    \item Trovare la pagina nella memoria non volatile
    \item Scegliere un frame da usare e caricare la pagina nel frame
\end{itemize}

ed alla chiusura del processo:
\begin{itemize}
    \item Rilasciare la tabella delle pagine e le pagine caricate
    \item Capire quali pagine sono condivise con altri processi.
\end{itemize}

\subsubsection{Gestione della page fault}
\begin{enumerate}
    \item \textbf{Trap nel kernel}: il contatore del programma viene salvato nello stack.
    \item \textbf{Avvio una routine} per salvare i valori dei registri, e invoco il gestore dei page fault
    \item \textbf{Trovo la pagina} virutale mancante (nei registri hardware o analizzo il testo del programma)
    \item \textbf{Controllo} se l'indirizzo richiesto e' valido
    \item Se non ci sono frame liberi eseguo un \textbf{algoritmo di sostituzione}, se la pagina e' sporca la devo scrivere in memoria non volatile.
    \item \textbf{Carico} la pagina nel frame
    \item \textbf{Aggiorno} la tabella delle pagine
    \item \textbf{Ripristino} lo stato del programma in modo da rieseguire l'istruzione
    \item \textbf{Schedulo} il processo per l'esecuzione
    \item Rieseguo!
\end{enumerate}

\subsection{Bloccare le pagine in memoria durante I/O}
Per colpa dello scheduler, potrebbe accadere che un processo richieda una pagina andando a sfrattare 
quella che sta usando un dispositivo di I/O che sta scrivendo usando il DMA. Otterrei dati invalidi, quindi
devo bloccare o \textbf{"pinnare"} le pagine che sono bloccate da I/O.

\subsection{Gestione dello swap}
In UNIX le pagine soggette a swap sono inserite in una partizione speciale, con un file system semplificato.
Oppure potrei avere l'allocazione in memoria di scambio o ad ogni processo associo un'area di scambio (calcolabile tramite offset).

Bisogna tenere pero' conto del fatto che le dimensioni dei programmi in memoria possono crescere: \textit{riservo aree separate per testo, dati e stack}.

\subsection{Segmentation}
Devo evitare che i programmi si sovrappongano tra di loro per via della crescita. Introduco degli \textit{indirizzi virtual multipli e indipendenti}: \textbf{segmenti}
Ogni segmento rappresenta uno \textbf{spazio degli indirizzi separato} dagli altri segmenti. 
Quando faccio un riferimento in memoria ho bisogno del: \textit{numero di segmento e del indirizzo nel segmento}.

La segmentazione facilita la condivisione delle risorse, permette ai programmi di crescere, e permette di applicare delle
misure di sicurezza.

Un compilatore per gestire tutti i simboli che incontra puo' eseguire in modo efficiente con la segmentazione:
ogni lista di simboli si trova in un segmento.

Nota:
\begin{itemize}
    \item il programmatore deve manipolare direttamente i segmenti
    \item i segmenti hanno grandezza variabile, le pagine no
\end{itemize}

Con la segmentazione ho lo stesso problema delle pagine: si frammenta la memoria. \textbf{Checkerboarding} si
risolve compattando la memoria.

\subsection{multics}
Ma chi se lo incula multics

    \newpage
    \section {File System}
        Bisogna andare in contro ai problemi della memorizzazione:
\begin{itemize}
    \item La RAM ha uno \textbf{spazio limitato}, e le sue informazioni si perdono.
    \item Devo garantire l'\textbf{accesso concorrente} alle informazioni.
\end{itemize}

La memorizzazione a lungo termine deve:
\begin{itemize}
    \item Gestire tante informazioni.
    \item Devono persistere sul disco.
    \item Deve essere condivisa tra processi.
\end{itemize}

\subsection{File System}
Un disco e' una sequenza lineare di \textbf{blocchi}: posso scrivere o leggere su \texttt{k}.
L'astrazione del file risove tutti i problemi elencati sopra, il sistema operativo quando gestisce un file
si occupa della: \textbf{struttura, denominazione, accesso, protezione e implementazione.}

Un file system ha informazioni visibili all'utente e altre nascoste (come la gestione della memoria e la sua struttura).

\begin{definition} File System \\
    Un modo per \textbf{organizzare} e memorizzare le informazioni. \\
    Sono un'astrazione per i dispositivi: Dischi Rigidi, SSD, ecc...
\end{definition}

\subsection{Nomi dei file}
I file sono l'astrazione che permettono di salvare e leggere su disco, nascondendo i dettagli tecnici all'utente.
\begin{itemize}
    \item \textbf{Nomenclatura dei file}: I file sono identificati tramite nomi (le possibilita dipendono dal file system)
    \item \textbf{Lunghezza}: a seconda del sistema operativo il nome di un file potrebbe avere una lunghezza massima.
    \item \texttt{ext4}: non posso usare certi nomi speciali. \texttt{FAT12} non posso usare alcuni caratteri.
    \item \textbf{Case Sensitive?}
\end{itemize}

L'estensione di un file puo' essere convenzionale, come per esempio in UNIX. In Windows invece 
l'estensione specifica quale programma puo' aprire il file.

\subsection{Struttura Di Un File}
\paragraph{Sequenza Non Strutturata Di Byte}
In questo caso i file sono una serie di byte senza significato, i programmi utente li devono interpretare: ricade nel caso di 
\textbf{UNIX, Linux, MacOS e Windows}.

\paragraph{Sequenza di Record di Lunghezza Fissa}
Il modello storico basato sulle vecchie schede perforate

\paragraph{Fle come Albero di Record}
Il file e' organizzato come un'albero

\subsection{Tipi di File}
In Unix e Windows esistono \textbf{File Normali} e \textbf{Directory}, le directory sono file che mantentono
una gerarchia dei file. \\
Ci sono due tipi di file \textbf{speciali}:
\begin{itemize}
    \item \textbf{file a caratteri}: per gestire dispositivi \textit{I/O}.
    \item \textbf{file speciali a blocchi}: per gestire i \textit{dischi}.
\end{itemize}

I File Normali possono essere interpretati come:
\begin{itemize}
    \item \textbf{ASCII}: file di testo.
    \item \textbf{Binari}: non leggibili come testo, vanno interpretati dai programmi.
\end{itemize}

\paragraph{File Eseguibilie}
Un file eseguibile e' una sequenza di byte ben strutturata, le sue \textbf{componenti} sono:
\begin{itemize}
    \item \textbf{Numero Magico}: il file e' eseguibile?
    \item \textbf{Intestazione/Header}: Punto di ingresso, dimensioni di ogni componente del file.
    \item \textbf{Testo e Dati}: Le parti del programma da caricare in memoria
    \item \textbf{Tabella dei simboli}: usata per il debug.
\end{itemize}

\paragraph{Archivio}
Sono file binari (interpretati come testo non avrebbero senso), che raccolgono procedure di libreria compilate
che devono essere collegate.

\paragraph{Accesso ai file}
L'accesso per via dell'hardware era \textbf{sequenziale} (si leggevano dei nastri magnetici).
Con l'avvento dei dischi si poteva leggere con \textbf{accesso casuale}.

Per leggere posso indicare una posizione specifica, dove inzia la lettura oppure con \texttt{seek}
posso impostare la posizione nel file corrente che sto leggendo: implementato sia un Unix che Windows.

\paragraph{Attributi comuni dei file.}
\begin{itemize}
    \item \textbf{Protezione e Accesso}: chi puo accedere al file? Posso leggere, scrivere e eseguire?
    \item Flag di controllo: sola lettura, nascosto, file di backup.
    \item Binario o ASCII?
    \item Attributi Temporali: Data e Ora, Ultimo accesso, Ultima modifica
    \item Dimensione.
\end{itemize}

\subsection{Programmazione!}
\subsubsection{Leggere un file}
\begin{lstlisting}[language=c]
    int fd = open("foo.txt", O_RDONLY);
    char buf[512];
    ssize_t bytes_read = read(fd, buf, 512);
    close(fd);
    printf("read %zd: %s\n", bytes_read, buf);
\end{lstlisting}

\subsubsection{\texttt{lseek}}
\begin{lstlisting}[language=c]
    int fd = open("foo.txt", O_RDONLY);
    lseek(fd, 128, SEEK_CUR);
    char buf[8];
    read(fd, buf, 8);
    close(fd);
\end{lstlisting}

Sposta di 128 byte in avanti la posizione nel file, a partire dalla posizione corrente.

\subsubsection{Scrivere}

\begin{lstlisting}[language=c]
    int fd = open("foot.txt", O_WRONLY |
         O_CREAT | O_TRUNC);
    char buf[] = "Dio Porco!";
    write(fd, buf, strlen(buf));
    close(fd);
\end{lstlisting}

La flag \lstinline{O_TRUNC} indica che se il file esiste lo ridimensiono a 0. 

\subsection{Directory}
\begin{definition} Directory \\
Sono file che tengono traccia degli altri file nel file system. 
\end{definition}

In generale una singola directory, denominata \texttt{root}, contiene tutti gli altri file: organizzazione a \textbf{Singolo Livello}.
L'organizzazione in directory offre semplicita e rapidita nella locazione di un file. 

E' da notare che l'uso dei file come astrazione e' ampiamente utilizzata: dispostivi embedded come fotocamere e lettori mp3, e tecnologie \textbf{RFID}.
L'idea dei file e' antica, ma si rinnova ad ogni generazione.

L'organizzazione a singolo livello non e' pratica per noi umani che teniamo vogliamo tenere in memoria
migliaia di file: si introduce una struttura ad albero per organizzare logicamente i file.

Di solito ogni utente ha una sua cartella privata per i suoi dati.

I percorsi che portano ad un file possono essere:
\begin{itemize}
    \item \textbf{Assoluti}: utile per caricare le procedure di libreria
    \item \textbf{Relativi}: e' legata alla \textbf{directory di lavoro}
\end{itemize}

Le operazioni possibili su una directory sono:
\begin{itemize}
    \item \texttt{create}: crea una directory vuota.
    \item \texttt{delete}: elimina la directory solo se e' vuota.
    \item \texttt{opendir}: apri e leggi il contenuto.
    \item \texttt{closedir}: chiudila e libera le risorse.
\end{itemize}

L'istruzione \texttt{readdir}, restituisce le prossima voce in una directory aperta.

\subsection{Archiviazione}
I file TAR (Tape Archive) servono per raccogliore piu file in un unico archivio, mantenendone la struttura
e i permessi originali. 

Di solito, dopo aver archiviato dei file, si comprimono con \texttt{gz}: il suo spazio viene ridotto.

\begin{lstlisting}[language=bash]
    tar -czvf nome-archivio.tar.gz /percorso/della/cartella
\end{lstlisting}
Dove:
\begin{itemize}
    \item \texttt{c}: crea un nuovo archivio
    \item \texttt{z}: comprimi usando l'algoritmo gzip
    \item \texttt{v}: verbose
    \item \texttt{f}: specifica il nome del file di archivio
\end{itemize}

Per "aprire" un archivio tar compresso con gzip, invece:
\begin{lstlisting}[language=bash]
    tar -xzvf nome-archivio.tar.gz
\end{lstlisting}
Simile a prima, l'opzione \texttt{x} indica che voglio estrarre. 

Altri comandi e formati per comprimere e creare archivi:
\begin{itemize}
    \item \texttt{zip}: riduce la dimensione dei file singolarmente e poi li archvia, e' ampiamente supportato.
    \item \texttt{unzip}: decomprime e estrare gli archivi \texttt{zip}
    \item \texttt{tar.gz}: raccoglie i file in un unico archivio e lo comprime. La compressione e' elevata, ed e' in grado di mantenere la sua struttura e i permessi
\end{itemize}

In conlusione \texttt{tar.gz} ha un tasso di compressione piu elevato rispetto a \texttt{zip}, ma quest'ultimo e' piu veloce nel crearli.
Tutta via \texttt{tar.gz} mantiene i meta dati dei file, mentre \texttt{zip} e' supportato su diverse piattaforme.
        
    \newpage
    \section {Implementazione Del File System}
        \subsection{Come memorizzare i file?}
\begin{definition} File System \\
    Il file system e' il metodo utilizzato per organizzare e memorizzare dati sui dispositivi di memoria.
\end{definition}

\begin{definition} Partizioni \\
    Un disco puo' essere suddiviso in piu' partizioni, ciascuna con un proprio file system, \textbf{indipendente}.    
\end{definition}

\subsubsection{MBR - Master Boot Record}
L'MBR si trova nel settore 0 del disco ed e' essenziale per avviare il computer, contiene
la tabella delle partizioni, dove ogni \texttt{entry} ha informazioni su inizio e fine della partizione.

All'avvio del sistema il \textbf{BIOS} \textit{legge ed esegue l'MBR} che trova la partizione \textbf{attiva}, quella da usare per caricare il \textbf{boot block}
per avviare il sistema operativo.

\begin{quote}
    Ogni partizione inizia con un \textbf{boot block} anche se non ce l'ha!
\end{quote}

\subsubsection{Nuova scuola: UEFI (Unified Extensible Firmware Interface)}
E' un sistema che sostituisce il vecchio BIOS. Permette:
\begin{itemize}
    \item Avvio piu veloce: inizializzazione dell'hardware ottimizato.
    \item Comptabilita architetture 32 e 64 bit.
    \item Supporto di Interfaccia utente avanazata e mouse.
    \item Secure Boot
\end{itemize}

UEFI funziona con GPT: posso avere dischi piu grandi di \texttt{2,2 TB}.
\paragraph{GUID Partition Table (GPT)}
E' un sistema avanza di gestione delle partizione, supporta dischi enormi, 
puo' gestire un numero illimitato di partizioni (limitato dal sistema operativo che eseguiremo), include un sistema
di \textbf{backup della tabella delle partizioni} e fa anche il \textbf{CRC}. \\

Nei dischi GPT esiste la \textbf{EFI System Partition} (ESP), una partizione speciale che contiene i file di avvio: bootloader e driver, utilizza
il file system \textbf{FAT32} in modo da essere compatibile con UEFI. 

\paragraph{Secure Boot}
Impedisce l'avvio di software non autirizzato: questo software controllare le firme digitali
del \textit{boot loader, del sistema operativo e dei driver}, in modo da avviare solo software con firme
autentiche: \textit {previene l'esecuzione di malware e rootkit all'avvio}.

Lo svantaggio e' che alcuni sistemi operativi le cui firme non sono riconosciute potrebbero non avviarsi.

\subsection{Implementazione dei file}
Bisognia gestire l'associazione tra \textit{file e blocchi sul disco}. Ci sono vari tipi di allocazione 
per i file.

\subsubsection{Allocazione contigua}
I file sono memorizzati come sequenze contigue sul disco: \textit{un file da 50 blocchi, occupera 50 blocchi
consecutivi sul disco}.
Questo metodo e' semplice da implementare ed ha un'alta efficienza di lettura, con una sola lettura posso leggere il file
senza dover cambiare posizione durante la lettura (meglio per i dischi meccanici).  \\

Si ha pero' un problema analogo a quello della RAM: come gestisco la frammentazione? Potrebbe essere 
difficile trovare spazi per infilare i miei file, oppure anche se c'e' abbastanza spazio e' talmente frammentato
da non avere blocchi continui a disposizione.

\subsection{Allocazione a Liste Concatentate}
i file sono \textit{liste concatenate di blocchi sul disco}, ogni blocco ha un \textbf{puntatore} al blocco successivo. \\

Permette di usare meglio lo spazio sul disco, posso facilmente implementare la struttura a directory ma:
\begin{itemize}
    \item richiede di fare molti accessi casuali (devo chiamare \texttt{seek}).
    \item parte dei bit nel blocco sono usati per gestire i puntatori, quindi un blocco potrebbe non contenere dati che occupano una \textbf{dimensione standard}, ossia una potenza di 2. 
\end{itemize} 

\subsection{FAT}
Uso una File allocation table, caricata in RAM! Questa contiene le sequenze di blocchi che compongono un file.
E' efficiente ma per dischi troppo grandi si usa troppa RAM, la tabella e' allocata interamente per tutti i blocchi, anche
quelli inutilizzati, si usa piu che altro per chiavette usb e schede sd di piccole dimensioni.

\subsection{I-NODE}
\begin{definition} I-NODE
    Un Index-Node e' una struttura dati fondamentale in \texttt{ext2/ext3/ext4} (usato in Linux). 
    Contiene le informazioni su un file, \textit{tranne il nome e il contenuto}, ma ne comprende i metadati.
\end{definition}

Ogni file e directory e' rappresentato da un i-node, questi sono organizzati in una tabella. 
La tabella FAT rispetto a quella degli i-node non permette di gestire in modo ottimale i meta dati (altrimenti la grandezza della tabella in RAM crescerebbe).
\textit{I sistemi i-node sono efficienti su dischi grandi}. \\

Un'i-node e' una struttura, di \textbf{dimensione fissa}, che contiene attributi e \textit{gli indirizzi a blocchi} di un file, ad ogni i-node corrisponde un file.
\textit{Rispetto alla FAT, solo gli i-node dei file aperti si trovano in RAM.} \\

Dato che gli i-node hanno dimensione fissa, per i file che hanno tanti blocchi, l'ultimo indirizzo di un i-node potrebbe
puntare per esempio ad un \textit{blocco contentente altri puntatori}, posso gestire efficaciemente file grandi. \\

\subsection{Implementazione Delle Directory}
Le directory mappano i nomi ASCII dei file sulle informazioni per trovare i dati sul disco (primo blocco da leggere o i-node corrispondente).

Come alloco i dati sulla directory? Posso avere un tipo di associazione: \texttt{nome file|attributi} oppure \texttt{nome file|i-node}.
Ma i nomi dei file hanno lunghezze variaibile, di solito si imposta un limite massimo di caratteri: 255. Ma pochi file usano tutti e 255 
i caratteri quindi e' saggio usare delle voci con \textbf{lunghezza variabile}:
\begin{itemize}
    \item Ogni voce ha un header che dice quanto e' lungo il nome del file. \textit{Ma quando elimino file e cartelle si creano buchi di lunghezza variabile!}
    \item Il nome dei file e' contenuto in un \textbf{heap}, subito dopo la fine della directory. Cosi ogni entry ha lunghezza fissa.
\end{itemize}

Ora sorge un'altro problema: la ricerca del file in una directory e' lineare. Usiamo quindi una
tabella di \texttt{hash}, quindi se ho $n$ file, per ognuno di questi grazie ad una funzione di hash
sono in grado di accedere in tempo costante alla lista.

Se a piu file corrisponde lo stesso indice dopo che ho applicato l'hasing, uso delle liste concatenate.
Per velocizzare le cose posso usare una cache che e' efficiente se richiedo un piccolo insieme di file, bisogna
notare che stiamo \textit{aggiungendo complessita al nostro problema}, magari per directory piccole non serve
fare tutto sto casino.

\subsection{Link}
I file possono essere condivisi tra piu utenti tramite:
\begin{itemize}
    \item Hard Link: file che puntano \textbf{all'i-node} condiviso.
    \item Soft Link: file che puntano \textbf{al nome} di un file condiviso.
\end{itemize}

L'\textbf{Hard Link} e' ideale quando ho molti proprietari, l'i-node viene eliminato quando non ci sono piu'
Hard Link a questo, indipendentemente dal numero di Hard Link ho comunque un solo i-node.
L'eliminazione di un i-node si puo' effettuare quando per esempio il contatore di hard link a questo va a 0. \\

Il \textbf{Soft Link} e' piu' rognoso. Per ogni soft link ho un i-node che punta al percorso del file condiviso.

I link introducono un problema: se faccio il backup del disco e trovo i link che devo fa?

\subsection{Spazio del Disco}
Come tengo traccia dello spazio libero?
\paragraph{Lista Concatenata}
Uso una Lista Concatenata di blocchi, questi contengono gli indirizzi dei blocchi liberi.
In caso un blocco non basti a referenziare tutti i blocchi liberi, questo punta ad un altro blocco contenente
altri blocchi liberi.

\paragraph{Bitmap}
Uso una bitmap per tracciare i blocchi liberi, quindi un bit per ogni blocco del disco, generalmente
richiede meno spazio rispetto alla lista concatenata.

\paragraph{Ottimizzazioni sulla lista concatenata}
Conviene che ciascun blocco ha un \textbf{contatore} che segna i bit liberi, e un numero che rappresenta
il \textit{blocco a partire dal quale sono liberi}

\paragraph{Free List}
Uso una lista concatenata di puntatori: "free list". \textit{Mantengo solo un blocco di puntatori in memoria} e uso questo per
prendere i blocchi da usare.

Per migliorare le performance posso dividere un blocco di puntatori in due e metterlo tra ram e rom. (????)

\subsection{Meccanismo di Quote}
L'amministratore di sistema assegna a ogni utente un \textbf{massimo} \textit{di file e blocchi da usare}.
Il sistema operativo deve controllare che le quote siano rispettate:
\begin{itemize}
    \item Ho una tabella dei file aperti, ognuno ha un proprietario.
    \item Una tabella delle quote tiene conto delle statistiche di ogni utente.
    \item Nella tabella dei file, ad ogni entry corrisponde un puntatore ad una entry nella tabella delle quote.
\end{itemize}
Per ogni utente ci sono dei limiti  \textbf{soft} e \textbf{hard} da rispettare.

\subsection{\texttt{ext2}}
Le componenti  chiave sono:
\begin{itemize}
    \item \textbf{Superblocco}: layout della partizione, numero di I-node, e numero di blocchi.
    \item \textbf{Descrittore del Gruppo}: dettagli su blocchi liberi, I-node
    \item \textbf{Bitmap}: traccia i  blocchi liberi e gli i-node liberi
\end{itemize}

    
    \newpage
    \section {Programmazione concorrente}
        Nota: \texttt{unistd.h} contiene tutte le API richieste dallo standard POSIX,
cioe' i wrapper per le chiamate di sistema richieste da POSIX.

\subsection{\texttt{write}}
\begin{lstlisting}[language=c]
    #include <unistd.h>

       ssize_t write(int fd, const void buf[.count], size_t count);
\end{lstlisting}

Scrive \texttt{count} byte presi dal buffer in input.

Esempio:
\begin{lstlisting}[language=c]
    #include <unistd.h>

    char msg[] = "Hello World!\n";
    write(STDOUT, msg, sizeof(msg));
\end{lstlisting}

Esempio deprecato (pericoloso), bypasso il la funzione wrapper:
\begin{lstlisting}[language=c]
    #include <unistd.h>
    #include <stdio.h>
    #include <sys/syscall.h>

    char msg[] = "Hello World!\n";
	int nr = SYS_write;
	syscall(nr, STDOUT, msg, sizeof(msg));
\end{lstlisting}

\subsection{\texttt{fork}}
\begin{lstlisting}[language=c]
    #include <stdio.h> 
    #include <stdlib.h>
    #include <unistd.h> 
    int main(){

        int pid, child_status;

        if ((pid = fork()) == 0) {
            // Codice del figlio
        } else { 
            // Codice del padre
            wait(&child_status);
        }
    }
\end{lstlisting}

\subsection{\texttt{fork}, \text{dup}, \text{execl}, \text{waitpid}}

\begin{lstlisting}[language=c]
    #include <unistd.h>

    int dup(int oldfd);
    int dup2(int oldfd, int newfd);
\end{lstlisting}

\texttt{dup} crea un nuovo file descriptor che punta allo 
stesso file del descrittore \texttt{oldfd}. Il risultato e' che
se prima chiudo un file descriptor e poi chiamo \texttt{dup}, 
usare il file descriptor chiuso equivale ad utilizzare il nuovo file descriptor.
Infatti, puntano entrambi allo stesso file.

Per meglio precisare: il vecchio file descriptor scelto da \text{dup} e' quello 
piu' basso che e' stato chiuso.

\begin{lstlisting}[language=c]
    #include <stdio.h> 
    #include <stdlib.h>
    #include <unistd.h> 
    int main(){

        pid_t cat_pid, sort_pid;
        int fd[2];

        pipe(fd);
        
        cat_pid = fork();
        if(cat_pid==0){
            // processo cat
            close(fd[PIPE_RD]);
            close(STDOUT);

            // ora l'output e' dirottato sulla pipe
            dup(fd[PIPE_WR]);
            execl("/bin/cat", "cat", "names.txt", NULL);
        }

        sort_pid = fork();

        if(sort_pid == 0){
            close(fd[PIPE_WR]);
            close(STDIN);

            dup(fd[PIPE_RD]);

            execl("/usr/bin/sort", "sort", NULL)
        }

        // chiude il file descriptor
        close(fd[PIPE_RD]);
        close(fd[PIPE_WR]);

        // aspetta che il processo termina
        waitpid(cat_pid, NULL, 0);
        waitpid(sort_pid, NULL, 0);
    }
\end{lstlisting}

\subsection{\texttt{signal}}
\begin{lstlisting}[language=c]
    #include <signal.h>
    #include <unistd.h>
    #include <string.h>

    void sigHandler(int signum){
        printf("Interrupt sig %d: %s", signum, 
            strsignal(signum));
        exit(signum);
    }

    int main(){
        signal(SIGINT, sigHandler); // CTRL+C

        while(1){
            printf("Sleep\n");
            sleep(1);
        }

        return 0;
    }
\end{lstlisting}

Si puo' fare la stessa cosa ma con \texttt{SIGALARM}.
Questo segnale viene scatenato dalla funzione \texttt{alarm(1)}.

\subsection{Shell}
\begin{lstlisting}[language=c]
#include <stdio.h>
#include <stdlib.h>
#include <string.h>
#include <sys/types.h>
#include <sys/wait.h>
#include <unistd.h>

int main(){
    while(1){
        char cmd[256], *args[256];
        int status;
        pid_t pid;

        read_command(cmd, args);

        if (strcmp(cmd,"exit") == 0) 
            exit(0);
        
        pid = fork();
        if(pid < 0){
            perror("fork failed");
            exit(0);
        }
        if (pid == 0){
            if(execvp(cmd, args) == -1){
                perror("...");
                exit(0);
            }
        } else {
            if (wait(&status) == -1){
                perror("...");
                exit(0);
            }
        }

        free_args(args);
    }
}
\end{lstlisting}

\texttt{perror} scrive un messaggio nello \texttt{stderr}. \\
\texttt{wait} aspetta che un figlio cambi stato.

\begin{lstlisting}[language=c]
void read_command(char *cmd, char **args){
    char* input = NULL;
    size_t len = 0;
    printf("myshell> ");
    ssize_t read_len = 
        getline(&input, &len, stdin);

    if(read_len == -1){
        // CTRL-D pressed.
        printf("\n");
        exit(0);
    }

    chat* token = strtok(input, " \n");

    int i = 0;
    while(token){
        args[i] = strdup(token);
        if(i == 0)
            strcpy(cmd, token);
        token = strtok(NULL, " \n");
        i++;
    }
    args[i] = NULL;
    free(input);
}

void free_args(char** args){
    int i = 0;
    while(args[i]){
        free(args[i])
        i++;
    }
}
\end{lstlisting}

\texttt{strtok} prende la sequenza di caratteri contenuta in uno 
dei token forniti. Con \lstinline{strtok(NULL, " \n")} ottengo la sequenza successiva.

    \section {Programmazione: Thread e sincronismo}
        \subsection{\texttt{pthread\_create}, \texttt{pthread\_join}, \texttt{pthread\_exit}}

\texttt{pthread\_join} aspetta che il thread specificato termini, posso anche
fornire un puntatore alla variabile che conterra' il valore di ritorno. \\

\texttt{pthread\_create} crea il thread. Puo' richiedere un oggetto \texttt{pthread\_attr\_t}
con gli attributi di inizializzazione del thread. 

\begin{lstlisting}[language=C]
#define NUMBER_OF_THREADS 10

void *print_hello_world(void * tid){
    int r = rand()%5;
    sleep(r);
    printf("Thread: %d\n", tid);
    pthread_exit(NULL);
}

int main(int arg, char* argv[]) {
    pthread_t threads[NUMBER_OF_THREADS];
    int status, i;

    for(i=0; i < NUMBER_OF_THREADS; i++){
        status = pthread_create(&threads[i], NULL, 
            print_hello_world, (void *)i);

        if(status == 0)
            printf("thread %d created, status %d\n", 
                i, status);
        else {
            printf("error: thread %d\n", i);
            exit(-1);
        }
    }

    for (i=0; i < NUMBER_OF_THREADS; i++) {
        status = pthread_join(threads[i], NULL);
        printf("%d terminated, status %d", i, status);

        if(status != 0){
            printf("%d has a problem\n", i);
            exit(-1);
        }
    }
    return 0;
}
\end{lstlisting}

\subsection{Semafori}
\texttt{mutex, empty e full} sono i semafori che regolano l'accesso al buffer. 
Il produtture prima di inserire un oggetto nel buffer controlla il semaforo \texttt{empty} e \texttt{mutex}.
Il consumatore invece controlla il semafor \texttt{full} 

\begin{lstlisting}[language=c]
#include <pthread.h> 
#include <stdio.h> 
#include <stdlib.h>
#include <unistd.h>
#include <semaphore.h>

#define N 10
#define TRUE 1

sem_t mutex;
sem_t empty;
sem_t full;

int buffer[N];
int in = 0;

void down(sem_t *sem){
    sem_wait(sem);
}

void up(sem_t *sem){
    sem_post(sem);
}

void insert_item(){
    int item = rand() % 100;
    printf("%d, %d", item, n+1);
    buffer[in] = item;
    in++;
}

void remove_item() {
    printf("%i", in);
    int item = buffer[in - 1];
    printf("\nPrelevo %d da posizione %d\n", item, in);
    in--;
}

void *producer(void *args){
    while(TRUE) {
        down(&empty);
        down(&mutex);

        insert_item();
        print_buffer();

        up(&mutex);
        up(&full);
    }
}

void *consumer(void *args) {
    while (TRUE){
        down(&full);
        down(&mutex);

        remove_item();
        print_buffer();

        up(&mutex);
        up(&empty);
    }
}

int main() {
    pthread_t prod, cons;

    sem_init(&mutex, 0, 1);
    sem_init(&mutex, 0, N);
    sem_init(&full,  0, 0);

    pthread_create(&prod, NULL, producer, NULL);
    pthread_create(&prod, NULL, consumer, NULL);

    pthread_join(prod, NULL);
    pthread_join(cons, NULL);

    sem_destroy(&mutex);
    sem_destroy(&empty);
    sem_destroy(&full);

    return 0;
}
\end{lstlisting}

\subsection{Lettori e Scrittore}
Ipotizzo di avere tanti lettori e un solo scrittore. L'accesso ai lettori e' regolato nel seguente modo:
la variabile \texttt{rc} conta quanti lettori ci sono al corrente, una volta che un lettore blocca lo scrittore 
dall'accesso al buffer, non c'e' bisogno che gli altri lettori facciano lo stesso. Solo il primo blocca \texttt{db}
e l'ultimo lettore a legger sblocca \texttt{db}.

L'unica cosa che deve essere regolata tra tutti i lettori e' l'accesso al \texttt{rc} stesso.

\begin{lstlisting}[language=c]
void *reader(void *argc){
    wihle(TRUE){
        down(&mutex);
        rc++;
        if(rc==1) down(&db);
        up(&mutex);

        read_database();

        down(&mutex);
        rc--;
        if(rc==0) up(&db);
        up(&mutex);

        use_data_read();
    }
}

void *writer(void *arg){
    while(TRUE)[
        think_up_data();
        
        // L'accesso al database e' bloccato
        down(&db);
        write_database();
        up(&db);
    ]
}
\end{lstlisting}

\subsection{Mutex}
\begin{lstlisting}[language=c]
pthread_mutex_t the_mutex;
pthread_cond_t condc, condp;

int buffer = 0;

void *producer(void *ptr){
    int i;
    for (i = 1; i <= MAX; i++) {
        pthread_mutex_lock(&the_mutex);

        while (buffer != 0)
            pthread_cond_wait(&condp, &the_mutex);
        
        buffer = i;

        pthread_cond_signal(&condc);
        pthread_mutex_unlock(&the_mutex);
    }

    pthread_exit(0);
}

void *consumer(void *ptr) {
    int i;
    for(i=1; i<=MAX; i++){
        pthreaD_mutex_lock(&the_mutex);
        while(buffer == 0){
            pthrea_cond_wait(&condc, &the_mutex);
        }

        // usa il buffer

        pthread_cond_signal(&condp);
        pthread_mutex_unlock(&the_mutex);
    }
    pthread_exit(0);
}
\end{lstlisting}

\subsection{Esercizio}
Il programma chiesto deve creare altri due processi \texttt{p1} e \texttt{p2}.
Entrambi generano dei numeri da 0 a 100, ma, p1 invia al processo padre solo
i numeri dispari, mentre p2 invia al padre solo i numeri pari.
Una volta che il processo padre ha ricevuto una coppia di numeri pari e dispari, ne fa la 
somma, se e' maggiore di 190 l'esecuzione del programma termina.

Le chiamate di sisteme e procedure necessarie per creare il codice sono:
\begin{itemize}
    \item \texttt{pipe}, per creare le due pipe, una per processo figlio.
    \item \texttt{fork}, per creare i processi figli.
    \item \texttt{perror}, per comunicare gli errori delle chiamate di sistema.
    \item \texttt{read/write}, per leggere e scrivere sui file descriptor.
    \item \texttt{kill}, per mandare il segnale di terminazione ai processi figli.
    \item \texttt{waitpid}, per aspettare che i figli terminino l'esecuzione.
\end{itemize}

Per randomizzare al meglio la generazione dei numeri si aggiora il seed con: \texttt{srand(time(NULL) \^ getpid())}.

    \section {Programmazione: File}
        \subsection{Creazione di un file}
\begin{lstlisting}[language=c]
int main(int argc, char *argv[]){
    int in_fd, out_fd;
    int rd_count, wt_count;
    char buffer[BUF_SIZE];

    if (argc != 3){
        fprintf(stderr, "Diocane!");
        exit(1);
    }

    in_fd = open(argv[1], O_RDONLY);
    if (in_fd < 0) exit(2);

    out_fd = creat(argv[2], OUTPUT_MODE);
    if(out_fd < 0) exit(3);

    while(TRUE){
        rd_count = read(in_fd, buffer, BUF_SIZE);
        if(rd_count <= 0) break;

        wt_count = write(out_fd, buffer, rd_count);
        if(wt_count <= 0) exit(4);
    }

    close(in_fd);
    close(out_fd);

    if(rd_count == 0) exit(0);
    else exit(5);
}
\end{lstlisting}

\subsection{Contenuto di una Directory}
\begin{lstlisting}[language=c]
void printFileInfo(const char* dirName, 
                   const char* fileName){
    struct stat fileInfo;
    char filePath[1024];
    snprintf(filePath, sizeof(filePath, "%s/%s"), 
        dirName, fileName);
    
    if (stat(filePath, &fileInfo)<0){
        // errore...
    }

    printf("%llu ", fileinfo.st_size);

    printf((S_ISDIR(fileInfo.st_mode)) ? "d" : "-");
    printf((fileInfo.st_mode & S_IRUSR) ? "r" : "-");
    printf((fileInfo.st_mode & S_IWUSR) ? "w" : "-");
    printf((fileInfo.st_mode & S_IXUSR) ? "x" : "-");
    printf((fileInfo.st_mode & S_IRGRP) ? "r" : "-");
    printf((fileInfo.st_mode & S_IWGRP) ? "w" : "-");
    printf((fileInfo.st_mode & S_IXGRP) ? "x" : "-");
    printf((fileInfo.st_mode & S_IROTH) ? "r" : "-");
    printf((fileInfo.st_mode & S_IWOTH) ? "w" : "-");
    printf((fileInfo.st_mode & S_IXOTH) ? "x" : "-");

    struct passwd *pw = getpwuid(fileInfo.st_uid);
    struct group  *gr = getgrgid(fileInfo.st_gid);
    printf(" %s %s", pw->pw_name, gr->gr_name);

    char dateStr[128];
    strftime(dateStr, sizeof(dateStr), "%b %d %H:%M", localtime(&fileInfo.st_mtime));
    printf(" %s", dateStr);

    printf(" %s\n", fileName);
}

int main(...) {
    DIR *dirp;
    struct dirent *dirent;

    if (argc != 2){
        ...
    }

    dirp = opendir(argv[1]);
    if (dirp == NULL){
        ...
    }

    while((dirent = readdir(dirp)) != NULL){
        printFileInfo(argv[1], dirent->d_name);
    }

    closedir(dirp);
    return 0;
}
\end{lstlisting}
    
\end{document}