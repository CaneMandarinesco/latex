\begin{definition}{Tipo di dato}
        
    Il \textbf{tipo di dato} definisce una collezione di oggetti, e che tipo di operazioni posso
    fare su questi.
\end{definition}

\begin{definition}{Struttura dati}

    La \textbf{struttura dati} e' l'organizzazione dei dati.
\end{definition}

    \subsection{Il Dizionario}
        \begin{lstlisting}[escapechar=@]
            tipo Dizionario: 
            dati:
                insieme @\(S\)@ di coppie @\((elem, chiave)\)@ 
            operazioni:
                insert(elem e, chiave k).
                delete(chiave k)
                search(chiave k) @\( \to \)@ elem

        \end{lstlisting}
    
    \subsection{La Pila}
    \begin{lstlisting}[escapechar=@]
        tipo Pila: 
        dati:
            sequenza @\(S\)@ di @\(n\)@ elementi. 
        operazioni:
            isEmpty().
            push(@\(elem\) \(e\)@)
                @aggiunge \(e\) come ultimo elemento di \(S\)@
            pop()
                @togli e restituisci l'ultimo elemento di \(S\)@
            top()
                @restituisci l'ultimo elemento di \(S\) senza toglierlo@

    \end{lstlisting}

    \newpage
    \subsection{La Coda}
    \begin{lstlisting}[escapechar=@]
        tipo Coda: 
        dati:
            sequenza @\(S\)@ di @\(n\)@ elementi. 
        operazioni:
            isEmpty().
            enqueue(@\(elem\) \(e\)@)
                @aggiunge \(e\) come ultimo elemento di \(S\)@
            dequeue()
                @togli e restituisci il primo elemento di \(S\)@
            first()
                @restituisci il primo elemento di \(S\) senza toglierlo@

    \end{lstlisting}

    \subsection{Tecniche di rappresentazione}
        Ci sono due tecniche per creare strutture dati:
        \begin{itemize}
            \item \textbf{Array} indicizzato.
            \item \textbf{Record} collegati fra puntatori.
        \end{itemize} 
        
        Un array indicizzato ha le seguenti proprieta:
        \begin{itemize}
            \item Prop. \textbf{Forte}: gli indici sono numeri consecutivi.
            \item Prop. \textbf{Debole}: non posso aggiungere nuove celle.
            \item \textbf{Contro}: dimensione fissa.
        \end{itemize}

        Se voglio implementare un dizionario con un array mi conviene usare un array \textbf{sovraddimensionato},
        che ha spazio per piu di \(n\) elementi. Cosi posso fare l'inserimento in \(O(1)\), 
        tuttavia \texttt{search} e \texttt{delete} richiedono \(O(n)\) \\

        Oppure posso usare un array \textbf{ordinato}, dove \texttt{search} costa \(O(\log n)\), mentre \texttt{insert} e \texttt{delete}
        costano \(O(n)\). \\

        La rappresentazione collegata, ovvero una lista, invece ha le seguenti proprieta:
        \begin{itemize}
            \item Prop. \textbf{Forte}: posso aggiungere o toglie record in tempo costante.
            \item Prop. \textbf{Debole}: i record non sono consecutivi.
            \item \textbf{Contro}: accesso sequenziale
        \end{itemize}

        Se per implementare il dizionario uso una lista non ordinata, ho che 
        \texttt{search} e \texttt{delete} impiegano \(O(n)\), mentre \texttt{insert} impiega tempo costante.

        La rappresentazione usando una lista ordinata invece e' sconveniente: mantenere la struttura ordinata
        aggiunge complessita', e tutte le operazioni richiedono \(O(n)\).
    
    \subsection{Alberi}
        Per rappresentare un'albero si puo' ricorrere ai \textbf{vettori posizionali}, dove ogni cella \(i\) contiene
        una coppia \((info, parent)\), ossia il suo contenuto \textbf{informativo} e l'indice del \textbf{padre}.

        Oppure si possono usare record collegati da puntatori, per esempio:
        \begin{itemize}
            \item Ogni record contiene due puntatori: figlio destro e sinistro. 
            \item Ogni record ha una lista di puntatori ai figli
            \item Ogni nodo, nel suo record, punta al primo figlio: \(v\). Nel caso ci siano altri figli, \(v\) punta al fratello successivo.
        \end{itemize}

    \subsection{Visite di alberi}
        \subsubsection{DFS}
        L'algoritmo DFS (Visita in Profondita'), parte dalla radice e procede visitando di figlio in figlio fino alla foglia.
        Una volta "visitata" la foglia si torna al padre e si controlla se ce ne siano altre, e si visitano. 
        Altrimenti si risale l'albero e si fa la stessa cosa con i nodi al livello superiore.
        
        \begin{lstlisting}[escapechar=@]
            algoritmo visitaDFS(@\(nodo\) \(r\)@)
                Pila S
                S.push(r)
                while (not S.isEmpty()) do
                    u = S.pop()
                    if(@\(u \neq null\)@) then
                        @visita il nodo u@
                        S.push(@figlio destro di \(u\)@)
                        S.push(@figlio sinistro di \(u\)@)
        \end{lstlisting}
        
        L'algoritmo inizia mettendo \(r\) nella radice. Appena
        entro nel loop, con \texttt{pop()} ottengo la radice, e vado a mettere
        figlio destro e sinistro nella Pila.
        All'iterazione successiva, con \texttt{pop()}, ottengo l'ultimo figlio messo con \texttt{push()},
        ossia il figlio sinistro di \(u\). 
        
        \subsubsection{DFS Ricorsivo}
        Esiste anche una versione ricorsiva, che funziona per alberi binari, ed e' molto interessante.
        \begin{lstlisting}[escapechar=@]
            algoritmo visitaDFSRicorsiva(@\(nodo\) \(r\)@)
                if(@\(r \neq null\)@) then
                    @visita il nodo \(r\)@
                    visitaDFSRicorsiva(@figlio sinistro di \(r\)@)
                    visitaDFSRicorsiva(@figlio destro di \(r\)@)
        \end{lstlisting}

        scambiando le tre righe dopo l'\texttt{if} posso:
        \begin{itemize}
            \item Visitare in preordine (DFS standard)
            \item Visitare simmetricamente: prima chiamo \texttt{visitaDFSRicorsiva} sul figlio sinistro, poi visito e poi faccio la chiamata sul figlio destro.
            \item Vistare in postordine: prima chiamo \texttt{visitaDFSRicorsiva} sul figlio sinistro, poi su quello destro e infine visito.
        \end{itemize}
        
        \subsubsection{BFS}
        Partendo dalla radice, prima di scendere al livello sotto, visito tutti i nodi. Si implementa con 
        la Coda.

        \begin{lstlisting}[escapechar=@]
            algoritmo visitaBFS(@\(nodo\) \(r\)@)
                Coda C
                C.enqueue(r)
                while (not C.isEmpty()) do
                    u = C.dequeue()
                    if(@\(u \neq null\)@) then
                        @visita il nodo \(u\)@
                        C.enqueue(@figlio sintro di \(u\)@)
                        C.enqueue(@figlio destro di \(u\)@)
        \end{lstlisting}

        La logica di funzionamento e' simile al DFS.
