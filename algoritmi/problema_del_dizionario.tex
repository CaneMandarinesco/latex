\subsection{Implementare il dizionario}
Per l'implementazione abbiamo due possibilita:
\begin{itemize}
    \item alberi BST (binario di ricerca): \(O(\text{altezza albero})\)
    \item albero AVL: \(O(log n)\)
\end{itemize}

\subsubsection{Albero BST}
    Ogni nodo ha \(elem(v)\) e \(chiave(v)\), la chiave si trova in un dominio totalmente ordinato
    e vale che:
    \begin{itemize}
        \item nel sottoalbero sinitro di \(v\) le chavi sono \(\leq chiave(v)\)
        \item nel sottoalbero destro di \(v\) le chavi sono \(\ge chiave(v)\)
    \end{itemize}

    \begin{definition}{Proprieta del BST}

        Nella foglia piu a sinistra si trova il minimo. Nel nodo piu a destra si trova il massimo.
    \end{definition}

    \begin{definition}{Visita simmetrica del BST}

        Usando la visita \textbf{simmetrica} su un BST, si guardano i nodi in ordine crescente. 
    \end{definition}
    
    \paragraph{Dimostrazione}
        Per induzione, con un albero di altezza \(h=1\) e' ovvio. \\
        Ora per un \(h\) generico, i figli della radice sono alberi di altezza $\leq h-1$.
        Per ipotesi induttiva questo e' visitato correttamente, allora l'albero di altezza $h$ e' visitato correttamente.

    \begin{definition}{predecessore e successore}

        Il predecessore di un nodo $u$ e' il nodo $v$ con chiave massima $\leq chiave(u)$ \\
        Il successore di un nodo $u$ e' il nodo $v$ avente minima chiave $\geq chiave(u)$
    \end{definition}

    \paragraph{Implementare l'operazione delete}
    Il predecessore di un nodo:
    \begin{itemize}
        \item si trova nel sottoalbero sinistro
        \item se non c'e' un sottoalbero sinistro il predecessore e' un suo antenato
    \end{itemize}

    \begin{lstlisting}[mathescape=true]
        algoritmo pred(nodo u) $\to nodo$
            if($u$ ha figlio sinistro $sin(u)$) then
                return max($sin(u)$)
            while ($parent(u) \neq null$ e $u$ e' figlio sinistro di suo padre) do
                u = parent(u)
            return $parent(u)$
    \end{lstlisting}

    Il successore di un nodo $u$ invece e' il minimo nel sottoalbero destro di $u$.

    Per implementare l'operazione \texttt{delete} devo considerare 3 casi:
    \begin{itemize}
        \item $u$ e' una foglia
        \item $u$ ha un solo figlio
        \item $u$ ha solo due figli
    \end{itemize}

    Nell'ultimo caso devo sostituire $u$ con il suo predecessore o il suo successore $v$.

    \paragraph{Considerazioni sul costo delle operazioni}
    Tutte hanno costo $O(h)$, ma al caso peggiore si ha $O(n)$ in caso di
    alberi profondi e sbilanciati.
    Un albero e' bilanciato se $h=O(log n)$

\subsubsection{Albero AVL}
    E' un BST, ma andiamo a considerare il fattore di bilanciamento $\beta(v)$, il cui modulo,  per ogni nodo, deve essere sempre $\leq 1$. 
    Questo valore si ottiene facendo la differenza delle altezze del sottoalbero sinistro e destro del nodo $v$.

    \paragraph{Altezza di un AVL}
    Tra tutti gli AVL, quello piu sbilanciato e' l'albero di Fibonacci: e' l'albero AVL di altezza $h$ con
    il numero minimo di nodi $n_h$.

    \begin{definition}
        L'altezza di un'albero AVL e' $h=O(log n)$
    \end{definition}
    \paragraph{Dimostrazione}
        Sappiamo che $n_h = F_{h+3}-1= \Theta(\phi^n)$.
        Quindi $h = \Theta(log n_h) = O(log n)$
    \paragraph{Implementazione del dizionario}
        Ogni volta che elimino o inserisco un nodo devo mantenere il bilanciamento grazie alle rotazioni, ho 4 casi in base a quale sottoalbero mi sbilancia $v$:
        \begin{itemize}
            \item SS: e' il sottoalbero sinistro del figlio sinistro di $v$
            \item DD: e' il sottoalbero destro del figlio destro di $v$
            \item SD: e' il sottoalbero destro del figlio sinistro
            \item DS: e' il sottoalbero sinistro del figlio destro
        \end{itemize}

        I casi sono simmetrici due a due. \\
        Il caso SS si risolve con una rotazione a destra con perno $v$.
        Il caso SD si risolve con una rotazione a sinistra sul figlio sinistro del nodo critico $z$ e l'altra verso destra sul nodo $v$

        L'operazione di rotazione richiede tempo costante.

        Di conseguenza posso implementare: \texttt{insert} e \texttt{delete} in tempo $O(logn)$ \\

        Per implementare l'operazione \texttt{insert} una volta inserito il valore, devo ricalcolare i fattori di bilanciamento,
        e una voltra trovato il nodo critico eseguo un bilanciamento tramite rotazione: \textbf{ne basta uno!} \\

        Per implementare l'operazione \texttt{delete} una volta eliminata la chiave, effettuo il bilanciamento sul nodo critico.
        Se il bilanciamento cambia l'altezza dell'albero devo controllare se al nodo superiore si e' sbilanciato qualcosa, e cosi via 
        fino alla radice.