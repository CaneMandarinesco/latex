\section*{Esercizio 1}

\subsection{Idea}
Per risolvere il problema si fa una visita BFS, sul grafo $G$ delle configurazioni, a partire dalla configurazione iniziale.
Due nodi $v,u$ sono collegati nel grafo, se e solo se e' possibile passare da una configurazione all'altra in una mossa.
Ogni nodo del grafo $G$ puo' essere rappresentato come una lista di valori: [Coppia p1, Coppia p2],
ossia le posizioni che occupa la pedina (dove $p1=p2$ se questa e' in piedi). 
La visita BFS produce l'albero $T$, che e' radicato nella configurazione iniziale.

Conoscendo $p1,p2$ e $M$ posso scoprire tutte le mosse possibili a partire dalla posizione $[p1,p2]$.
La procedura \emph{muoviPedina(\textbf{Nodo} v, \textbf{Mossa} m) $\to$ \textbf{Nodo}} dato in input la posizione della pedina
per ognuna delle 4 mosse possibili: $\{up,down,left,right\}$, ritorna la posizione della pedina dopo aver effettuato 
quella la mossa. \\

L'algoritmo termina quando la vista BFS trova la configurazione finale.\\

Per costruire la sequenza di mosse posso risalire l'albero $T$ a partire dalla configurazione finale. Il metodo
$convertiInMossa$(\textbf{Nodo} from, \textbf{Nodo} to) $\to$ \textbf{Mossa} $\in \{u,d,l,r\}$ ritorna il nome 
della mossa che mi porta da $from$ a $to$.
Ogni nodo della \emph{ListaCollegata L} ha due puntatori: al nodo precedente e al nodo successore, quindi posso
leggere la lista di mosse in due modi versi: dalla prima all'ultima e dall'ultima alla prima.
Il metodo $L.insertPrec$(\textbf{Nodo} v) imposta il puntatore al precedente dell'ultimo nodo aggiunto in $L$,
alla prossima chiamta di $insertPrec$ il riferimento al nodo precedente sara' impostato sul nodo di $v$.


\subsection{Codice}
\begin{algorithm}[H]
\SetAlgoLined
\KwIn{Coppia $c_1 = (i,j)$, Coppia $c_2 = (s,t)$, Matrice $M$}
\KwOut{Sequenza di mosse ottimale per risolvere il livello, o $False$ se cio' non e' possibile}

Coda F \;
Matrice $M'$ di grandezza $n \times n$\;
Albero T \;

\textbf{Nodo} u = $[c_1, c_1]$ \tcp { Configurazione iniziale }
$dist(u) = 0$ \;
\textbf{Nodo} h = \textbf{null} \tcp { Configurazione finale }

\tcp {Marca u e mettilo nell'albero}
F.enqueue(u) \;
M[$c_1$].insert($c_1$) \;   
Rendi il nodo $u$ radice in $T$ \;

risolto $\gets$ \textbf{false} \;
\While{\textbf{not} $F.empty()$}{
    \textbf{Nodo} $v \gets F.dequeue()$ \;
    \ForEach{mossa $\in \{u,d,l,r\}$}{
        \tcp {Ottieni la nuova posizione dopo aver effettuato la mossa}
        Coppie $p1, p2 \gets muoviPedina(v, mossa)$ \;
        \uIf{$p1$ e $p2$ sono dentro la mappa e $M[p1], M[p2] \neq -1$}{

            \tcp{Controlla se la mossa e' marcata}
            \uIf {$p2 \notin M'[p1]$}{
                \textbf{Nodo} $k = [p1, p2]$ \;
                $dist(k) = dist(v)+1$ \;
                F.enqueue($k$) \;
                Rendi $v$ padre del nodo $k$ in $T$ \;
                \tcp{Marca la posizione}
                M[p1].insert(p2) \;
            } 

            \tcp{Controlla se sono in piedi sulla casella finale}
            \uIf {$p1$ == $p2$ == $ c_2$}{
                risolto $\gets$ \textbf{true} \;
                h = k \;
                \textbf{break} \;
            }
        }
    }
}

\uIf {\textbf{not risolto}} {
    \Return risolto \;
}

\tcp{risali l'albero per ottenere la sequenza di mosse da ritornare}
ListaCollegata L \;
\While {$h \neq u$ } {
    \textbf{Nodo} $h' \gets padre(h)$ \;
    \textbf{Mossa} m = convertiInMossa(h',h) \;
    L.insertPrec($m$) \;
    $h=h'$
}

\Return L \;


\caption{Bloxorz Solver}
\end{algorithm}

\subsection{Correttezza}
L'algoritmo si basa su due proprieta' della vista BFS.

\paragraph{Prop. 1}
Nella visita BFS si ha che $l(v) = d_{sv}$: ossia la profondita (o livello) in cui si trova $v$ 
nell'albero generato da BFS e' uguale al cammino minimo che porta dalla radice a $v$ nel grafo. \\

\paragraph{Prop. 2}
Se durante la vista BFS la coda contiene i vertici: $\{v_1, \dots, v_p\}$, allora $l(v_i) \leq l(v_{i+1})$
per $1 \leq i \leq p-1$. \\

Per cui, nel momento in cui trovo la mossa che mi porta alla configurazione finale $h$, questo nodo ha $l(h) = d_{sh} = l_h$.
Ora caso mai trovassi un'altra sequenza di mosse che mi porta ad $h$, con lunghezza $l_h'$, sono sicuro che $l_h' \geq l_h$,e 
che quindi $d_{sh}' \geq d_{sh}$.

\subsection{Complessita'}
Per semplicita' di analisi anlizziamo il grafo $G'$ costruito sulla matrice M di grandezza $n \times n$, dove ogni casella e' praticabile, e ognuna di queste 
ha $4$ posizioni possibili, e per ogni posizione (ossia nodo un nodo) ci sono 4 archi verso altri 4 nodi (un'arco per 
mossa). \\

Quindi nel grafo ci sono $O(4 \cdot n^2)$ nodi e $O(4 \cdot 4 \cdot n^2)$ archi. L'algoritmo al caso peggiore
scopre tutti i nodi del grafo, e per farlo pratica tutti gli archi che portano a configurazioni valide, quindi
l'algoritmo esegue in tempo $O(n^2)$

La complessita' spaziale e' anch'essa $O(n^2)$, l'algoritmo prevede l'uso della matrice $M'$ di grandezza $n \times n \times 5$.
La matrice contiene liste collegate che al loro interno hanno al massimo 5 elementi (uno per posizione possibile su quella casella).

Invece la lista collegata, riprendendo quanto detto prima, ha al massimo $O(n^2)$ mosse in $G'$ che risolvono il problema. 


%%% NON TOCCARE QUESTA PARTE
\newpage
\begin{tcolorbox}[
    colback=white,           % sfondo
    colframe=black,          % colore bordo
    coltitle=black,          % colore del testo
    colbacktitle=gray!50,    % sfondo titolo
    boxrule=1pt,             % spessore del bordo
    title=\textbf{Commenti Esercizio}
    ]
    
\vspace{45\baselineskip}

\end{tcolorbox}
\newpage
