\section*{Esercizio 1}

\subsection{Idea}
Faccio una vista BFS sulle mosse possibili.

\subsection{Codice}
\begin{algorithm}[H]
\SetAlgoLined
\KwIn{Coppia $c_1 = (i,j)$, Coppia $c_2 = (s,t)$, Matrice $M$}
\KwOut{Sequenza di mosse ottimale per risolvere il livello, o $False$ se cio' non e' possibile}

Coda F \;
Matrice $M'$ di grandezza $n \times n$\;
Albero T \;
\tcp { --- }
u = $[c_1, c_1, 0]$ \;
h = \textbf{null} \;
F.enqueue(u) \;
M[i,j].insert($c_1$) \;    $v \gets F.dequeue()$ \;
Rendi il nodo $u$ radice in $T$ \;
risolto $\gets$ \textbf{false} \;
\While{\textbf{not} $F.empty()$}{
    $v \gets F.dequeue()$ \;
    \ForEach{mossa $\in \{u,d,l,r\}$}{
        \tcp {Ottieni la nuova posizione dopo aver effettuato la mossa}
        $p1, p2 \gets muoviPedina(v, mossa)$ \;
        \uIf{$p1$ e $p2$ sono dentro la mappa e $M[p1], M[p2] \neq -1$}{

            \tcp{Controlla se la mossa e' marcata}
            \uIf {$p2 \notin M'[p1]$}{
                $k = [p1, p2, dist(v)+1]$ \;
                F.enqueue($k$) \;
                Rendi $v$ padre del nodo $k$ in $T$ \;
                \tcp{Marca la posizione}
                M[p1].insert(p2) \;
            } 

            \tcp{Controlla se sono in piedi sulla casella finale}
            \uIf {$p1$ == $p2$ == $ c_2$}{
                risolto $\gets$ \textbf{true} \;
                h = k \;
                \textbf{break} \;
            }
        }
    }
}

\uIf {\textbf{not risolto}} {
    \Return risolto \;
}

\tcp{risali l'albero per ottenere la sequenza di mosse da ritornare}
ListaCollegata L \;
\While {$h \neq $ } {
    $h \gets padre(h)$ \;
    L.insert(h) \;
}

\Return L \;


\caption{Bloxorz Solver}
\end{algorithm}

\subsection{Correttezza}

\subsection{Complessita'}

%%% NON TOCCARE QUESTA PARTE
\newpage
\begin{tcolorbox}[
    colback=white,           % sfondo
    colframe=black,          % colore bordo
    coltitle=black,          % colore del testo
    colbacktitle=gray!50,    % sfondo titolo
    boxrule=1pt,             % spessore del bordo
    title=\textbf{Commenti Esercizio}
    ]
    
\vspace{45\baselineskip}

\end{tcolorbox}
\newpage
