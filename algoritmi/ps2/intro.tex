\section*{Introduzione}
\LaTeX\ è un sistema di composizione tipografica progettato per creare documenti di alta qualità, particolarmente adatto per testi tecnici e scientifici.
A differenza degli strumenti che ti costringono a preoccuparti di ogni singolo dettaglio di formattazione, \LaTeX\ ti lascia libero di concentrarti sul contenuto, occupandosi automaticamente di impaginazione, numerazione, e referenze.

\paragraph{Perché usare \LaTeX?}
\begin{description}
    \item[Tipografia impeccabile:] \LaTeX\ produce testi e formule matematiche che fanno impallidire qualsiasi altro strumento. E non c'è bisogno di perdere tempo a cercare ``il giusto font" o a sistemare i margini che non si allineano mai come vorresti.
    
    \item[Automazione:] Indici, bibliografie, numerazioni automatiche: nessun bisogno di aggiornare manualmente numeri di pagina o titoli. \LaTeX\ lo fa per te, sempre!
    
    \item[Coerenza:] Un documento \LaTeX\ non cambierà improvvisamente formato a causa di un'interferenza accidentale tra immagini e testo. Una volta che il layout è impostato, resterà sempre lo stesso. Garantito!
    
\end{description}
 
\paragraph{E le alternative?}
C'è chi, ancora oggi, si affida a strumenti che costringono a cliccare senza sosta su menu e pulsanti, a spostare immagini a mano, a premere 15 volte il tasto ``spazio" per allineare i paragrafi e pregando che il layout non si sfasci all'ultimo momento, e a piangere ogni volta che qualcosa non si allinea come dovrebbe.
Ma il vero divertimento arriva quando il layout cambia misteriosamente dopo aver aggiunto un singolo paragrafo.\footnote{Ne sapete qualcosa?}
Chi usa \LaTeX\ risponde: ``Noi non abbiamo tempo per questo."

\paragraph{Obiettivo di questa guida}
Questa breve guida ti fornirà gli strumenti di base per iniziare a usare \LaTeX. Imparerai come creare liste, tabelle, scrivere formule matematiche, inserire immagini e gestire referenze senza dover mai più mettere mano a ``quel" programma.
Non è difficile come sembra, e una volta provato, non tornerai più indietro.

\bigskip
\begin{tcolorbox}[
    colback=orange!10,
    colframe=orange!70!black,
    title=\textbf{Importante!},
    ]
    Una volta finita questa guida, commenta la riga 66 del file \texttt{main.tex}.
    I commenti in \LaTeX\ si fanno antecedendo il simbolo \texttt{\%} alla riga che vuoi commentare.
    Questo serve per far sparire questa guida una volta che dovrete consegnare il pdf al docente.
\end{tcolorbox}

\subsection*{Liste}

\subsubsection*{Lista non numerate (\texttt{itemize})}
\begin{itemize}
    \item Primo elemento
    \item Secondo elemento
    \item Terzo elemento
\end{itemize}

\subsubsection*{Lista numerate (\texttt{enumerate})}
\begin{enumerate}
    \item Primo elemento
    \item Secondo elemento
    \item Terzo elemento
\end{enumerate}

\subsubsection*{Lista descrittive (\texttt{description})}
\begin{description}
    \item[Primo] Descrizione del primo elemento.
    \item[Secondo] Descrizione del secondo elemento.
    \item[Terzo] Descrizione del terzo elemento.
\end{description}

\subsection*{Font particolari}
\begin{description}
    \item[Grassetto:] \textbf{testo in grassetto}
    \item[Corsivo:] \emph{testo in corsivo} o \textit{testo in corsivo}
    \item[Monospaziato:] \texttt{testo monospaziato}
\end{description}

\subsection*{Formule matematiche}

\subsubsection*{Formule in linea}
Il teorema di \emph{Pitagora} è espresso come $a^2 + b^2 = c^2$.

\subsubsection*{Formule in blocco}
La seguente formula è in blocco
\[
a^2 + b^2 = c^2
\]

\subsubsection*{Formule numerate}
\begin{equation}\label{eq:energy}
E = mc^2
\end{equation}

\subsubsection*{Formule su più righe}
\begin{align*}
    Var(aX) &= \mathbb{E}[(aX - \mathbb{E}[aX])^2]\\
    &= \mathbb{E}[\big(a(X - \mathbb{E}[X])\big)^2]\\
    &= \mathbb{E}[a^2 \cdot (X - \mathbb{E}[X])^2]\\
    &= a^2\mathbb{E}[(X - \mathbb{E}[X])^2] = a^2 \cdot Var(X)
\end{align*}

\subsection*{Tabelle}
\begin{table}[H]
\centering
\begin{tabular}{|c||c|c|}
\hline
Colonna 1 & Colonna 2 & Colonna 3 \\ \hline\hline
A         & B         & C         \\ \hline
D         & E         & F         \\ \hline
\end{tabular}
\caption{An example table.}
\label{tab:table_example}
\end{table}

\subsection*{Immagini}

\subsubsection*{Singola immagine}
\begin{figure}[H] % "H" forza l'immagine ad apparire dove la metti. è meglio mettere "h" oppure "ht" , perchè sarà latex a posizionarla dove è meglio.
    \centering
    \includegraphics[width=0.25\linewidth]{images/frog.jpg}
    \caption{Descrizione immagine}
    \label{fig:label-immagine.}
\end{figure}

\subsubsection*{Due immagini affiancate}
\begin{figure}[H]
    \centering
    \begin{subfigure}[b]{0.45\textwidth}
        \centering
        \includegraphics[width=.4\textwidth]{images/frog.jpg}
        \caption{Prima immagine}
        \label{fig:prima}
    \end{subfigure}
    \hfill
    \begin{subfigure}[b]{0.45\textwidth}
        \centering
        \includegraphics[width=.4\textwidth]{images/frog.jpg}
        \caption{Seconda immagine}
        \label{fig:seconda}
    \end{subfigure}
    \caption{Due immagini affiancate}
    \label{fig:due_immagini}
\end{figure}

\subsection*{Pseudocodice}

\begin{algorithm}[H]
\SetAlgoLined 
\KwIn{Lista di numeri $A$, un valore $x$}
\KwOut{Indice dell'elemento trovato oppure $-1$ se non esiste}

\tcp{Esempio di un commento}
$n \gets \text{lunghezza di } A$\;
$found \gets \textbf{false}$\;
$result \gets -1$\;

\tcp{Esempio di ciclo for}
\For{$i \gets 1$ \textbf{to} $n$}{
    \tcp{Esempio if}
    \If{$A[i] == x$}{
        $found \gets \textbf{true}$\;
        $result \gets i$\;
        \textbf{break}\;
    }
}

\tcp{Esempio if-else}
\uIf{\textbf{not} $found$}{
    \Repeat{una condizione speciale è soddisfatta}{
        $result \gets result - 1$\;
    }
} \Else {
    \Return{$result$}\;
}

\tcp{Esempio di ciclo Foreach}
\ForEach{elemento $y$ \textbf{in} $A$}{
    \If{$y > x$}{
        \textbf{continue}\; % Salta al prossimo elemento
    }
    $result \gets y$\;
}

\tcp{Esempio di ciclo While}
\While{$found == \textbf{false}$ \textbf{and} $result < n$}{
    $result \gets result + 1$\;
}

\Return{$result$}\;

\caption{Esempio di Algoritmo con tutte le keyword principali}
\end{algorithm}



\newpage
