
\begin{lstlisting}[mathescape=True]
tipo CodaPriorita:
dati:
    un insieme di $S$ di $n$ elementi con chiavi prese da un campo ordinato
operazioni
    findMin() $\to elem$
    insert($elem e$, $chiave k$)
        aggiungi a $S$ un nuovo elemento $e$ con chiave $k$
    delete($elem e$)
        cancella da $S$ l'elemento $e$ 
            nota: ho il riferimento dell'oggetto!

    deleteMin()
        cancella da $S$ l'elemento con chiave minima
    
    increaseKey($elem e$, $chiave d$)
        incrementa della quantita $d$ la chiave $e$
    decreaseKey($elem e$, $chiave d$)
        decrementa della quantita $d$ la chiave $e$
    merge(CodaPriorita $c_1$, CodaPriorita $c_2$)
        restituisce una nuova coda: $c_3 = c_1 \cup c_2$
\end{lstlisting}

Ci sono varie implementazioni possibili: 
\begin{itemize}
    \item Con array non ordinato:  \texttt{insert} e \texttt{delete} eseguono in $O(1)$, \texttt{FindMin} e \texttt{DeleteMin} eseguono in $\Theta(n)$
    \item Con array ordinato: \texttt{insert} e \texttt{delete} eseguono in $O(n)$. Per la proprieta dell'ordinamento so trovare e eliminare il minimo in $O(1)$
    \item Lista non ordinata: tempi uguali all'array non ordinato
    \item Lista ordinata: tempi uguali all'array ordinato, ma riesco ad eliminare in $O(1)$
\end{itemize}

\subsection{d-Heap}
    \begin{lstlisting}[mathescape=true]
        procedura muoviAlto(v)
            while($v \neq radice(T)$ and $chiave(v) < chiave(padre(v))$) do
                svambia di posto $v$ e $padre(v)$ in $T$
            
        procedura muoviBasso(v)
            repeat
                sia $u$ il figlio di $v$ con chiave mininima
                if($v$ non ha figlio o $chiave(v) \leq chiave(u)$) break
                scambia $v$ e $u$ in $T$
    \end{lstlisting}

    L'operazione \texttt{insert} inserisce una nuova foglia e su questa ripristino la proprita di ordinamento. \\

    L'operazione \texttt{delete} scambia  il nodo $v$ con una foglia qualunque $u$, infine devo ripristinare
    la proprieta di ordinamento usando \texttt{muoviAlto} o \texttt{muoviBasso} ($O(d log_d n)$).\\

    L'operazione \texttt{decreaseKey} viene eseguita in tempo $O(log_d n)$ dato che devo eseguire solo muoviAlto
    al caso peggiore, invece \texttt{increaseKey} al caso peggiore esegue in $O(d log_d n)$.

    L'operazione \texttt{merge} puo' essere implementata in due modi: 
    \begin{itemize}
        \item Creo una nuova coda da 0
        \item Ad una delle due aggiungo l'altra. Con $k=min\{ |c_1|, |c_2|\}$, e quindi l'operazione ha costo $O(k log n)$
        Quand'e' che $k log n = o(n)$? risolvendo l'ugualgianza se: $k = o(n/logn)$, in questo caso conviene applicare questa tecnica.
    \end{itemize}

\subsection{Heap Binomiali}
    Nota: l'ordinamento non e' mantenuto orizzontalmente. \\

    \begin{lstlisting}
        procedura ristruttura()
            i=0
            while( esistono ancora due $B_i$) do
                fondi i due $B_i$ per formare un 
                albero $B_{i+1}$
                poni la radice del'albero con chiave
                minore, come padre dell'altro
                i+=1
    \end{lstlisting}

    \texttt{insert} si implementa aggiungendo un albero $B_0$, e applico \texttt{ristruttura},
    fino a che non si risolve. \\

    \texttt{deleteMin} trova la radice con chiave minima. \\

    \texttt{decreaseKey} aggiorna il valore della chiave e ripristina l'ordinamento. \\

    \texttt{delete} richiama \texttt{decreaseKey} fino a che il nodo non sale alla radice, poi viene eliminato
    in tempo costante.

    \texttt{merge} unisce $c1$ e $c2$ in un unico albero binomiale: con l'operazione \texttt{ristruttura} comincio
    a ripristinare la proprieta di unicita.

\subsection{Heap di Fibonacci}
