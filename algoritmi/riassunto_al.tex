\documentclass{article}

\usepackage[italian]{babel}
\usepackage{listings}
\usepackage{amsmath}

\title{Riassuntazzo Algoritmi e Strutture Dati Mod. 1}
\author{Davide Luci}
\date{2024}

\newtheorem{definition}{Definition}[section]

\begin{document}

    \maketitle
    \pagenumbering{gobble}
    \tableofcontents
    \newpage

    \pagenumbering{arabic}

    \setlength{\parindent}{0pt}
    \section{ Prima di leggere...}
    Questo documento non e' esaustivo al fine di comprendere al pieno la materia trattata.
    Le definizioni sono scarne, e spesso non esaustive, perche' ho preferito solo appuntare i concetti chiavi
    da ripassare in modo da trovare una giusta via di mezzo tra studiare e riassumere. Si puo' dire che sia compito del 
    lettore, quello di arricchire il contenuto di questo documento.

    \newpage

    \section*{ Introduzione}
    Diocane

    \newpage
    \section{ Notazione asintotica e Modelli di calcolo}

    \newpage
    \section{ Sorting }

    \newpage
    \section{ Capitolo 3: Strutture dati elementari }
        \begin{definition}{Tipo di dato}
        
    Il \textbf{tipo di dato} definisce una collezione di oggetti, e che tipo di operazioni posso
    fare su questi.
\end{definition}

\begin{definition}{Struttura dati}

    La \textbf{struttura dati} e' l'organizzazione dei dati.
\end{definition}

    \subsection{Il Dizionario}
        \begin{lstlisting}[escapechar=@]
            tipo Dizionario: 
            dati:
                insieme @\(S\)@ di coppie @\((elem, chiave)\)@ 
            operazioni:
                insert(elem e, chiave k).
                delete(chiave k)
                search(chiave k) @\( \to \)@ elem

        \end{lstlisting}
    
    \subsection{La Pila}
    \begin{lstlisting}[escapechar=@]
        tipo Pila: 
        dati:
            sequenza @\(S\)@ di @\(n\)@ elementi. 
        operazioni:
            isEmpty().
            push(@\(elem\) \(e\)@)
                @aggiunge \(e\) come ultimo elemento di \(S\)@
            pop()
                @togli e restituisci l'ultimo elemento di \(S\)@
            top()
                @restituisci l'ultimo elemento di \(S\) senza toglierlo@

    \end{lstlisting}

    \newpage
    \subsection{La Coda}
    \begin{lstlisting}[escapechar=@]
        tipo Coda: 
        dati:
            sequenza @\(S\)@ di @\(n\)@ elementi. 
        operazioni:
            isEmpty().
            enqueue(@\(elem\) \(e\)@)
                @aggiunge \(e\) come ultimo elemento di \(S\)@
            dequeue()
                @togli e restituisci il primo elemento di \(S\)@
            first()
                @restituisci il primo elemento di \(S\) senza toglierlo@

    \end{lstlisting}

    \subsection{Tecniche di rappresentazione}
        Ci sono due tecniche per creare strutture dati:
        \begin{itemize}
            \item \textbf{Array} indicizzato.
            \item \textbf{Record} collegati fra puntatori.
        \end{itemize} 
        
        Un array indicizzato ha le seguenti proprieta:
        \begin{itemize}
            \item Prop. \textbf{Forte}: gli indici sono numeri consecutivi.
            \item Prop. \textbf{Debole}: non posso aggiungere nuove celle.
            \item \textbf{Contro}: dimensione fissa.
        \end{itemize}

        Se voglio implementare un dizionario con un array mi conviene usare un array \textbf{sovraddimensionato},
        che ha spazio per piu di \(n\) elementi. Cosi posso fare l'inserimento in \(O(1)\), 
        tuttavia \texttt{search} e \texttt{delete} richiedono \(O(n)\) \\

        Oppure posso usare un array \textbf{ordinato}, dove \texttt{search} costa \(O(\log n)\), mentre \texttt{insert} e \texttt{delete}
        costano \(O(n)\). \\

        La rappresentazione collegata, ovvero una lista, invece ha le seguenti proprieta:
        \begin{itemize}
            \item Prop. \textbf{Forte}: posso aggiungere o toglie record in tempo costante.
            \item Prop. \textbf{Debole}: i record non sono consecutivi.
            \item \textbf{Contro}: accesso sequenziale
        \end{itemize}

        Se per implementare il dizionario uso una lista non ordinata, ho che 
        \texttt{search} e \texttt{delete} impiegano \(O(n)\), mentre \texttt{insert} impiega tempo costante.

        La rappresentazione usando una lista ordinata invece e' sconveniente: mantenere la struttura ordinata
        aggiunge complessita', e tutte le operazioni richiedono \(O(n)\).
    
    \subsection{Alberi}
        Per rappresentare un'albero si puo' ricorrere ai \textbf{vettori posizionali}, dove ogni cella \(i\) contiene
        una coppia \((info, parent)\), ossia il suo contenuto \textbf{informativo} e l'indice del \textbf{padre}.

        Oppure si possono usare record collegati da puntatori, per esempio:
        \begin{itemize}
            \item Ogni record contiene due puntatori: figlio destro e sinistro. 
            \item Ogni record ha una lista di puntatori ai figli
            \item Ogni nodo, nel suo record, punta al primo figlio: \(v\). Nel caso ci siano altri figli, \(v\) punta al fratello successivo.
        \end{itemize}

    \subsection{Visite di alberi}
        \subsubsection{DFS}
        L'algoritmo DFS (Visita in Profondita'), parte dalla radice e procede visitando di figlio in figlio fino alla foglia.
        Una volta "visitata" la foglia si torna al padre e si controlla se ce ne siano altre, e si visitano. 
        Altrimenti si risale l'albero e si fa la stessa cosa con i nodi al livello superiore.
        
        \begin{lstlisting}[escapechar=@]
            algoritmo visitaDFS(@\(nodo\) \(r\)@)
                Pila S
                S.push(r)
                while (not S.isEmpty()) do
                    u = S.pop()
                    if(@\(u \neq null\)@) then
                        @visita il nodo u@
                        S.push(@figlio destro di \(u\)@)
                        S.push(@figlio sinistro di \(u\)@)
        \end{lstlisting}
        
        L'algoritmo inizia mettendo \(r\) nella radice. Appena
        entro nel loop, con \texttt{pop()} ottengo la radice, e vado a mettere
        figlio destro e sinistro nella Pila.
        All'iterazione successiva, con \texttt{pop()}, ottengo l'ultimo figlio messo con \texttt{push()},
        ossia il figlio sinistro di \(u\). 
        
        \subsubsection{DFS Ricorsivo}
        Esiste anche una versione ricorsiva, che funziona per alberi binari, ed e' molto interessante.
        \begin{lstlisting}[escapechar=@]
            algoritmo visitaDFSRicorsiva(@\(nodo\) \(r\)@)
                if(@\(r \neq null\)@) then
                    @visita il nodo \(r\)@
                    visitaDFSRicorsiva(@figlio sinistro di \(r\)@)
                    visitaDFSRicorsiva(@figlio destro di \(r\)@)
        \end{lstlisting}

        scambiando le tre righe dopo l'\texttt{if} posso:
        \begin{itemize}
            \item Visitare in preordine (DFS standard)
            \item Visitare simmetricamente: prima chiamo \texttt{visitaDFSRicorsiva} sul figlio sinistro, poi visito e poi faccio la chiamata sul figlio destro.
            \item Vistare in postordine: prima chiamo \texttt{visitaDFSRicorsiva} sul figlio sinistro, poi su quello destro e infine visito.
        \end{itemize}
        
        \subsubsection{BFS}
        Partendo dalla radice, prima di scendere al livello sotto, visito tutti i nodi. Si implementa con 
        la Coda.

        \begin{lstlisting}[escapechar=@]
            algoritmo visitaBFS(@\(nodo\) \(r\)@)
                Coda C
                C.enqueue(r)
                while (not C.isEmpty()) do
                    u = C.dequeue()
                    if(@\(u \neq null\)@) then
                        @visita il nodo \(u\)@
                        C.enqueue(@figlio sintro di \(u\)@)
                        C.enqueue(@figlio destro di \(u\)@)
        \end{lstlisting}

        La logica di funzionamento e' simile al DFS.

        \newpage
    \section {Problema del Dizionario}
        \subsection{Implementare il dizionario}
Per l'implementazione abbiamo due possibilita:
\begin{itemize}
    \item alberi BST (binario di ricerca): \(O(\text{altezza albero})\)
    \item albero AVL: \(O(log n)\)
\end{itemize}

\subsubsection{Albero BST}
    Ogni nodo ha \(elem(v)\) e \(chiave(v)\), la chiave si trova in un dominio totalmente ordinato
    e vale che:
    \begin{itemize}
        \item nel sottoalbero sinitro di \(v\) le chavi sono \(\leq chiave(v)\)
        \item nel sottoalbero destro di \(v\) le chavi sono \(\ge chiave(v)\)
    \end{itemize}

    \begin{definition}{Proprieta del BST}

        Nella foglia piu a sinistra si trova il minimo. Nel nodo piu a destra si trova il massimo.
    \end{definition}

    \begin{definition}{Visita simmetrica del BST}

        Usando la visita \textbf{simmetrica} su un BST, si guardano i nodi in ordine crescente. 
    \end{definition}
    
    \paragraph{Dimostrazione}
        Per induzione, con un albero di altezza \(h=1\) e' ovvio. \\
        Ora per un \(h\) generico, i figli della radice sono alberi di altezza $\leq h-1$.
        Per ipotesi induttiva questo e' visitato correttamente, allora l'albero di altezza $h$ e' visitato correttamente.

    \begin{definition}{predecessore e successore}

        Il predecessore di un nodo $u$ e' il nodo $v$ con chiave massima $\leq chiave(u)$ \\
        Il successore di un nodo $u$ e' il nodo $v$ avente minima chiave $\geq chiave(u)$
    \end{definition}

    \paragraph{Implementare l'operazione delete}
    Il predecessore di un nodo:
    \begin{itemize}
        \item si trova nel sottoalbero sinistro
        \item se non c'e' un sottoalbero sinistro il predecessore e' un suo antenato
    \end{itemize}

    \begin{lstlisting}[mathescape=true]
        algoritmo pred(nodo u) $\to nodo$
            if($u$ ha figlio sinistro $sin(u)$) then
                return max($sin(u)$)
            while ($parent(u) \neq null$ e $u$ e' figlio sinistro di suo padre) do
                u = parent(u)
            return $parent(u)$
    \end{lstlisting}

    Il successore di un nodo $u$ invece e' il minimo nel sottoalbero destro di $u$.

    Per implementare l'operazione \texttt{delete} devo considerare 3 casi:
    \begin{itemize}
        \item $u$ e' una foglia
        \item $u$ ha un solo figlio
        \item $u$ ha solo due figli
    \end{itemize}

    Nell'ultimo caso devo sostituire $u$ con il suo predecessore o il suo successore $v$.

    \paragraph{Considerazioni sul costo delle operazioni}
    Tutte hanno costo $O(h)$, ma al caso peggiore si ha $O(n)$ in caso di
    alberi profondi e sbilanciati.
    Un albero e' bilanciato se $h=O(log n)$

\subsubsection{Albero AVL}
    E' un BST, ma andiamo a considerare il fattore di bilanciamento $\beta(v)$, il cui modulo,  per ogni nodo, deve essere sempre $\leq 1$. 
    Questo valore si ottiene facendo la differenza delle altezze del sottoalbero sinistro e destro del nodo $v$.

    \paragraph{Altezza di un AVL}
    Tra tutti gli AVL, quello piu sbilanciato e' l'albero di Fibonacci: e' l'albero AVL di altezza $h$ con
    il numero minimo di nodi $n_h$.

    \begin{definition}
        L'altezza di un'albero AVL e' $h=O(log n)$
    \end{definition}
    \paragraph{Dimostrazione}
        Sappiamo che $n_h = F_{h+3}-1= \Theta(\phi^n)$.
        Quindi $h = \Theta(log n_h) = O(log n)$
    \paragraph{Implementazione del dizionario}
        Ogni volta che elimino o inserisco un nodo devo mantenere il bilanciamento grazie alle rotazioni, ho 4 casi in base a quale sottoalbero mi sbilancia $v$:
        \begin{itemize}
            \item SS: e' il sottoalbero sinistro del figlio sinistro di $v$
            \item DD: e' il sottoalbero destro del figlio destro di $v$
            \item SD: e' il sottoalbero destro del figlio sinistro
            \item DS: e' il sottoalbero sinistro del figlio destro
        \end{itemize}

        I casi sono simmetrici due a due. \\
        Il caso SS si risolve con una rotazione a destra con perno $v$.
        Il caso SD si risolve con una rotazione a sinistra sul figlio sinistro del nodo critico $z$ e l'altra verso destra sul nodo $v$

        L'operazione di rotazione richiede tempo costante.

        Di conseguenza posso implementare: \texttt{insert} e \texttt{delete} in tempo $O(logn)$ \\

        Per implementare l'operazione \texttt{insert} una volta inserito il valore, devo ricalcolare i fattori di bilanciamento,
        e una voltra trovato il nodo critico eseguo un bilanciamento tramite rotazione: \textbf{ne basta uno!} \\

        Per implementare l'operazione \texttt{delete} una volta eliminata la chiave, effettuo il bilanciamento sul nodo critico.
        Se il bilanciamento cambia l'altezza dell'albero devo controllare se al nodo superiore si e' sbilanciato qualcosa, e cosi via 
        fino alla radice.
        \newpage
    \section {Problema della Coda con priorita'}
        
\begin{lstlisting}[mathescape=True]
tipo CodaPriorita:
dati:
    un insieme di $S$ di $n$ elementi con chiavi prese da un campo ordinato
operazioni
    findMin() $\to elem$
    insert($elem e$, $chiave k$)
        aggiungi a $S$ un nuovo elemento $e$ con chiave $k$
    delete($elem e$)
        cancella da $S$ l'elemento $e$ 
            nota: ho il riferimento dell'oggetto!

    deleteMin()
        cancella da $S$ l'elemento con chiave minima
    
    increaseKey($elem e$, $chiave d$)
        incrementa della quantita $d$ la chiave $e$
    decreaseKey($elem e$, $chiave d$)
        decrementa della quantita $d$ la chiave $e$
    merge(CodaPriorita $c_1$, CodaPriorita $c_2$)
        restituisce una nuova coda: $c_3 = c_1 \cup c_2$
\end{lstlisting}

Ci sono varie implementazioni possibili: 
\begin{itemize}
    \item Con array non ordinato:  \texttt{insert} e \texttt{delete} eseguono in $O(1)$, \texttt{FindMin} e \texttt{DeleteMin} eseguono in $\Theta(n)$
    \item Con array ordinato: \texttt{insert} e \texttt{delete} eseguono in $O(n)$. Per la proprieta dell'ordinamento so trovare e eliminare il minimo in $O(1)$
    \item Lista non ordinata: tempi uguali all'array non ordinato
    \item Lista ordinata: tempi uguali all'array ordinato, ma riesco ad eliminare in $O(1)$
\end{itemize}

\subsection{d-Heap}
    \begin{lstlisting}[mathescape=true]
        procedura muoviAlto(v)
            while($v \neq radice(T)$ and $chiave(v) < chiave(padre(v))$) do
                svambia di posto $v$ e $padre(v)$ in $T$
            
        procedura muoviBasso(v)
            repeat
                sia $u$ il figlio di $v$ con chiave mininima
                if($v$ non ha figlio o $chiave(v) \leq chiave(u)$) break
                scambia $v$ e $u$ in $T$
    \end{lstlisting}

    L'operazione \texttt{insert} inserisce una nuova foglia e su questa ripristino la proprita di ordinamento. \\

    L'operazione \texttt{delete} scambia  il nodo $v$ con una foglia qualunque $u$, infine devo ripristinare
    la proprieta di ordinamento usando \texttt{muoviAlto} o \texttt{muoviBasso} ($O(d log_d n)$).\\

    L'operazione \texttt{decreaseKey} viene eseguita in tempo $O(log_d n)$ dato che devo eseguire solo muoviAlto
    al caso peggiore, invece \texttt{increaseKey} al caso peggiore esegue in $O(d log_d n)$.

    L'operazione \texttt{merge} puo' essere implementata in due modi: 
    \begin{itemize}
        \item Creo una nuova coda da 0
        \item Ad una delle due aggiungo l'altra. Con $k=min\{ |c_1|, |c_2|\}$, e quindi l'operazione ha costo $O(k log n)$
        Quand'e' che $k log n = o(n)$? risolvendo l'ugualgianza se: $k = o(n/logn)$, in questo caso conviene applicare questa tecnica.
    \end{itemize}

\subsection{Heap Binomiali}
    Nota: l'ordinamento non e' mantenuto orizzontalmente. \\

    \begin{lstlisting}
        procedura ristruttura()
            i=0
            while( esistono ancora due $B_i$) do
                fondi i due $B_i$ per formare un 
                albero $B_{i+1}$
                poni la radice del'albero con chiave
                minore, come padre dell'altro
                i+=1
    \end{lstlisting}

    \texttt{insert} si implementa aggiungendo un albero $B_0$, e applico \texttt{ristruttura},
    fino a che non si risolve. \\

    \texttt{deleteMin} trova la radice con chiave minima. \\

    \texttt{decreaseKey} aggiorna il valore della chiave e ripristina l'ordinamento. \\

    \texttt{delete} richiama \texttt{decreaseKey} fino a che il nodo non sale alla radice, poi viene eliminato
    in tempo costante.

    \texttt{merge} unisce $c1$ e $c2$ in un unico albero binomiale: con l'operazione \texttt{ristruttura} comincio
    a ripristinare la proprieta di unicita.

\subsection{Heap di Fibonacci}

        \newpage

    \section {Grafi}
        \begin{definition}{Grafo}

    Un \textbf{Grafo} $G=(V,E)$ consiste in un insieme $V$ di nodi
    e un insieme $E$ di coppie (non ordinate) di vertici: gli \textbf{archi}.
\end{definition}

\begin{definition}{Grafo diretto}

    Un \textbf{grafo diretto} $D=(V,A)$ e' un insieme di $V$ vertici e $A$ coppie 
    ordinate di vertici.
\end{definition} 

Terminologia standard: 
\begin{itemize}
    \item $n=|V|$
    \item $m=|E|$
    \item $(u,v)$ e' incidente a $u$ e $v$ (ossia gli estremi), se sono adiacenti.
    \item grado di $G = max_{v \in V}\{\Delta(v)\}$
\end{itemize}

\begin{definition}{Proprieta' del grado}
    \begin{itemize}
        \item $\sum_{v \in V} \Delta(v) = 2m$
        \item $\sum_{v \in V} \Delta_{out}(v) = \Delta_{in}(v) = m$
    \end{itemize}
\end{definition}

\begin{definition}{Grafo Pesato}
    
    Un grafo pesato $G=(V,E,w)$, e' un grafo in cui ad ogni arco
    la funzione $w$ gli associa un valore, di solitoreale.
\end{definition}


\begin{definition} {Albero}

    Un albero e' un grafo conesso ed aciclico
\end{definition}

\begin{definition}{Numero di archi in un albero.}

    Sia $T=(V,E)$ un'albero, allora $|E|=|V|-1$.
\end{definition}

\paragraph{Dimostrazione}
La definizione si dimostra per induzione su $|V|$. Per il caso base $|V|=1$ e' verificato.
Per il caso induttivo, poiche' $T$ e' connesso ha almeno una foglia e rimuovendola si ottiene un grafo
connesso e aciclico con $n-1$ nodi che ha $n-2$ archi. Aggiungendola si conclude che $T$ ha $n-1$ archi.

\begin{definition}{Ciclo Euleriano}

    Dato un grafo $G$, un \textbf{ciclo Euleriano} e' un cammino che passa
    per tutti gli archi di $G$ una ed una sola volta.

    Tale cammino esiste se e solo se tutti i nodi hanno \textbf{grado pari}, tranne due (ossia 
    gli estremi del cammino).
\end{definition}

\subsection{Tecniche di rappresentazione}

    Per i grafi non diretti:
    \begin{itemize}
        \item \textbf{Matrice di adiacenza}: ogni cella indica se esiste un'arco. $O(n^2)$
        \item \textbf{Liste di adiacenza}: lista di elementi $v_0,...,v_n$. Una lista di puntatore 
        descrive con quali nodi e' collegato $v_i$. $O(m+n)$
    \end{itemize}

    Per i grafi diretti si procede analogamente.

    Con la matrice i costi computazionali sono:
    \begin{itemize}
        \item G.N.: Elenco di archi incidenti in $v$: $O(n)$
        \item G.N.: C'e' l'arco $(u,v)$? $O(1)$
        \item G.D.: Elenco di archi uscenti da $v$: $O(n)$
        \item G.D.: C'e' l'arco $(u,v)$? $O(1)$
    \end{itemize}

    Con le liste i costi computazionali sono:
    \begin{itemize}
        \item G.N.: Elenco di archi incidenti in $v$: $O(\Delta(v))$
        \item G.N.: C'e' l'arco $(u,v)$? $O(min\{\Delta(u), \Delta(v)\})$
        \item G.D.: Elenco di archi uscenti da $v$: $O(\Delta(v))$
        \item G.D.: C'e' l'arco $(u,v)$? $O(\Delta(u))$
    \end{itemize}

\subsection{Scopo e tipi di vista}
Una visita di $G$ permette di esaminarne i nodi in modo sistematico:
\begin{itemize}
    \item BFS: visita in ampiezza
    \item DFS: visita in profondita
\end{itemize}

\subsubsection{Visita in ampiezza}
Dato una grafo $G$ e un nodo $s$ trova tutte le distanze/cammini minimi da 
$s$ verso ogni altro nodo $v$.

\begin{lstlisting}[mathescape=true]
    algoritmi visitaBFS(vertice s) $\to$ albero
        rendi tutti i vertici non marcati
        T = albero formato da un solo nodo $s$
        Coda F
        marca il vertice s; dist(s) = 0
        F.enqueue(s)
        while(not F.isempty()) dp
            u = F.dequeue()
            for each( arco(u,v) in G) do
                if($v$ non e' ancora marcato) then
                    F.enqueue($v$)
                    marca il vertice $v$; dist(v) = dist(u) + 1
                    rendi $u$ il padre di $v$ in $T$
        return T
\end{lstlisting}

L'algoritmo genera un'\textbf{albero BFS} dove i nodi sono ordinati 
in base alla distanza dal nodo: $s$.

Il costo dela visista in ampiezza e':
\begin{itemize}
    \item $O(n^2)$ se uso la matrice di adiacenza.
    \item $O(n+m)$ se uso le liste di adiacenza. Se il grafo e' connesso allora $m \geq n-1$, quindi $O(m)$.
    \subitem Inoltre $m \leq n(n-1)/2$, quindi e' vero che $O(m+n) = O(n^2)$.
    \subitem Se ho $m=o(n^2)$ le liste di adiacenza sono \textbf{efficienti}.
\end{itemize}

\begin{definition}{Livello e distanza da $v$}

    Per ogni nodo $v$, il suo livello nell'albero BFS e' pari
    alla distanza di $v$ dalla sorgente $s$.
\end{definition}

\paragraph{Dimostrazione Informale} 
    Intuitivamente \dots

\subsubsection{Visita in profondita'}
e' come esplorare un labirinto usando un gesso e una corda:
con la \textbf{corda} (nel codice una pila) riesco a tornare indietro se trovo un vicolo cieco,
con il \textbf{gesso} (nel codice una variabile booleana) segno le strade gia' controllate.

\begin{lstlisting}[mathescape=true]
    procedura visitaDFSRRicorsiva($vertice v, albero T$)
        marca e visita il vertice $v$
        for each( arco(v,w)) do
            if (w non e' marcati) then  
                aggiungi l'arco $(v,w)$ all'albero $T$
                visitaDFSRicorsiva(w,T)
    algoritmo visitaDFS($vertice s$) $\to albero$
        T = albero vuoto
        visitaDFSRicorsiva(s,t)
        return T
\end{lstlisting}

La visita \textbf{DFS} genera un'albero $T$ a partire dal vertice $s$,
la logica e' simile alla \textbf{vista in preordine} gia vista prima.

\newpage
\section{Applicazioni della visita DFS}
Provia a tenere in memoria il \texttt{clock}, questa variabile viene incrementata di
1 ogni volta che visito un nodo. Abbiamo due tempi:
\begin{itemize}
    \item $pre(v)$: il tempo in cui \textbf{inizio} a visitare il nodo $v$
    \item $post(v)$: il tempo in cui ho \textbf{finito} di visitare il nodo $v$
\end{itemize}

usiamo la notazione $\underset{v}{[} \ \underset{u}{[} \ \underset{u}{]} \ \underset{v}{]}$ per dire che il nodo $u$ e'
stato visitato dentro il tempo di visita di $v$, ovvero: $pre(v) < pre(u) < post(u) < post(v)$

Posso riconoscere grazie a $pre$ e $post$ il tipo di arco che c'e tra i nodi $v, u$:
\begin{itemize}
    \item \textbf{in avanti}: $\underset{v}{[} \ \underset{u}{[} \ \underset{u}{]} \ \underset{v}{]}$
    \item \textbf{in indietro}: $\underset{u}{[} \ \underset{v}{[} \ \underset{v}{]} \ \underset{u}{]}$
    \item \textbf{traversale}: $\underset{v}{[} \ \underset{v}{]} \ \underset{u}{[} \ \underset{u}{]}$
\end{itemize}

\subsection{Riconoscere la presenza di un ciclo in grafo diretto}
\begin{definition} Cicli con DFS \\
    In un grafo diretto $G$, c'e' un ciclo se e solo se \textbf{DFS} trova
    un arco all'indietro.
\end{definition}

Infatti se c'e' un arco all'indietro nell'albero allora e' possibile completare un ciclo (posso risalire l'albero).
Invece, se c'e' un ciclo $<v_0, v_1, \dots, v_k=v_0>$ e scopro per primo $v_i$, e a cascata scopro gli altri nodi. Pero' $v_i-1$
e' raggiungibile da $v_i$, che e' un arco all'indietro nell'albero.

\subsection{Ordinamento Topologico}
\begin{definition} DAG \\
    Un \textbf{DAG} e' un grafo che non contiene cicli diretti.
\end{definition}

\begin{definition} Ordinamento Topologico \\
    Un \textbf{ordinamento topologico} di un grafo diretto e' una funzione $\sigma: V \to \{1,2,\dots,n\}$ tale che
    per ogni arco $(u,v) \in E$, $\sigma(u)<\sigma(v)$
\end{definition}

In un'ordinamento topologico individuo due tipi di nodi:
\begin{itemize}
    \item \textbf{sorgente}: ha solo archi uscenti, ha $\sigma(v)$ con valore minimo.
    \item \textbf{pozzo}: ha solo archi entranti, ha $\sigma(v)$ con valore massimo
\end{itemize}

L'ordinamento topologico serve per costruire la \textbf{rete delle dipendenze}, devo 
trovare l'ordine per risolvere i compiti in modo da risolvere le dipendenza.

Quali grafi ammettono un'ordine topologico?
\begin{definition}
    
    Un grafo diretto $G$ ammette un ordinamento topologico se e solo se $G$ e' un \textbf{grafo senza cicili diretti}.
\end{definition}

Se per assurdo $G$ ha un ordinamento topologico e ha anche un ciclo $<v_0,v_1,\dots,v_k=v_0>$, allora dovrebbe
essere che: $\sigma(v_0) < \sigma(v_0) < \dots < \sigma(v_k) = \sigma(v_0)$.

Per dimostrare se $G$ e' un DAG allora ammette un ordinamento topologico forniamo un'algoritmo:
\begin{lstlisting}[mathescape=true]
    top=n; L=lista vuota;
    fai la visita DFS su G:
        quando finisco di visitare un nodo:
        $\sigma(v)=top$; $top-=1$
        aggiungi $v$ in testa alla lista $L$
\end{lstlisting}

Nota che in un $DAG$ non ci \textit{possono essere archi all'indietro} 

Un'implementazione alternativa e' la seguente:
\begin{lstlisting}[mathescape=true]
    $\hat{G}$ = G
    ord = lista vuota di vertici
    while(esiste un vertice $u$ 
            senza archi entranti in $\hat G$) do:
        appendi $u$ come ultimo elemento di $ord$
        rimuovi $u$ da $\hat G$ e tutti gli archi uscenti.
    
    if $\hat G$ non e' vuoto then errore
    return ord
\end{lstlisting}

Questo algoritmo prende se c'e' un nodo senza vertici entranti: se $G$ e' aciclico e' sempre 
possibili. Una volta che tolgo a $G$ il nodo analizzato, avro' un'altro grafo $G'$ aciclico.

\subsection{Componenti Fortemente Connesse}
Una componente fortemente connessa di un grafo $G$, e' un insieme \textbf{massimale} di vertici
$C \subseteq V$ tale che per ogni coppia di nodi $u,v$ in $C$, $u$ e' raggiungibile da $v$
e viceversa.

Per massimale si intende che se a $C$ gli aggiungo un nodo di $V$, la proprieta qui sopra non vale piu'.

L'algoritmo che si occupa di cercare queste componenti sfrutta le seguenti proprieta':
\begin{enumerate}
    \item Se faccio una vista DFS a partire dal nodo $u$, questa termina quando visito tutti i nodi raggiungibili da $u$.
    \item Se $C$ e $C'$ sono due componenti e c'e' un arco da un nodo in $C$ verso uno in $C'$ allora il valore $post()$ piu' grande in $C$
    e' maggiore del piu alto valore in post di $C'$. Se visito prima $C$ allora la visita termina quando visito tutti i nodi in $C$ e $C'$, e finisco di visitare su $C$, altrimenti
    se visito prima $C'$ la visita termina quando esaurisco i nodi in $C'$ e termina sempre su $C$.
    \item Il nodo che ha il valore piu grande di $post()$ appartiene a una componente sorgente. Come facevamo prima nel DAG.
\end{enumerate}

\begin{lstlisting}[mathescape=true]
    calcola $G^R$
    esegui DFS($G^R$) per trovare i valori post(v)
    return CompConnesse(G)

    CompConnesse(grafo G)
        for each nodo v do 
            imposta v come non marcato
        Comp = $\emptyset$
        for each $nodo$ $v$ in ordine
            decrescente di $post(v)$ do
            if($v$ e' non marcato) then
                T = albero vuoto
                visitaDFSRicorsiva(v,T)
                aggiugni T a Comp
        return comp
\end{lstlisting}

    \section {Grafi: Dijkstra}
        Con $G=(V,E,w)$ un grafo orientato o non, il  costo della lunghezza del cammino
$\pi = <v_0, v_1, \dots, v_k>$ e' $w(\pi) = \sum_{i=1}^k w(v_{i-1}, v_i)$.

Identifichiamo con $d_G(u,v)$ il costo del cammino minimo da $u$ a $v$, e puo' essere:
\begin{itemize}
    \item $d(u,v)=+\infty$ se non c'e' un cammino da $u$ a $v$
    \item $d(u,v)=-\infty$ se c'e' un ciclo con \textbf{costo negativo}
\end{itemize}

\begin{definition} Sottocammini di un cammino minimo \\
    Ogni sottocammino di un cammino minimo e' un cammino minimo.
\end{definition}

Si dimostra con \textbf{cut\&paste}. se $x$ e $y$ sono nel "cammino minimo" da $u$ a $v$ ma 
per la coppia $(x,y)$ esiste un'altro cammino minimo diverso dai nodi in $u$, $v$ allora il
cammino minimo da $u$ a $v$ e' diverso.

Vogliamo risolvere due problemi: 
\begin{itemize}
    \item Dato $G=(V,E,w)$, con $s \in V$, voglio calcolare le distanze di tutti i nodi da $s$. ossia tutti i $d_G(s,v)$.
    \item Dato $G=(V,E,w)$, con $s \in V$, voglio l'albero dei cammini minimi di $G$ radicato in $s$.
\end{itemize}

In teoria risolvere un problema vuol dire risolvere anche l'altro.

\subsection{Albero dei cammini minimi: SPT}
L'albero dei cammini di un grafo $G$ e' tale se per ogni $v \in V$ vale che $d_T(s,v)=d_G(s,v)$.
In oltre, per i grafi non pesati, e' vero che: \textit{l'albero SPT radicato in s e' uguale all'albero BFS radicato in s}.

Per trovare i cammini minimi si usa Dijkstra.
\subsection{Dijkstra}
\begin{enumerate}
    \item Mantengo per ogni nodo una \textbf{stima per eccesso} della distanza: $D_{sv}$
    \item Mantengo un insieme $X$ di nodi le cui stime sono \textbf{esatte}. Mantengo anche un'albero $T$
    che contiene i cammini minimi verso i nodi in $X$. Inizialmente: $X=\{s\}$, $T$ non ha archi.
    \item ad ogni passo aggiungo a $X$ un nodo $u$ con stima minima e aggiungo quell'arco a $T$.
    \item aggiorno le stime guardando i nodi adiacenti a $u$.
\end{enumerate}

la stima di un nodo $y$ e' $D_{sy} = min\{D_{sx} + w(x,y) : (x,y) \in E, x \in X\}$. 
E' una stima, perche' se poi troviamo un arco $(u,y)$ tale che: $D_{su} + w(u,y) < D_{sx} + w(x,y)$ allora
rimpiazziamo $(x,y)$ con $(u,y)$.

\textit{Le stime dei nodi si tengono in una coda con priorita.}

\begin{lstlisting}[mathescape=true]
    algoritmo Dijkstra($grafo$ $G$, $vertice$ $s$) $\to albero$
        for each(vertice $u$ in $G$) do $D_{su}$ = $+\infty$
        $\hat T$ = albero formato dal nodo $s$; $X=\emptyset$
        CodaPriorita S
        $D_{ss} = 0$
        while(not S.isEmpty()) do
            u = S.deleteMin(); $X=X \cap \{u\}$;
            for each( $arco(u,v)$ in $G$) do
                if($D_{sv} = +infty$) then
                    S.insert(v, $D_{su}$ + w(u,v))
                    $D_{sv} = D_{su} + w(u,v)$
                    rendi $u$ padre di $v$ in $\hat T$
                else if($D_{su} + w(u,v) < D_{sv}$) then
                    S.decreaseKey(v, $D_{sv} - D_{su} - w(u,v)$)
                    $D_{sv} = D_{su} + w(u,v)$
                    rendi $u$ nuovo padre di $v$ in $\hat T$
        return $\hat T$
\end{lstlisting}
\end{document}