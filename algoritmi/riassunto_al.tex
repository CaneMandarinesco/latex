\documentclass{article}

\usepackage[italian]{babel}
\usepackage{listings}
\usepackage{amsmath}
\usepackage{subfiles}

\title{Riassuntazzo Algoritmi e Strutture Dati Mod. 1}
\author{Davide Luci}
\date{2024}

\newtheorem{definition}{Definition}[section]

\begin{document}

    \maketitle
    \pagenumbering{gobble}
    \tableofcontents
    \newpage

    \pagenumbering{arabic}

    \setlength{\parindent}{0pt}
    \section{ Prima di leggere...}
    Questo documento non e' esaustivo al fine di comprendere al pieno la materia trattata.
    Le definizioni sono scarne, e spesso non esaustive, perche' ho preferito solo appuntare i concetti chiavi
    da ripassare in modo da trovare una giusta via di mezzo tra studiare e riassumere. Si puo' dire che sia compito del 
    lettore, quello di arricchire il contenuto di questo documento.

    \newpage

    \section*{ Introduzione}
    Diocane

    \newpage
    \section{ Notazione asintotica e Modelli di calcolo}

    \newpage
    \section{ Sorting }

    \newpage
    \section{ Capitolo 3: Strutture dati elementari }
        \subfile{strutture_dati_elementari.tex}
    \section {Problema del Dizionario}
        \subfile{problema_del_dizionario.tex}
    \end{document}