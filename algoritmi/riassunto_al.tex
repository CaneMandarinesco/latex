\documentclass{article}

\usepackage[italian]{babel}
\usepackage{listings}
\usepackage{amsmath}

\title{Riassuntazzo Algoritmi e Strutture Dati Mod. 1}
\author{Davide Luci}
\date{2024}

\newtheorem{definition}{Definition}[section]

\begin{document}

    \maketitle
    \pagenumbering{gobble}
    \tableofcontents
    \newpage

    \pagenumbering{arabic}

    \setlength{\parindent}{0pt}
    \section{ Prima di leggere...}
    Questo documento non e' esaustivo al fine di comprendere al pieno la materia trattata.
    Le definizioni sono scarne, e spesso non esaustive, perche' ho preferito solo appuntare i concetti chiavi
    da ripassare in modo da trovare una giusta via di mezzo tra studiare e riassumere. Si puo' dire che sia compito del 
    lettore, quello di arricchire il contenuto di questo documento.

    \newpage

    \section*{ Introduzione}
    Diocane

    \newpage
    \section{ Notazione asintotica e Modelli di calcolo}

    \newpage
    \section{ Sorting }

    \newpage
    \section{ Capitolo 3: Strutture dati elementari }
        \begin{definition}{Tipo di dato}
        
    Il \textbf{tipo di dato} definisce una collezione di oggetti, e che tipo di operazioni posso
    fare su questi.
\end{definition}

\begin{definition}{Struttura dati}

    La \textbf{struttura dati} e' l'organizzazione dei dati.
\end{definition}

    \subsection{Il Dizionario}
        \begin{lstlisting}[escapechar=@]
            tipo Dizionario: 
            dati:
                insieme @\(S\)@ di coppie @\((elem, chiave)\)@ 
            operazioni:
                insert(elem e, chiave k).
                delete(chiave k)
                search(chiave k) @\( \to \)@ elem

        \end{lstlisting}
    
    \subsection{La Pila}
    \begin{lstlisting}[escapechar=@]
        tipo Pila: 
        dati:
            sequenza @\(S\)@ di @\(n\)@ elementi. 
        operazioni:
            isEmpty().
            push(@\(elem\) \(e\)@)
                @aggiunge \(e\) come ultimo elemento di \(S\)@
            pop()
                @togli e restituisci l'ultimo elemento di \(S\)@
            top()
                @restituisci l'ultimo elemento di \(S\) senza toglierlo@

    \end{lstlisting}

    \newpage
    \subsection{La Coda}
    \begin{lstlisting}[escapechar=@]
        tipo Coda: 
        dati:
            sequenza @\(S\)@ di @\(n\)@ elementi. 
        operazioni:
            isEmpty().
            enqueue(@\(elem\) \(e\)@)
                @aggiunge \(e\) come ultimo elemento di \(S\)@
            dequeue()
                @togli e restituisci il primo elemento di \(S\)@
            first()
                @restituisci il primo elemento di \(S\) senza toglierlo@

    \end{lstlisting}

    \subsection{Tecniche di rappresentazione}
        Ci sono due tecniche per creare strutture dati:
        \begin{itemize}
            \item \textbf{Array} indicizzato.
            \item \textbf{Record} collegati fra puntatori.
        \end{itemize} 
        
        Un array indicizzato ha le seguenti proprieta:
        \begin{itemize}
            \item Prop. \textbf{Forte}: gli indici sono numeri consecutivi.
            \item Prop. \textbf{Debole}: non posso aggiungere nuove celle.
            \item \textbf{Contro}: dimensione fissa.
        \end{itemize}

        Se voglio implementare un dizionario con un array mi conviene usare un array \textbf{sovraddimensionato},
        che ha spazio per piu di \(n\) elementi. Cosi posso fare l'inserimento in \(O(1)\), 
        tuttavia \texttt{search} e \texttt{delete} richiedono \(O(n)\) \\

        Oppure posso usare un array \textbf{ordinato}, dove \texttt{search} costa \(O(\log n)\), mentre \texttt{insert} e \texttt{delete}
        costano \(O(n)\). \\

        La rappresentazione collegata, ovvero una lista, invece ha le seguenti proprieta:
        \begin{itemize}
            \item Prop. \textbf{Forte}: posso aggiungere o toglie record in tempo costante.
            \item Prop. \textbf{Debole}: i record non sono consecutivi.
            \item \textbf{Contro}: accesso sequenziale
        \end{itemize}

        Se per implementare il dizionario uso una lista non ordinata, ho che 
        \texttt{search} e \texttt{delete} impiegano \(O(n)\), mentre \texttt{insert} impiega tempo costante.

        La rappresentazione usando una lista ordinata invece e' sconveniente: mantenere la struttura ordinata
        aggiunge complessita', e tutte le operazioni richiedono \(O(n)\).
    
    \subsection{Alberi}
        Per rappresentare un'albero si puo' ricorrere ai \textbf{vettori posizionali}, dove ogni cella \(i\) contiene
        una coppia \((info, parent)\), ossia il suo contenuto \textbf{informativo} e l'indice del \textbf{padre}.

        Oppure si possono usare record collegati da puntatori, per esempio:
        \begin{itemize}
            \item Ogni record contiene due puntatori: figlio destro e sinistro. 
            \item Ogni record ha una lista di puntatori ai figli
            \item Ogni nodo, nel suo record, punta al primo figlio: \(v\). Nel caso ci siano altri figli, \(v\) punta al fratello successivo.
        \end{itemize}

    \subsection{Visite di alberi}
        \subsubsection{DFS}
        L'algoritmo DFS (Visita in Profondita'), parte dalla radice e procede visitando di figlio in figlio fino alla foglia.
        Una volta "visitata" la foglia si torna al padre e si controlla se ce ne siano altre, e si visitano. 
        Altrimenti si risale l'albero e si fa la stessa cosa con i nodi al livello superiore.
        
        \begin{lstlisting}[escapechar=@]
            algoritmo visitaDFS(@\(nodo\) \(r\)@)
                Pila S
                S.push(r)
                while (not S.isEmpty()) do
                    u = S.pop()
                    if(@\(u \neq null\)@) then
                        @visita il nodo u@
                        S.push(@figlio destro di \(u\)@)
                        S.push(@figlio sinistro di \(u\)@)
        \end{lstlisting}
        
        L'algoritmo inizia mettendo \(r\) nella radice. Appena
        entro nel loop, con \texttt{pop()} ottengo la radice, e vado a mettere
        figlio destro e sinistro nella Pila.
        All'iterazione successiva, con \texttt{pop()}, ottengo l'ultimo figlio messo con \texttt{push()},
        ossia il figlio sinistro di \(u\). 
        
        \subsubsection{DFS Ricorsivo}
        Esiste anche una versione ricorsiva, che funziona per alberi binari, ed e' molto interessante.
        \begin{lstlisting}[escapechar=@]
            algoritmo visitaDFSRicorsiva(@\(nodo\) \(r\)@)
                if(@\(r \neq null\)@) then
                    @visita il nodo \(r\)@
                    visitaDFSRicorsiva(@figlio sinistro di \(r\)@)
                    visitaDFSRicorsiva(@figlio destro di \(r\)@)
        \end{lstlisting}

        scambiando le tre righe dopo l'\texttt{if} posso:
        \begin{itemize}
            \item Visitare in preordine (DFS standard)
            \item Visitare simmetricamente: prima chiamo \texttt{visitaDFSRicorsiva} sul figlio sinistro, poi visito e poi faccio la chiamata sul figlio destro.
            \item Vistare in postordine: prima chiamo \texttt{visitaDFSRicorsiva} sul figlio sinistro, poi su quello destro e infine visito.
        \end{itemize}
        
        \subsubsection{BFS}
        Partendo dalla radice, prima di scendere al livello sotto, visito tutti i nodi. Si implementa con 
        la Coda.

        \begin{lstlisting}[escapechar=@]
            algoritmo visitaBFS(@\(nodo\) \(r\)@)
                Coda C
                C.enqueue(r)
                while (not C.isEmpty()) do
                    u = C.dequeue()
                    if(@\(u \neq null\)@) then
                        @visita il nodo \(u\)@
                        C.enqueue(@figlio sintro di \(u\)@)
                        C.enqueue(@figlio destro di \(u\)@)
        \end{lstlisting}

        La logica di funzionamento e' simile al DFS.

        \newpage
    \section {Problema del Dizionario}
        \subsection{Implementare il dizionario}
Per l'implementazione abbiamo due possibilita:
\begin{itemize}
    \item alberi BST (binario di ricerca): \(O(\text{altezza albero})\)
    \item albero AVL: \(O(log n)\)
\end{itemize}

\subsubsection{Albero BST}
    Ogni nodo ha \(elem(v)\) e \(chiave(v)\), la chiave si trova in un dominio totalmente ordinato
    e vale che:
    \begin{itemize}
        \item nel sottoalbero sinitro di \(v\) le chavi sono \(\leq chiave(v)\)
        \item nel sottoalbero destro di \(v\) le chavi sono \(\ge chiave(v)\)
    \end{itemize}

    \begin{definition}{Proprieta del BST}

        Nella foglia piu a sinistra si trova il minimo. Nel nodo piu a destra si trova il massimo.
    \end{definition}

    \begin{definition}{Visita simmetrica del BST}

        Usando la visita \textbf{simmetrica} su un BST, si guardano i nodi in ordine crescente. 
    \end{definition}
    
    \paragraph{Dimostrazione}
        Per induzione, con un albero di altezza \(h=1\) e' ovvio. \\
        Ora per un \(h\) generico, i figli della radice sono alberi di altezza $\leq h-1$.
        Per ipotesi induttiva questo e' visitato correttamente, allora l'albero di altezza $h$ e' visitato correttamente.

    \begin{definition}{predecessore e successore}

        Il predecessore di un nodo $u$ e' il nodo $v$ con chiave massima $\leq chiave(u)$ \\
        Il successore di un nodo $u$ e' il nodo $v$ avente minima chiave $\geq chiave(u)$
    \end{definition}

    \paragraph{Implementare l'operazione delete}
    Il predecessore di un nodo:
    \begin{itemize}
        \item si trova nel sottoalbero sinistro
        \item se non c'e' un sottoalbero sinistro il predecessore e' un suo antenato
    \end{itemize}

    \begin{lstlisting}[mathescape=true]
        algoritmo pred(nodo u) $\to nodo$
            if($u$ ha figlio sinistro $sin(u)$) then
                return max($sin(u)$)
            while ($parent(u) \neq null$ e $u$ e' figlio sinistro di suo padre) do
                u = parent(u)
            return $parent(u)$
    \end{lstlisting}

    Il successore di un nodo $u$ invece e' il minimo nel sottoalbero destro di $u$.

    Per implementare l'operazione \texttt{delete} devo considerare 3 casi:
    \begin{itemize}
        \item $u$ e' una foglia
        \item $u$ ha un solo figlio
        \item $u$ ha solo due figli
    \end{itemize}

    Nell'ultimo caso devo sostituire $u$ con il suo predecessore o il suo successore $v$.

    \paragraph{Considerazioni sul costo delle operazioni}
    Tutte hanno costo $O(h)$, ma al caso peggiore si ha $O(n)$ in caso di
    alberi profondi e sbilanciati.
    Un albero e' bilanciato se $h=O(log n)$

\subsubsection{Albero AVL}
    E' un BST, ma andiamo a considerare il fattore di bilanciamento $\beta(v)$, il cui modulo,  per ogni nodo, deve essere sempre $\leq 1$. 
    Questo valore si ottiene facendo la differenza delle altezze del sottoalbero sinistro e destro del nodo $v$.

    \paragraph{Altezza di un AVL}
    Tra tutti gli AVL, quello piu sbilanciato e' l'albero di Fibonacci: e' l'albero AVL di altezza $h$ con
    il numero minimo di nodi $n_h$.

    \begin{definition}
        L'altezza di un'albero AVL e' $h=O(log n)$
    \end{definition}
    \paragraph{Dimostrazione}
        Sappiamo che $n_h = F_{h+3}-1= \Theta(\phi^n)$.
        Quindi $h = \Theta(log n_h) = O(log n)$
    \paragraph{Implementazione del dizionario}
        Ogni volta che elimino o inserisco un nodo devo mantenere il bilanciamento grazie alle rotazioni, ho 4 casi in base a quale sottoalbero mi sbilancia $v$:
        \begin{itemize}
            \item SS: e' il sottoalbero sinistro del figlio sinistro di $v$
            \item DD: e' il sottoalbero destro del figlio destro di $v$
            \item SD: e' il sottoalbero destro del figlio sinistro
            \item DS: e' il sottoalbero sinistro del figlio destro
        \end{itemize}

        I casi sono simmetrici due a due. \\
        Il caso SS si risolve con una rotazione a destra con perno $v$.
        Il caso SD si risolve con una rotazione a sinistra sul figlio sinistro del nodo critico $z$ e l'altra verso destra sul nodo $v$

        L'operazione di rotazione richiede tempo costante.

        Di conseguenza posso implementare: \texttt{insert} e \texttt{delete} in tempo $O(logn)$ \\

        Per implementare l'operazione \texttt{insert} una volta inserito il valore, devo ricalcolare i fattori di bilanciamento,
        e una voltra trovato il nodo critico eseguo un bilanciamento tramite rotazione: \textbf{ne basta uno!} \\

        Per implementare l'operazione \texttt{delete} una volta eliminata la chiave, effettuo il bilanciamento sul nodo critico.
        Se il bilanciamento cambia l'altezza dell'albero devo controllare se al nodo superiore si e' sbilanciato qualcosa, e cosi via 
        fino alla radice.
        \newpage
    \section {Problema della Coda con priorita'}
        
\begin{lstlisting}[mathescape=True]
tipo CodaPriorita:
dati:
    un insieme di $S$ di $n$ elementi con chiavi prese da un campo ordinato
operazioni
    findMin() $\to elem$
    insert($elem e$, $chiave k$)
        aggiungi a $S$ un nuovo elemento $e$ con chiave $k$
    delete($elem e$)
        cancella da $S$ l'elemento $e$ 
            nota: ho il riferimento dell'oggetto!

    deleteMin()
        cancella da $S$ l'elemento con chiave minima
    
    increaseKey($elem e$, $chiave d$)
        incrementa della quantita $d$ la chiave $e$
    decreaseKey($elem e$, $chiave d$)
        decrementa della quantita $d$ la chiave $e$
    merge(CodaPriorita $c_1$, CodaPriorita $c_2$)
        restituisce una nuova coda: $c_3 = c_1 \cup c_2$
\end{lstlisting}

Ci sono varie implementazioni possibili: 
\begin{itemize}
    \item Con array non ordinato:  \texttt{insert} e \texttt{delete} eseguono in $O(1)$, \texttt{FindMin} e \texttt{DeleteMin} eseguono in $\Theta(n)$
    \item Con array ordinato: \texttt{insert} e \texttt{delete} eseguono in $O(n)$. Per la proprieta dell'ordinamento so trovare e eliminare il minimo in $O(1)$
    \item Lista non ordinata: tempi uguali all'array non ordinato
    \item Lista ordinata: tempi uguali all'array ordinato, ma riesco ad eliminare in $O(1)$
\end{itemize}

\subsection{d-Heap}
    \begin{lstlisting}[mathescape=true]
        procedura muoviAlto(v)
            while($v \neq radice(T)$ and $chiave(v) < chiave(padre(v))$) do
                svambia di posto $v$ e $padre(v)$ in $T$
            
        procedura muoviBasso(v)
            repeat
                sia $u$ il figlio di $v$ con chiave mininima
                if($v$ non ha figlio o $chiave(v) \leq chiave(u)$) break
                scambia $v$ e $u$ in $T$
    \end{lstlisting}

    L'operazione \texttt{insert} inserisce una nuova foglia e su questa ripristino la proprita di ordinamento. \\

    L'operazione \texttt{delete} scambia  il nodo $v$ con una foglia qualunque $u$, infine devo ripristinare
    la proprieta di ordinamento usando \texttt{muoviAlto} o \texttt{muoviBasso} ($O(d log_d n)$).\\

    L'operazione \texttt{decreaseKey} viene eseguita in tempo $O(log_d n)$ dato che devo eseguire solo muoviAlto
    al caso peggiore, invece \texttt{increaseKey} al caso peggiore esegue in $O(d log_d n)$.

    L'operazione \texttt{merge} puo' essere implementata in due modi: 
    \begin{itemize}
        \item Creo una nuova coda da 0
        \item Ad una delle due aggiungo l'altra. Con $k=min\{ |c_1|, |c_2|\}$, e quindi l'operazione ha costo $O(k log n)$
        Quand'e' che $k log n = o(n)$? risolvendo l'ugualgianza se: $k = o(n/logn)$, in questo caso conviene applicare questa tecnica.
    \end{itemize}

\subsection{Heap Binomiali}
    Nota: l'ordinamento non e' mantenuto orizzontalmente. \\

    \begin{lstlisting}
        procedura ristruttura()
            i=0
            while( esistono ancora due $B_i$) do
                fondi i due $B_i$ per formare un 
                albero $B_{i+1}$
                poni la radice del'albero con chiave
                minore, come padre dell'altro
                i+=1
    \end{lstlisting}

    \texttt{insert} si implementa aggiungendo un albero $B_0$, e applico \texttt{ristruttura},
    fino a che non si risolve. \\

    \texttt{deleteMin} trova la radice con chiave minima. \\

    \texttt{decreaseKey} aggiorna il valore della chiave e ripristina l'ordinamento. \\

    \texttt{delete} richiama \texttt{decreaseKey} fino a che il nodo non sale alla radice, poi viene eliminato
    in tempo costante.

    \texttt{merge} unisce $c1$ e $c2$ in un unico albero binomiale: con l'operazione \texttt{ristruttura} comincio
    a ripristinare la proprieta di unicita.

\subsection{Heap di Fibonacci}

    \end{document}