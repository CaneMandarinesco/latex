\begin{definition}{Grafo}

    Un \textbf{Grafo} $G=(V,E)$ consiste in un insieme $V$ di nodi
    e un insieme $E$ di coppie (non ordinate) di vertici: gli \textbf{archi}.
\end{definition}

\begin{definition}{Grafo diretto}

    Un \textbf{grafo diretto} $D=(V,A)$ e' un insieme di $V$ vertici e $A$ coppie 
    ordinate di vertici.
\end{definition} 

Terminologia standard: 
\begin{itemize}
    \item $n=|V|$
    \item $m=|E|$
    \item $(u,v)$ e' incidente a $u$ e $v$ (ossia gli estremi), se sono adiacenti.
    \item grado di $G = max_{v \in V}\{\Delta(v)\}$
\end{itemize}

\begin{definition}{Proprieta' del grado}
    \begin{itemize}
        \item $\sum_{v \in V} \Delta(v) = 2m$
        \item $\sum_{v \in V} \Delta_{out}(v) = \Delta_{in}(v) = m$
    \end{itemize}
\end{definition}

\begin{definition}{Grafo Pesato}
    
    Un grafo pesato $G=(V,E,w)$, e' un grafo in cui ad ogni arco
    la funzione $w$ gli associa un valore, di solitoreale.
\end{definition}


\begin{definition} {Albero}

    Un albero e' un grafo conesso ed aciclico
\end{definition}

\begin{definition}{Numero di archi in un albero.}

    Sia $T=(V,E)$ un'albero, allora $|E|=|V|-1$.
\end{definition}

\paragraph{Dimostrazione}
La definizione si dimostra per induzione su $|V|$. Per il caso base $|V|=1$ e' verificato.
Per il caso induttivo, poiche' $T$ e' connesso ha almeno una foglia e rimuovendola si ottiene un grafo
connesso e aciclico con $n-1$ nodi che ha $n-2$ archi. Aggiungendola si conclude che $T$ ha $n-1$ archi.

\begin{definition}{Ciclo Euleriano}

    Dato un grafo $G$, un \textbf{ciclo Euleriano} e' un cammino che passa
    per tutti gli archi di $G$ una ed una sola volta.

    Tale cammino esiste se e solo se tutti i nodi hanno \textbf{grado pari}, tranne due (ossia 
    gli estremi del cammino).
\end{definition}

\subsection{Tecniche di rappresentazione}

    Per i grafi non diretti:
    \begin{itemize}
        \item \textbf{Matrice di adiacenza}: ogni cella indica se esiste un'arco. $O(n^2)$
        \item \textbf{Liste di adiacenza}: lista di elementi $v_0,...,v_n$. Una lista di puntatore 
        descrive con quali nodi e' collegato $v_i$. $O(m+n)$
    \end{itemize}

    Per i grafi diretti si procede analogamente.

    Con la matrice i costi computazionali sono:
    \begin{itemize}
        \item G.N.: Elenco di archi incidenti in $v$: $O(n)$
        \item G.N.: C'e' l'arco $(u,v)$? $O(1)$
        \item G.D.: Elenco di archi uscenti da $v$: $O(n)$
        \item G.D.: C'e' l'arco $(u,v)$? $O(1)$
    \end{itemize}

    Con le liste i costi computazionali sono:
    \begin{itemize}
        \item G.N.: Elenco di archi incidenti in $v$: $O(\Delta(v))$
        \item G.N.: C'e' l'arco $(u,v)$? $O(min\{\Delta(u), \Delta(v)\})$
        \item G.D.: Elenco di archi uscenti da $v$: $O(\Delta(v))$
        \item G.D.: C'e' l'arco $(u,v)$? $O(\Delta(u))$
    \end{itemize}

\subsection{Scopo e tipi di vista}
Una visita di $G$ permette di esaminarne i nodi in modo sistematico:
\begin{itemize}
    \item BFS: visita in ampiezza
    \item DFS: visita in profondita
\end{itemize}

\subsubsection{Visita in ampiezza}
Dato una grafo $G$ e un nodo $s$ trova tutte le distanze/cammini minimi da 
$s$ verso ogni altro nodo $v$.

\begin{lstlisting}[mathescape=true]
    algoritmi visitaBFS(vertice s) $\to$ albero
        rendi tutti i vertici non marcati
        T = albero formato da un solo nodo $s$
        Coda F
        marca il vertice s; dist(s) = 0
        F.enqueue(s)
        while(not F.isempty()) dp
            u = F.dequeue()
            for each( arco(u,v) in G) do
                if($v$ non e' ancora marcato) then
                    F.enqueue($v$)
                    marca il vertice $v$; dist(v) = dist(u) + 1
                    rendi $u$ il padre di $v$ in $T$
        return T
\end{lstlisting}

L'algoritmo genera un'\textbf{albero BFS} dove i nodi sono ordinati 
in base alla distanza dal nodo: $s$.

Il costo dela visista in ampiezza se uso la matrice di adiacenza e' $O(n^2)$,
invece con le liste di adiacenza:  $O(n+m)$, se il grafo e' \textbf{connesso} allora $m\geq n-1$ e quindi
$O(n+m)=O(m)$. Sappiamo anche che $m\leq n(n-1)/2$ quindi $O(m+n)=O(n^2)$, per $m=o(n^2)$ l'algoritmo e molto efficiente.

\begin{definition}{Livello e distanza da $v$}

    Per ogni nodo $v$, il suo livello nell'albero BFS e' pari
    alla distanza di $v$ dalla sorgente $s$.
\end{definition}

\paragraph{Dimostrazione Informale} 
    Intuitivamente \dots

\subsubsection{Visita in profondita'}
e' come esplorare un labirinto usando un gesso e una corda:
con la \textbf{corda} (nel codice una pila) riesco a tornare indietro se trovo un vicolo cieco,
con il \textbf{gesso} (nel codice una variabile booleana) segno le strade gia' controllate.

\begin{lstlisting}[mathescape=true]
    procedura visitaDFSRRicorsiva($vertice v, albero T$)
        marca e visita il vertice $v$
        for each( arco(v,w)) do
            if (w non e' marcati) then  
                aggiungi l'arco $(v,w)$ all'albero $T$
                visitaDFSRicorsiva(w,T)
    algoritmo visitaDFS($vertice s$) $\to albero$
        T = albero vuoto
        visitaDFSRicorsiva(s,t)
        return T
\end{lstlisting}

La visita \textbf{DFS} genera un'albero $T$ a partire dal vertice $s$,
la logica e' simile alla \textbf{vista in preordine} gia vista prima.