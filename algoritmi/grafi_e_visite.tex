\begin{definition}{Grafo}

    Un \textbf{Grafo} $G=(V,E)$ consiste in un insieme $V$ di nodi
    e un insieme $E$ di coppie (non ordinate) di vertici: gli \textbf{archi}.
\end{definition}

\begin{definition}{Grafo diretto}

    Un \textbf{grafo diretto} $D=(V,A)$ e' un insieme di $V$ vertici e $A$ coppie 
    ordinate di vertici.
\end{definition} 

Terminologia standard: 
\begin{itemize}
    \item $n=|V|$
    \item $m=|E|$
    \item $(u,v)$ e' incidente a $u$ e $v$ (ossia gli estremi), se sono adiacenti.
    \item grado di $G = max_{v \in V}\{\Delta(v)\}$
\end{itemize}

\begin{definition}{Proprieta' del grado}
    \begin{itemize}
        \item $\sum_{v \in V} \Delta(v) = 2m$
        \item $\sum_{v \in V} \Delta_{out}(v) = \Delta_{in}(v) = m$
    \end{itemize}
\end{definition}

\begin{definition}{Grafo Pesato}
    
    Un grafo pesato $G=(V,E,w)$, e' un grafo in cui ad ogni arco
    la funzione $w$ gli associa un valore, di solitoreale.
\end{definition}


\begin{definition} {Albero}

    Un albero e' un grafo conesso ed aciclico
\end{definition}

\begin{definition}{Numero di archi in un albero.}

    Sia $T=(V,E)$ un'albero, allora $|E|=|V|-1$.
\end{definition}

\paragraph{Dimostrazione}
La definizione si dimostra per induzione su $|V|$. Per il caso base $|V|=1$ e' verificato.
Per il caso induttivo, poiche' $T$ e' connesso ha almeno una foglia e rimuovendola si ottiene un grafo
connesso e aciclico con $n-1$ nodi che ha $n-2$ archi. Aggiungendola si conclude che $T$ ha $n-1$ archi.

\begin{definition}{Ciclo Euleriano}

    Dato un grafo $G$, un \textbf{ciclo Euleriano} e' un cammino che passa
    per tutti gli archi di $G$ una ed una sola volta.

    Tale cammino esiste se e solo se tutti i nodi hanno \textbf{grado pari}, tranne due (ossia 
    gli estremi del cammino).
\end{definition}

\subsection{Tecniche di rappresentazione}

    Per i grafi non diretti:
    \begin{itemize}
        \item \textbf{Matrice di adiacenza}: ogni cella indica se esiste un'arco. $O(n^2)$
        \item \textbf{Liste di adiacenza}: lista di elementi $v_0,...,v_n$. Una lista di puntatore 
        descrive con quali nodi e' collegato $v_i$. $O(m+n)$
    \end{itemize}

    Per i grafi diretti si procede analogamente.

    Con la matrice i costi computazionali sono:
    \begin{itemize}
        \item G.N.: Elenco di archi incidenti in $v$: $O(n)$
        \item G.N.: C'e' l'arco $(u,v)$? $O(1)$
        \item G.D.: Elenco di archi uscenti da $v$: $O(n)$
        \item G.D.: C'e' l'arco $(u,v)$? $O(1)$
    \end{itemize}

    Con le liste i costi computazionali sono:
    \begin{itemize}
        \item G.N.: Elenco di archi incidenti in $v$: $O(\Delta(v))$
        \item G.N.: C'e' l'arco $(u,v)$? $O(min\{\Delta(u), \Delta(v)\})$
        \item G.D.: Elenco di archi uscenti da $v$: $O(\Delta(v))$
        \item G.D.: C'e' l'arco $(u,v)$? $O(\Delta(u))$
    \end{itemize}

\subsection{Scopo e tipi di vista}
Una visita di $G$ permette di esaminarne i nodi in modo sistematico:
\begin{itemize}
    \item BFS: visita in ampiezza
    \item DFS: visita in profondita
\end{itemize}

\subsubsection{Visita in ampiezza}
Dato una grafo $G$ e un nodo $s$ trova tutte le distanze/cammini minimi da 
$s$ verso ogni altro nodo $v$.

\begin{lstlisting}[mathescape=true]
    algoritmi visitaBFS(vertice s) $\to$ albero
        rendi tutti i vertici non marcati
        T = albero formato da un solo nodo $s$
        Coda F
        marca il vertice s; dist(s) = 0
        F.enqueue(s)
        while(not F.isempty()) dp
            u = F.dequeue()
            for each( arco(u,v) in G) do
                if($v$ non e' ancora marcato) then
                    F.enqueue($v$)
                    marca il vertice $v$; dist(v) = dist(u) + 1
                    rendi $u$ il padre di $v$ in $T$
        return T
\end{lstlisting}

L'algoritmo genera un'\textbf{albero BFS} dove i nodi sono ordinati 
in base alla distanza dal nodo: $s$.

Il costo dela visista in ampiezza e':
\begin{itemize}
    \item $O(n^2)$ se uso la matrice di adiacenza.
    \item $O(n+m)$ se uso le liste di adiacenza. Se il grafo e' connesso allora $m \geq n-1$, quindi $O(m)$.
    \subitem Inoltre $m \leq n(n-1)/2$, quindi e' vero che $O(m+n) = O(n^2)$.
    \subitem Se ho $m=o(n^2)$ le liste di adiacenza sono \textbf{efficienti}.
\end{itemize}

\begin{definition}{Livello e distanza da $v$}

    Per ogni nodo $v$, il suo livello nell'albero BFS e' pari
    alla distanza di $v$ dalla sorgente $s$.
\end{definition}

\paragraph{Dimostrazione Informale} 
    Intuitivamente \dots

\subsubsection{Visita in profondita'}
e' come esplorare un labirinto usando un gesso e una corda:
con la \textbf{corda} (nel codice una pila) riesco a tornare indietro se trovo un vicolo cieco,
con il \textbf{gesso} (nel codice una variabile booleana) segno le strade gia' controllate.

\begin{lstlisting}[mathescape=true]
    procedura visitaDFSRRicorsiva($vertice v, albero T$)
        marca e visita il vertice $v$
        for each( arco(v,w)) do
            if (w non e' marcati) then  
                aggiungi l'arco $(v,w)$ all'albero $T$
                visitaDFSRicorsiva(w,T)
    algoritmo visitaDFS($vertice s$) $\to albero$
        T = albero vuoto
        visitaDFSRicorsiva(s,t)
        return T
\end{lstlisting}

La visita \textbf{DFS} genera un'albero $T$ a partire dal vertice $s$,
la logica e' simile alla \textbf{vista in preordine} gia vista prima.

\newpage
\section{Applicazioni della visita DFS}
Provia a tenere in memoria il \texttt{clock}, questa variabile viene incrementata di
1 ogni volta che visito un nodo. Abbiamo due tempi:
\begin{itemize}
    \item $pre(v)$: il tempo in cui \textbf{inizio} a visitare il nodo $v$
    \item $post(v)$: il tempo in cui ho \textbf{finito} di visitare il nodo $v$
\end{itemize}

usiamo la notazione $\underset{v}{[} \ \underset{u}{[} \ \underset{u}{]} \ \underset{v}{]}$ per dire che il nodo $u$ e'
stato visitato dentro il tempo di visita di $v$, ovvero: $pre(v) < pre(u) < post(u) < post(v)$

Posso riconoscere grazie a $pre$ e $post$ il tipo di arco che c'e tra i nodi $v, u$:
\begin{itemize}
    \item \textbf{in avanti}: $\underset{v}{[} \ \underset{u}{[} \ \underset{u}{]} \ \underset{v}{]}$
    \item \textbf{in indietro}: $\underset{u}{[} \ \underset{v}{[} \ \underset{v}{]} \ \underset{u}{]}$
    \item \textbf{traversale}: $\underset{v}{[} \ \underset{v}{]} \ \underset{u}{[} \ \underset{u}{]}$
\end{itemize}

\subsection{Riconoscere la presenza di un ciclo in grafo diretto}
\begin{definition} Cicli con DFS \\
    In un grafo diretto $G$, c'e' un ciclo se e solo se \textbf{DFS} trova
    un arco all'indietro.
\end{definition}

Infatti se c'e' un arco all'indietro nell'albero allora e' possibile completare un ciclo (posso risalire l'albero).
Invece, se c'e' un ciclo $<v_0, v_1, \dots, v_k=v_0>$ e scopro per primo $v_i$, e a cascata scopro gli altri nodi. Pero' $v_i-1$
e' raggiungibile da $v_i$, che e' un arco all'indietro nell'albero.

\subsection{Ordinamento Topologico}
\begin{definition} DAG \\
    Un \textbf{DAG} e' un grafo che non contiene cicli diretti.
\end{definition}

\begin{definition} Ordinamento Topologico \\
    Un \textbf{ordinamento topologico} di un grafo diretto e' una funzione $\sigma: V \to \{1,2,\dots,n\}$ tale che
    per ogni arco $(u,v) \in E$, $\sigma(u)<\sigma(v)$
\end{definition}

In un'ordinamento topologico individuo due tipi di nodi:
\begin{itemize}
    \item \textbf{sorgente}: ha solo archi uscenti, ha $\sigma(v)$ con valore minimo.
    \item \textbf{pozzo}: ha solo archi entranti, ha $\sigma(v)$ con valore massimo
\end{itemize}

L'ordinamento topologico serve per costruire la \textbf{rete delle dipendenze}, devo 
trovare l'ordine per risolvere i compiti in modo da risolvere le dipendenza.

Quali grafi ammettono un'ordine topologico?
\begin{definition}
    
    Un grafo diretto $G$ ammette un ordinamento topologico se e solo se $G$ e' un \textbf{grafo senza cicili diretti}.
\end{definition}

Se per assurdo $G$ ha un ordinamento topologico e ha anche un ciclo $<v_0,v_1,\dots,v_k=v_0>$, allora dovrebbe
essere che: $\sigma(v_0) < \sigma(v_0) < \dots < \sigma(v_k) = \sigma(v_0)$.

Per dimostrare se $G$ e' un DAG allora ammette un ordinamento topologico forniamo un'algoritmo:
\begin{lstlisting}[mathescape=true]
    top=n; L=lista vuota;
    fai la visita DFS su G:
        quando finisco di visitare un nodo:
        $\sigma(v)=top$; $top-=1$
        aggiungi $v$ in testa alla lista $L$
\end{lstlisting}

Nota che in un $DAG$ non ci \textit{possono essere archi all'indietro} 

Un'implementazione alternativa e' la seguente:
\begin{lstlisting}[mathescape=true]
    $\hat{G}$ = G
    ord = lista vuota di vertici
    while(esiste un vertice $u$ 
            senza archi entranti in $\hat G$) do:
        appendi $u$ come ultimo elemento di $ord$
        rimuovi $u$ da $\hat G$ e tutti gli archi uscenti.
    
    if $\hat G$ non e' vuoto then errore
    return ord
\end{lstlisting}

Questo algoritmo prende se c'e' un nodo senza vertici entranti: se $G$ e' aciclico e' sempre 
possibili. Una volta che tolgo a $G$ il nodo analizzato, avro' un'altro grafo $G'$ aciclico.

\subsection{Componenti Fortemente Connesse}
Una componente fortemente connessa di un grafo $G$, e' un insieme \textbf{massimale} di vertici
$C \subseteq V$ tale che per ogni coppia di nodi $u,v$ in $C$, $u$ e' raggiungibile da $v$
e viceversa.

Per massimale si intende che se a $C$ gli aggiungo un nodo di $V$, la proprieta qui sopra non vale piu'.

L'algoritmo che si occupa di cercare queste componenti sfrutta le seguenti proprieta':
\begin{enumerate}
    \item Se faccio una vista DFS a partire dal nodo $u$, questa termina quando visito tutti i nodi raggiungibili da $u$.
    \item Se $C$ e $C'$ sono due componenti e c'e' un arco da un nodo in $C$ verso uno in $C'$ allora il valore $post()$ piu' grande in $C$
    e' maggiore del piu alto valore in post di $C'$. Se visito prima $C$ allora la visita termina quando visito tutti i nodi in $C$ e $C'$, e finisco di visitare su $C$, altrimenti
    se visito prima $C'$ la visita termina quando esaurisco i nodi in $C'$ e termina sempre su $C$.
    \item Il nodo che ha il valore piu grande di $post()$ appartiene a una componente sorgente. Come facevamo prima nel DAG.
\end{enumerate}

\begin{lstlisting}[mathescape=true]
    calcola $G^R$
    esegui DFS($G^R$) per trovare i valori post(v)
    return CompConnesse(G)

    CompConnesse(grafo G)
        for each nodo v do 
            imposta v come non marcato
        Comp = $\emptyset$
        for each $nodo$ $v$ in ordine
            decrescente di $post(v)$ do
            if($v$ e' non marcato) then
                T = albero vuoto
                visitaDFSRicorsiva(v,T)
                aggiugni T a Comp
        return comp
\end{lstlisting}