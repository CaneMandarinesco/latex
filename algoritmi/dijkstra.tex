Con $G=(V,E,w)$ un grafo orientato o non, il  costo della lunghezza del cammino
$\pi = <v_0, v_1, \dots, v_k>$ e' $w(\pi) = \sum_{i=1}^k w(v_{i-1}, v_i)$.

Identifichiamo con $d_G(u,v)$ il costo del cammino minimo da $u$ a $v$, e puo' essere:
\begin{itemize}
    \item $d(u,v)=+\infty$ se non c'e' un cammino da $u$ a $v$
    \item $d(u,v)=-\infty$ se c'e' un ciclo con \textbf{costo negativo}
\end{itemize}

\begin{definition} Sottocammini di un cammino minimo \\
    Ogni sottocammino di un cammino minimo e' un cammino minimo.
\end{definition}

Si dimostra con \textbf{cut\&paste}. se $x$ e $y$ sono nel "cammino minimo" da $u$ a $v$ ma 
per la coppia $(x,y)$ esiste un'altro cammino minimo diverso dai nodi in $u$, $v$ allora il
cammino minimo da $u$ a $v$ e' diverso.

Vogliamo risolvere due problemi: 
\begin{itemize}
    \item Dato $G=(V,E,w)$, con $s \in V$, voglio calcolare le distanze di tutti i nodi da $s$. ossia tutti i $d_G(s,v)$.
    \item Dato $G=(V,E,w)$, con $s \in V$, voglio l'albero dei cammini minimi di $G$ radicato in $s$.
\end{itemize}

In teoria risolvere un problema vuol dire risolvere anche l'altro.

\subsection{Albero dei cammini minimi: SPT}
L'albero dei cammini di un grafo $G$ e' tale se per ogni $v \in V$ vale che $d_T(s,v)=d_G(s,v)$.
In oltre, per i grafi non pesati, e' vero che: \textit{l'albero SPT radicato in s e' uguale all'albero BFS radicato in s}.

Per trovare i cammini minimi si usa Dijkstra.
\subsection{Dijkstra}
\begin{enumerate}
    \item Mantengo per ogni nodo una \textbf{stima per eccesso} della distanza: $D_{sv}$
    \item Mantengo un insieme $X$ di nodi le cui stime sono \textbf{esatte}. Mantengo anche un'albero $T$
    che contiene i cammini minimi verso i nodi in $X$. Inizialmente: $X=\{s\}$, $T$ non ha archi.
    \item ad ogni passo aggiungo a $X$ un nodo $u$ con stima minima e aggiungo quell'arco a $T$.
    \item aggiorno le stime guardando i nodi adiacenti a $u$.
\end{enumerate}

la stima di un nodo $y$ e' $D_{sy} = min\{D_{sx} + w(x,y) : (x,y) \in E, x \in X\}$. 
E' una stima, perche' se poi troviamo un arco $(u,y)$ tale che: $D_{su} + w(u,y) < D_{sx} + w(x,y)$ allora
rimpiazziamo $(x,y)$ con $(u,y)$.

\textit{Le stime dei nodi si tengono in una coda con priorita.}

\begin{lstlisting}[mathescape=true]
    algoritmo Dijkstra($grafo$ $G$, $vertice$ $s$) $\to albero$
        for each(vertice $u$ in $G$) do $D_{su}$ = $+\infty$
        $\hat T$ = albero formato dal nodo $s$; $X=\emptyset$
        CodaPriorita S
        $D_{ss} = 0$
        S.insert(s,0)
        while(not S.isEmpty()) do
            u = S.deleteMin(); $X=X \cap \{u\}$;
            for each( $arco(u,v)$ in $G$) do
                if($D_{sv} = +infty$) then
                    S.insert(v, $D_{su}$ + w(u,v))
                    $D_{sv} = D_{su} + w(u,v)$
                    rendi $u$ padre di $v$ in $\hat T$
                else if($D_{su} + w(u,v) < D_{sv}$) then
                    S.decreaseKey(v, $D_{sv} - D_{su} - w(u,v)$)
                    $D_{sv} = D_{su} + w(u,v)$
                    rendi $u$ nuovo padre di $v$ in $\hat T$
        return $\hat T$
\end{lstlisting}

\begin{definition} Lemma \\
    Quando il nodo $v$ viene estratto dalla coda con priorita' vale che:
    \begin{itemize}
        \item $D_{sv} = d(s,v)$
        \item Il cammino da $s$ a $v$ nell'albero corrente ha costo $d(s,v)$
    \end{itemize}
\end{definition}

se ci fosse un nodo $y$ con $d(s,y)$ con costo minore, questo sarebbe nella coda delle priorita' 
e sarebbe gia stato estratto.

L'algoritmo esegue in tempo $O(m+nlogn)$